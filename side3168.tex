\documentclass[12pt,oneside,a4paper]{article}

\usepackage[utf8]{inputenc} % Lærer LaTeX at forstå unicode - HUSK at filen skal
% være unicode (UTF-8), standard i Linux, ikke i
% Win.

\usepackage[danish]{babel} % Så der fx står Figur og ikke Figure, Resumé og ikke
% Abstract etc. (god at have).

\usepackage{graphicx}
\usepackage{amsfonts}

%\renewcommand{\mid}[1]{{\rm E}\!\left[#1\right]}
\newcommand{\bas}{\begin{eqnarray*}}
\newcommand{\eas}{\end{eqnarray*}}

\begin{document}

\section*{Løsning til 'En besynderlig ligning' side 3168}
Vi skal løse ligningen
$$
2\sqrt{50+10x} + \sqrt{50-10x} = (15+7x) \sqrt{5-2x}
$$
Vestresiden er defineret og ikke-negativ for $-5\le x\le 5$. Højresiden er
defineret og ikke-negativ for $-15/7 \le x \le 5/2$. I det følgende antager vi
derfor, at $-15/7 \le x \le 5/2$.

Begge sider af ligningen kvadreres:
$$
250 + 30x + 40\sqrt{25-x^2} = (15+7x)^2(5-2x)
$$
Dette kan omskrives til
$$
40\left(\sqrt{25-x^2}-2x\right) = 7(25+14x)\left(5-x^2\right)
$$
Nu ganges med $\sqrt{25-x^2}+2x$ på begge sider:
$$
200(5-x^2) = 7(25+14x)\left(5-x^2\right)\left(\sqrt{25-x^2}+2x\right)
$$
Heraf ser vi, at $x=\sqrt{5}$ er en løsning.

Nu undersøges om der er andre løsninger.
Antag derfor, at $x\neq \sqrt{5}$. Så forkortes med $5-x^2$:
$$
200 = 7(25+14x)\left(\sqrt{25-x^2}+2x\right)
$$
Dette kan omskrives til
$$
\sqrt{25-x^2} = \frac{200}{7(25+14x)}-2x
$$

Højresiden er aftagende for $x<-25/14$ og for $-25/14<x$.
For $x=-15/7$ er højresiden negativ, og der er derfor ingen løsninger for $-15/7\le x<-25/14$.

I intervallet $-25/14<x\le 0$ er venstresiden voksende mod værdien $5$ og højresiden 
er aftagende fra plus uendelig mod $8/7$. Derfor vil der være netop én løsning
i dette interval.

I intervallet $0\le x \le 5/2$ er venstresiden aftagende mod værdien $5/2 \sqrt{3} \approx 4,\!3$ og højresiden er 
aftagende fra værdien $8/7$. Der kan derfor ikke være nogen løsninger i dette interval.

For at komme lidt nærmere, så kvadreres ligningen, og alle led flyttes over på venstresiden:
$$
49(25+14x)^2(25-x^2) - (200-350x-196x^2)^2 = 0
$$
Venstresiden udregnes nu til
$$
-5\left(9604x^4+34300x^3-33075x^2-199500x-145125\right)
$$

Dette fjerdegradspolynomium har op til fire rødder, men vi er kun interesseret
i intervallet $-25/14<x\le 0$.
Lad $p(x)$ være polynomiet i parentesen. Da får vi efter nogle gæt, at
$$
p(-1) = -3396
$$
og
$$
p(-8/7) = 4859
$$

Hermed kan vi konkludere, at den oprindelige ligning har i alt to løsninger, og disse er givet ved:
$$
x = \sqrt{5}
$$
og
$$
-8/7 < x < -1
$$

\end{document}


