\documentclass[12pt,oneside,a4paper]{article}

\usepackage[utf8]{inputenc} % Lærer LaTeX at forstå unicode - HUSK at filen skal
% være unicode (UTF-8), standard i Linux, ikke i
% Win.

\usepackage[danish]{babel} % Så der fx står Figur og ikke Figure, Resumé og ikke
% Abstract etc. (god at have).

%\usepackage{graphicx}
\usepackage{amsfonts}
\usepackage{amsthm}        % Theorems
\usepackage{amsmath}
%\usepackage{hyperref}

%\renewcommand{\mid}[1]{{\rm E}\!\left[#1\right]}
\newcommand{\bas}{\begin{eqnarray*}}
\newcommand{\eas}{\end{eqnarray*}}
\newcommand{\be}{\begin{equation}}
\newcommand{\ee}{\end{equation}}
\newcommand{\bea}{\begin{eqnarray}}
\newcommand{\eea}{\end{eqnarray}}

\newtheorem{thm}{Sætning}[section]
\newtheorem{mydef}[thm]{Definition}
\newtheorem{eks}[thm]{Eksempel}

\title{Løsning af trediegradsligninger}
\date{Marts 2017}
\author{Michael Jørgensen}

\begin{document}

\maketitle
\section*{Opgave 1}
Lad os først kigge på en speciel type af trediegradsligning, nemlig
hvor leddet $x^2$ ikke findes. Det vil sige, vil vi løse ligningen:
\begin{equation}
    x^3 + px +q = 0,
    \label{eq1}
\end{equation}
hvor $p$ og $q$ er givne tal. 

For at løse ligning~(\ref{eq1}), så indfører vi to nye endnu ukendte tal $u$ og
$v$, og postulerer, at $x=u+v$. Dvs vi søger nogle ligninger, som $u$ og $v$
skal opfylde.

For det første skal $x$ være en løsning til den givne
trediegradsligning~(\ref{eq1}).
\begin{itemize}
    \item Udregn først $x^3 = (u+v)^3$ ved at ophæve parenteserne.
    \item Vis dernæst ved indsættelse i ligning~(\ref{eq1}), at ligningen kan
        skrives som
        \begin{equation}
            u^3+v^3 + (3uv+p)x + q=0
            \label{eq2}
        \end{equation}
\end{itemize}

Ligning~(\ref{eq2}) er én ligning, som $u$ og $v$ skal opfylde. Men da vi har
to variabler, så kan vi frit finde på endnu en ligning, som de skal opfylde, og
stadig håbe på, at der er løsningner.

Den ekstra betingelse vi kræver er, at det midterste led i ligning~(\ref{eq2}) er nul, dvs.
\begin{equation}
    3uv+p = 0
    \label{eq3}
\end{equation}
\begin{itemize}
    \item Gør rede for, at ligning~(\ref{eq2}) og~(\ref{eq3}) tilsammen kan skrives som følgende ligningssystem:
        \begin{eqnarray}
            3uv+p = 0 \label{eq4}\\
            u^3+v^3+q = 0 \label{eq5}
        \end{eqnarray}
\end{itemize}

        I ligning~(\ref{eq4}) skal du nu isolere $v$, og derefter indsætte dette
        i ligning~(\ref{eq5}).
\begin{itemize}
    \item Vis, at den fremkomne ligning i $u$ kan skrives som
        \begin{equation}
            u^6 + qu^3-\Big(\frac{p}{3}\Big)^3 = 0 \;.
            \label{eq6}
        \end{equation}
\end{itemize}

Dette ligner en sjettegradsligning, men ved at substituere $z=u^3$ bliver
det til en andengradsligning.


\begin{itemize}
    \item Løs denne, og vis at ligning~(\ref{eq6}) har to løsninger:

        \begin{equation}
            u =
            \sqrt[3]{-\frac{q}{2}\pm\sqrt{\Big(\frac{q}{2}\Big)^2+\Big(\frac{p}{3}\Big)^3}} \;.
            \label{eq7}
        \end{equation}
    \item Gør rede for, at hvis de to løsninger~(\ref{eq7}) kaldes henholdsvis $u$ og $v$, så opfylder de både ligning~(\ref{eq4}) og ligning~(\ref{eq5}).
    \item Skriv den fulde løsning til trediegradsligningen~(\ref{eq1}).
    \item Hvad skal gælde om sammenhængen mellem $p$ og $q$ for at denne metode
        kan bruges (uden at bruge komplekse tal)?
\end{itemize}

\section*{Opgave 2}
Ovenstående metode giver én løsning, men en trediegradsligning kan godt have
tre løsninger. Den næste metode benytter trigonometriske funktioner. For at
bruge denne metode skal vi først have bevist nogle regneregler. Vi tager som
givet følgende additionsformler:
\begin{eqnarray}
    \sin(A+B) = \sin(A)\cos(B) + \cos(A)\sin(B) \\
    \cos(A+B) = \cos(A)\cos(B) - \sin(A)\sin(B)
\end{eqnarray}

Ovenstående formler gælder for alle værdier af $A$ og $B$.
\begin{itemize}
    \item Benyt ovenstående formler til at vise, at 
        \begin{eqnarray}
            \sin(2A) = 2\sin(A)\cos(A) \\
            \cos(2A) = \cos^2(A) - \sin^2(A)
        \end{eqnarray}
    \item Benyt ovenstående resultater til at vise, at
        \begin{equation}
            \sin(3A) = 3\sin(A)\cos^2(A) - \sin^3(A)
            \label{eq12}
        \end{equation}
\end{itemize}

For at komme videre får vi brug for endnu en formel (også kendt som
"idiotformlen"):
\begin{equation}
    \cos^2(A) + \sin^2(A) = 1 \;.
    \label{eq11}
\end{equation}

\begin{itemize}
    \item Benyt idiotformlen~(\ref{eq11}) og vis, at ligning~(\ref{eq12}) kan 
        omskrives til:
        \begin{equation}
            \sin(3A) = 3\sin(A) - 4\sin^3(A) \;.
            \label{eq13}
        \end{equation}
    \item Nu isoleres $\sin^3(A)$. Vis, at det giver:
        \begin{equation}
            \sin^3(A) = \frac{3}{4} \sin(A) - \frac{1}{4}\sin(3A)\;.
        \end{equation}
\end{itemize}

Nu er vi klar til at indsætte $x=u\sin(v)$ i ligning~(\ref{eq1}).

\begin{itemize}
    \item Vis, at ligning~(\ref{eq1}) kan omskrives til
        \begin{equation}
            -\frac{u^3}{4}\sin(3v) + \Big(\frac{3}{4}u^2+p\Big)u\sin(v)+q=0\;.
            \label{eq14}
        \end{equation}
\end{itemize}

Igen har vi én ligning med to ubekendte, og igen vælger vi at sætte
det midterste led lig med nul, dvs.
        \begin{equation}
            \frac{3}{4}u^2+p = 0\;.
            \label{eq15}
        \end{equation}

\begin{itemize}
    \item Vis, at ligning~(\ref{eq14}) nu reduceres til:
        \begin{equation}
            \sin(3v) = \frac{4q}{u^3} \;.
            \label{eq16}
        \end{equation}
    \item Isolér $u$ i ligning~(\ref{eq15}) og indsæt i ligning~(\ref{eq16}).
         Vis, at løsnigen kan skrives om:
         \begin{equation}
             3v = \sin^{-1}\Bigg(\frac{q/2}{\sqrt{-(p/3)^3}}\Bigg) + n\cdot 2\pi\;,
             \label{eq17}
         \end{equation}
         hvor $n$ er et vilkårligt heltal.
     \item Forklar, hvorfor $n=0$, $n=1$, og $n=2$ giver tre forskellige værdier for $v$, som alle ligger i intervallet $[0, 2\pi[$.
     \item Opskriv den fulde formel for de tre løsninger til den oprindelige trediegradsligning~(\ref{eq1}).
    \item Hvad skal gælde om sammenhængen mellem $p$ og $q$ for at denne metode
        kan bruges (uden at bruge komplekse tal)?
\end{itemize}

\section*{Opgave 3}
Til slut kigger vi på den generelle trediegradsligning:
\begin{equation}
    x^3 + ax^2 + bx + c = 0 \,,
    \label{eq21}
\end{equation}
hvor $a$, $b$ og $c$ er givne tal.

Nu indføres substitutionen
\begin{equation}
    x = u - \frac{a}{3}
    \label{eq22}
\end{equation}

\begin{itemize}
    \item Indsæt ligning~(\ref{eq22}) i ligning~(\ref{eq21}) og vis, at svaret
        kan skrives på formen~(\ref{eq1}).
    \item Udtryk $p$ og $q$ ved de givne tal $a$, $b$ og $c$.
\end{itemize}


\end{document}

