\documentclass[12pt,oneside,a4paper]{article}

\usepackage[utf8]{inputenc}
% Lærer LaTeX at forstå unicode - HUSK at filen skal
% være unicode (UTF-8), standard i Linux, ikke i
% Win.

%\usepackage{graphicx}
\usepackage{amsfonts}
\usepackage{amsthm}        % Theorems
\usepackage{amsmath}
%\usepackage{hyperref}

%\renewcommand{\mid}[1]{{\rm E}\!\left[#1\right]}
\newcommand{\bas}{\begin{eqnarray*}}
\newcommand{\eas}{\end{eqnarray*}}
\newcommand{\be}{\begin{equation}}
\newcommand{\ee}{\end{equation}}
\newcommand{\bea}{\begin{eqnarray}}
\newcommand{\eea}{\end{eqnarray}}

\newtheorem{thm}{Theorem}[section]
\newtheorem{mydef}[thm]{Definition}
\newtheorem{eks}[thm]{Example}

\title{Modular Forms}
\date{April 2020}
\author{Michael Jørgensen}

\begin{document}

\maketitle
\section{Jacobi Theta function}
We start by defining the Jacobi Theta function:

\begin{mydef}
$$
\Theta(z, \tau) = \sum_{n=-\infty}^{\infty} \exp(2\pi inz + \pi i n^2 \tau),
$$
where $z$ is any complex number and $\tau$ is confined to the upper
half-plane.
\end{mydef}

\subsection{Periodicity}

Obvious properties of this function is that it is periodic in $z$ with period 1, i.e.
$$
\Theta(z+1, \tau) = \Theta(z, \tau).
$$

Furthermore, it is quasi-periodic in $z$ with period $\tau$, i.e.
$$
\Theta(z+\tau, \tau) = \exp[-\pi i (\tau + 2z)] \Theta(z, \tau).
$$

\subsection{More properties}
$$
\Theta\left(\frac{z}{\tau}, -\frac{1}{\tau}\right) = (-i\tau)^{1/2} \exp(\frac{\pi}{\tau}iz^2) \Theta(z, \tau).
$$

\section{$j$-invariant}
The $j$-invariant is a function $j(\tau)$ defined on the upper half-plane as follows:
$$
j(\tau) = 12^3 \frac{g_2(\tau)^3}{g_2(\tau)^3 - 27 g_3(\tau)^2}
$$
where
$$
g_2(\tau) = 60 \sum_{(m,n)\neq(0,0)} (m+n\tau)^{-4}
$$
$$
g_3(\tau) = 140 \sum_{(m,n)\neq(0,0)} (m+n\tau)^{-6}
$$

The function satisfies
$$
j(\exp(2\pi i/3) = 0
$$
and
$$
j(i) = 12^3.
$$

The function is clearly periodic with period $1$. We may therefore substitute
$q=\exp(2\pi i \tau)$ and this leads to the Laurent series:

$$
j(\tau) = q^{-1} + 744 + 196884q + \ldots
$$

\end{document}

