\documentclass[12pt,oneside,a4paper]{article}

\usepackage[utf8]{inputenc} % Lærer LaTeX at forstå unicode - HUSK at filen skal
% være unicode (UTF-8), standard i Linux, ikke i
% Win.

\usepackage[danish]{babel} % Så der fx står Figur og ikke Figure, Resumé og ikke
% Abstract etc. (god at have).

%\renewcommand{\mid}[1]{{\rm E}\!\left[#1\right]}
\newcommand{\bas}{\begin{eqnarray*}}
\newcommand{\eas}{\end{eqnarray*}}

\begin{document}

\section{Opgave 31}
Find samtlige hele positive tal $n$, for hvilke $n^2+20n+15$ er et kvadrattal.

\section{Løsning}
Vi skal altså finde talpar $(m,n)$ således at 
$$
n^2+20n+15 = m^2
$$
Heraf følger specielt, at $m>n$.
Ligningen kan også omskrives til
$$
(n+10)^2-85 = m^2
$$
Heraf følger tilsvarende, at $m<n+10$.

Nu indfører vi tallet $k = m-n$. Så gælder der, at $0 < k < 10$. Indsættes nu $m=n+k$ i ligningen giver det:
$$
n^2+20n+15 = n^2+2nk + k^2
$$
Her kan $n$ isoleres:
$$
n = \frac{k^2-15}{2(10-k)}
$$
For at $n$ skal være et positivt heltal, så skal $k$ være et ulige tal større end 4 og mindre end 10. Det giver mulighederne $5$, $7$ og $9$.

Vi samler resultaterne i følgende tabel
$$
\begin{array}{cc}
    k & n \\
    \hline
    5 & 1 \\
    7 & \frac{17}{3} \\
    9 & 33
\end{array}
$$

Der er således to løsninger: $n=1$ og $n=33$.

\end{document}

