\documentclass[12pt,oneside,a4paper]{article}

\usepackage[utf8]{inputenc} % Lærer LaTeX at forstå unicode - HUSK at filen skal
% være unicode (UTF-8), standard i Linux, ikke i
% Win.

\usepackage[danish]{babel} % Så der fx står Figur og ikke Figure, Resumé og ikke
% Abstract etc. (god at have).

%\renewcommand{\mid}[1]{{\rm E}\!\left[#1\right]}
\newcommand{\bas}{\begin{eqnarray*}}
\newcommand{\eas}{\end{eqnarray*}}

\begin{document}

\section{Geometriopgave 100}

\section{Løsning}
Uden tab af generalitet kan vi sætte $a=1$.

Vi indfører et koordinatsystem med origo i punktet $A$ og med $x$-aksen i retning mod punktet $C$. Så har punket $P$ koordinaterne $(\frac 34, \frac 34)$ og den lille cirkel har radius $r=\frac 14$.

Tangenten $DQ$ har ligningen $-hx+y=0$, hvor $h>1$ er (den endnu ukendte) hældning af linjen $DQ$. Afstanden fra punktet $P$ til denne linje skal derfor være ilg med radius. Med formlen for afstand fra punkt til linje giver det følgende ligning:
$$
\frac{\left|-hP_x+P_y\right|}{\sqrt{h^2+1}} = r
$$

Vi indsætter nu de kendte tal og det giver:
$$
\frac 34 \frac{h-1}{\sqrt{h^2+1}} = \frac 14
$$
Dette omskrives til følgende andengradsligning:
$$
4h^2-9h+4=0
$$

Punktet $Q$ vil have koordinaterne $(\frac 1h, 1)$, og den søgte afstand $|DQ|$ bliver da:
$$
|DQ| = \sqrt{\frac{1}{h^2} + 1} = \frac{\sqrt{h^2+1}}{h}
$$
Ud fra andengradsligningen får vi
$$
h^2+1 = \frac 94 h
$$
Dermed kan den søgte afstand skrives som:
$$
|DQ| = \frac{3}{2} \frac{1}{\sqrt h}
$$

Ved at dividere andengradsligningen med $h$ får vi
$$
4h-9+\frac 4h=0
$$
Dermed har vi, at 
$$
\left(2\sqrt h - \frac{2}{\sqrt h}\right)^2 = 4h + \frac 4h - 8 = 1
$$
Da $h>1$ har vi altså
$$
2\sqrt h - \frac{2}{\sqrt h} = 1
$$

Heri sætter vi nu
$$
x= \frac{1}{\sqrt h}
$$
Det giver følgende nye andengradsligning (efter multiplikation med $x$):
$$
-2x^2-x+2=0
$$
Da vi må have $0<x<1$ så er løsningen givet ved:
$$
\frac{1}{\sqrt h} = x=\frac{\sqrt{17}-1}{4}
$$
Den søgte afstand er da:
$$
|DQ| = \frac{3}{8} \left(\sqrt{17}-1\right)
$$


\end{document}

