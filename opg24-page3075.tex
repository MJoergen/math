\documentclass[12pt,oneside,a4paper]{article}

\usepackage[utf8]{inputenc} % Lærer LaTeX at forstå unicode - HUSK at filen skal
% være unicode (UTF-8), standard i Linux, ikke i
% Win.

\usepackage[danish]{babel} % Så der fx står Figur og ikke Figure, Resumé og ikke
% Abstract etc. (god at have).

%\renewcommand{\mid}[1]{{\rm E}\!\left[#1\right]}
\newcommand{\bas}{\begin{eqnarray*}}
\newcommand{\eas}{\end{eqnarray*}}

\begin{document}

\section{Opgave 24}
Det oplyses, at
\bas
p+q+r &=& 0 \\
a+b+c &=& 0 \\
\frac{p}{a} + \frac{q}{b} + \frac{r}{c} &=& 0
\eas
Vis, at
$$
pa^2+qb^2+rc^2=0
$$

\section{Løsning}
Betragt matricen $M$ defineret ved:
$$
M = \left(\begin{array}{ccc}
    1 & 1 & 1 \\
    \frac{1}{a} & \frac{1}{b} & \frac{1}{c} \\
    a^2 & b^2 & c^2 
\end{array}
\right)
$$
Determinanten af denne udregnes til
$$
\det(M) = \frac{b^2-c^2}{a} + \frac{c^2-a^2}{b} + \frac{a^2-b^2}{c}
$$

Antag nu yderligere, at $a+b+c=0$. Så kan denne determinant omskrives på følgende måde:
\bas
\det(M) &=& \frac{(b+c)(b-c)}{a} + \frac{(c+a)(c-a)}{b} + \frac{(a+b)(a-b)}{c} \\
        &=& \frac{-a(b-c)}{a} + \frac{-b(c-a)}{b} + \frac{-c(a-b)}{c} \\
        &=& -(b-c) - (c-a) - (a-b) \\
        &=& 0
\eas
Der gælder altså, at hvis $a+b+c=0$, så er de tre rækker i matricen $M$ lineært afhængige. Specielt kan den tredje række skrives som en linearkombination af de to første rækker.
Heraf følger, at hvis $p+q+r=0$ og $\frac{p}{a} + \frac{q}{b} + \frac{r}{c}=0$, så vil der også gælde, at $pa^2+qb^2+rc^2=0$.

Ovenstående udregning er tilstrækkelig, men den præcise linearkombination kan også findes. Vi ønsker altså at bestemme to tal $m$ og $n$, således at
\bas
a^2 &=& m+\frac{n}{a} \\
b^2 &=& m+\frac{n}{b} \\
c^2 &=& m+\frac{n}{c} 
\eas
Ved at trække de to første ligninger fra hinanden får man
$$
a^2-b^2 = n\left(\frac{1}{a}-\frac{1}{b}\right) = n \frac{b-a}{ab}
$$
Heraf følger, at 
$$
n = -ab(a+b)
$$
Under antagelsen $a+b+c=0$ giver det
$$
n=abc
$$
Herefter finder man, at
$$
m=a^2-bc
$$
Denne sidste ligning man omskrives til en symmetrisk form (igen under antagelse af $a+b+c=0$) ved at erstatte $a$ med $-(b+c)$. Det giver:
$$
m=(b+c)^2-bc = b^2+c^2+bc
$$
Vi har to forskellige udtryk for $m$. Ved at lægge dem sammen og dividere med to fås det symmetriske udtryk:
$$
m=\frac{1}{2} (a^2+b^2+c^2)
$$
Hermed er vi nået frem til følgende ligning:
$$
\frac{1}{2} (a^2+b^2+c^2) (p+q+r) + abc \left(\frac{p}{a} + \frac{q}{b} + \frac{r}{c} \right) = pa^2+qb^2+rc^2
$$
som gælder blot $a+b+c=0$.


\end{document}

