\documentclass[12pt,oneside,a4paper]{article}

\usepackage[utf8]{inputenc} % Lærer LaTeX at forstå unicode - HUSK at filen skal
% være unicode (UTF-8), standard i Linux, ikke i
% Win.

\usepackage[danish]{babel} % Så der fx står Figur og ikke Figure, Resumé og ikke
% Abstract etc. (god at have).

%\usepackage{graphicx}
\usepackage{amsfonts}
\usepackage{amsthm}        % Theorems
\usepackage{amsmath}
%\usepackage{hyperref}

%\renewcommand{\mid}[1]{{\rm E}\!\left[#1\right]}
\newcommand{\bas}{\begin{eqnarray*}}
\newcommand{\eas}{\end{eqnarray*}}
\newcommand{\be}{\begin{equation}}
\newcommand{\ee}{\end{equation}}
\newcommand{\bea}{\begin{eqnarray}}
\newcommand{\eea}{\end{eqnarray}}

\newtheorem{thm}{Sætning}[section]
\newtheorem{mydef}[thm]{Definition}
\newtheorem{eks}[thm]{Eksempel}

\DeclareMathSymbol{,}{\mathord}{letters}{"3B}

\title{Løsning af fjerdegradsligninger}
\date{Marts 2017}
\author{Michael Jørgensen}

\begin{document}

\maketitle
Vi betragter det generelle fjerdegradspolynonium
\begin{equation}
    x^4 + ax^3 + bx^2 + cx + d\;,
\end{equation}
og kalder rødderne for $x_1$, $x_2$, $x_3$ og $x_4$.
Opgaven i det følgende er at bestemme rødderne ud fra kendskabet til
koefficienterne $a$, $b$, $c$ og $d$.
For at gøre dette, gør vi brug af gruppe-teori, teori om
symmetriske polynomier, samt flittig brug af CAS.
Endvidere vil vi også beregne diskriminanten $D$ af polynomiet, som 
er defineret ved
\begin{equation}
    D =
    \left[(x_1-x_2)(x_1-x_3)(x_1-x_4)(x_2-x_3)(x_2-x_4)(x_3-x_4)\right]^2\;.
    \label{eq_dis4}
\end{equation}
Vi slutter af med at gennemregne et eksempel.

\section{Sammenhæng mellem rødder og koefficienter}
Til at begynde med løser vi den modsatte opgave, nemlig at bestemme
koefficienterne ud fra rødderne. Dette gøres ved at faktorisere polynomiet:
\begin{equation}
    x^4 + ax^3 + bx^2 + cx + d = (x-x_1)(x-x_2)(x-x_3)(x-x_4)\;,
    \label{fakt}
\end{equation}
og så samle led med potenser af $x$.
Dette giver (vha CAS) følgende resultat:
\bea
  -a &=& x_1 + x_2 + x_3 + x_4 \label{koef_a} \\
   b &=& x_1x_2 + x_1x_3 + x_1x_4 + x_2x_3 + x_2x_4 + x_3x_4  \label{koef_b}\\
  -c &=& x_1x_2x_3 + x_1x_2x_4 + x_1x_3x_4 + x_2x_3x_4  \label{koef_c}\\
   d &=& x_1x_2x_3x_4 \label{koef_d}
\eea
I stedet for at løse det oprindelige fjerdegradspolynomium, så kan vi i stedet løse ovenstående fire ligninger med fire ubekendte. Desværre er ligningerne ikke lineære, så umiddelbart hjælper det os ikke.

\section{Permutationer af rødder}
Faktoriseringen~(\ref{fakt}) er tydeligvis invariant over for permutationer af
rødderne. Dvs hvis f.eks. $x_1$ og $x_2$ bytter plads, så vil faktorisering være
uændret, fordi faktorernes orden ikke betyder noget.
Heraf følger, at de fire koefficienter $a$, $b$, $c$ og $d$ således også er
invariante over for alle 24 permutationer af rødderne. Ved inspektion af
ligningerne~(\ref{koef_a})-(\ref{koef_d}) ser vi således, at vilkårlige
ombytninger af rødderne ikke ændrer værdierne af $a$, $b$, $c$ eller $d$.

Omvendt gælder der, at ethvert udtryk, som er invariant over for vilkårlige
permutationer af rødderne, vil kunne omskrives til et algebraisk udtryk i 
koefficienterne $a$, $b$, $c$ og $d$, og dette endda på en éntydig måde.
Vi kan f.eks. foretage følgende beregning:
\bas
 && a^2 \\
 &=& (x_1+x_2+x_3+x_4)^2  \\
 &=& x_1^2+x_2^2+x_3^2+x_4^2 + 2(x_1x_2+x_1x_3+x_1x_4 +x_2x_3+x_2x_4+x_3x_4)\\
 &=& x_1^2+x_2^2+x_3^2+x_4^2 + 2b
\eas
Dette viser, at 
\begin{equation}
    x_1^2 + x_2^2 + x_3^2 + x_4^2 = a^2 - 2b\;.
\end{equation}
Venstresiden er tydeligvis invariant over for permutationer af rødderne, og
kan derfor, som det ses, udtrykkes ved koefficienterne $a$, $b$, $c$ og $d$.

På tilsvarende møde udregner vi
\bas
 && -ab \\
 &=& -3c + x_1^2(x_2+x_3+x_4) + x_2^2(x_1+x_3+x_4) \\
 && \quad + x_3^2(x_1+x_2+x_4) + x_4^2(x_1+x_2+x_3) \\
 &=& -3c + x_1^2(-a-x_1) + x_2^2(-a-x_2) + x_3^2(-a-x_3) + x_4^2(-a-x_4) \\
 &=& -3c - a(x_1^2+x_2^2+x_3^2+x_4^2) - (x_1^3+x_2^3+x_3^3+x_4^3) \\
 &=& -3c - a(a^2-2b) - (x_1^3+x_2^3+x_3^3+x_4^3) \;.
\eas
Heraf følger, at 
\begin{equation}
x_1^3+x_2^3+x_3^3+x_4^3 = -a^3 + 3ab - 3c \;.
\end{equation}

\section{Substitution} \label{sec_subst}
For at komme videre, indfører vi en smart substitution:
\bea
    T_1 &=& x_1 + x_2 - x_3 - x_4 \label{subst_t1}\\
    T_2 &=& x_1 - x_2 + x_3 - x_4 \label{subst_t2}\\
    T_3 &=& x_1 - x_2 - x_3 + x_4 \label{subst_t3}\;.
\eea
At dette skulle være en smart substitution bliver først klart i
afsnit~\ref{sec_omskriv}.  Det afgørende er, at hvert udtryk består af fire
led; to negative og to positive.

Hver af de tre nye variabler udgør, sammen med ligning~(\ref{koef_a}), et
system af fire lineære ligninger med fire ubekendte.  Med andre ord, det er
muligt at udtrykke rødderne ud fra kendskab til $T_1$, $T_2$ og $T_3$.
Udregning på CAS giver følgende svar:
\bea
    4x_1 &=& -a + T_1 + T_2 + T_3 \label{x1} \\
    4x_2 &=& -a + T_1 - T_2 - T_3 \label{x2} \\
    4x_3 &=& -a - T_1 + T_2 - T_3 \label{x3} \\
    4x_4 &=& -a - T_1 - T_2 + T_3 \label{x4} \;.
\eea
I det følgende vil vi beskrive, hvorledes vi kan bestemme værdierne af de nye
variable $T_1$, $T_2$ og $T_3$, og således med ovenstående formler bestemme
rødderne.

\section{Resultat af perutationer af rødderne}
Vi undersøger, hvorledes permutationer af rødderne påvirker de nye variable
$T_1$, $T_2$ og $T_3$.

Vi ser først på den simple ombytning $x_1 \leftrightarrow x_2$. Det ses
umiddelbart ud fra ligningerne~(\ref{subst_t1})-(\ref{subst_t3}), at $T_1$ er uændret, mens $T_2 \rightarrow -T_3$ og $T_3 \rightarrow -T_2$.

Dernæst undersøger vi den cykliske ombytning af de første tre rødder $x_1$,
$x_2$ og $x_3$, givet ved: $x_1 \rightarrow x_2 \rightarrow x_3 \rightarrow
x_1$.  Ved inspektion ser vi, at $T_1 \rightarrow -T_3$, $T_2 \rightarrow T_1$
og $T_3 \rightarrow -T_2$.

Ved at fortsætte på samme måde og undersøge samtlige 24 permutationer af
rødderne kommer vi frem til følgende skema:

\begin{tabular}{|c|r|r|r|}
    \hline 
    -    & $ T_1$ & $ T_2$ & $ T_3$ \\
    \hline 
    (12) & $ T_1$ & $-T_3$ & $-T_2$ \\
    (13) & $-T_3$ & $ T_2$ & $-T_1$ \\
    (14) & $-T_2$ & $-T_1$ & $ T_3$ \\
    (23) & $ T_2$ & $ T_1$ & $ T_3$ \\
    (24) & $ T_3$ & $ T_2$ & $ T_1$ \\
    (34) & $ T_1$ & $ T_3$ & $ T_2$ \\
    \hline 
    (12)(34) & $ T_1$ & $-T_2$ & $-T_3$ \\
    (13)(24) & $-T_1$ & $ T_2$ & $-T_3$ \\
    (14)(23) & $-T_1$ & $-T_2$ & $ T_3$ \\
    \hline 
    (123) & $-T_3$ & $ T_1$ & $-T_2$ \\
    (124) & $ T_3$ & $-T_1$ & $-T_2$ \\
    (134) & $-T_2$ & $ T_3$ & $-T_1$ \\
    (234) & $ T_3$ & $ T_1$ & $ T_2$ \\
    (321) & $ T_2$ & $-T_3$ & $-T_1$ \\
    (421) & $ T_3$ & $-T_1$ & $-T_2$ \\
    (431) & $-T_2$ & $ T_3$ & $-T_1$ \\
    (432) & $ T_3$ & $ T_1$ & $ T_2$ \\
    \hline 
    (1234) & $-T_3$ & $-T_2$ & $ T_1$ \\
    (1243) & $-T_2$ & $ T_1$ & $-T_3$ \\
    (1324) & $-T_1$ & $-T_3$ & $ T_2$ \\
    (1342) & $ T_2$ & $-T_1$ & $-T_3$ \\
    (1423) & $-T_1$ & $ T_3$ & $-T_2$ \\
    (1432) & $ T_3$ & $-T_2$ & $-T_1$ \\
    \hline
\end{tabular}

Vi ser således, at det hver gang fører til permutationer af de nye
variabler, samt eventuelle fortegnsskift. Endvidere ser vi, at produktet
$T_1 T_2 T_3$ er invariant over for alle permutationer. Det er derfor
muligt at udtrykke dette produkt udelukkende ved de oprindelige koefficienter.
Vi udregner derfor:
\bas
&& T_1 T_2 \\
&=& (x_1-x_4+x_2-x_3)(x_1-x_4-x_2+x_3) \\
&=& (x_1-x_4)^2 - (x_2-x_3)^2
\eas
Dernæst udregner vi:
\bas
&& T_1T_2T_3 \\
&=& (x_1+x_4-x_2-x_3)((x_1-x_4)^2 - (x_2-x_3)^2) \\
&=& (x_1-x-4)(x_1^2-x_4^2) + (x_2-x_3)(x_2^2-x_3^2) \\
&& \quad - (x_1+x_4)(x_2-x_3)^2 - (x_2+x_3)(x_1-x_4)^2 \\
&=& x_1^3 - x_1^2x_4 - x_1x_4^2 + x_4^3 + x_2^3 - x_2^2x_3 - x_2x_3^2 + x_3^3 \\
&& \quad - x_1x_2^2 + 2x_1x_2x_3 - x_1x_3^2 - x_4x_2^2 + 2x_2x_3x_4 - x_4x_3^2 \\
&& \quad - x_2x_1^2 + 2x_1x_2x_4 - x_2x_4^2 - x_3x_1^2 + 2x_1x_3x_4 - x_3x_4^2 \\
&=& -a^3 + 3ab - 3c - 2c - x_1^2(x_2+x_3+x_4) - x_2^2(x_1+x_3+x_4) \\
&& \quad - x_3^2(x_1+x_2+x_4) - x_4^2(x_1+x_2+x_3) \\
&=& -a^3 + 3ab - 5c - x_1^2(-a-x_1) - x_2^2(-a-x_2) \\
&& \quad - x_3^2(-a-x_3) - x_4^2(-a-x_4) \\
&=& -a^3 + 3ab - 5c + a(x_1^2+x_2^2+x_3^2+x_4^2) + x_1^3+x_2^3+x_3^3+x_4^3 \\
&=& -a^3 + 3ab - 5c + a(a^2-2b) - a^3 + 3ab - 3c \\
&=& -a^3 + 4ab - 8c \;.
\eas
Hermed har vi vist, at
\begin{equation}
T_1T_2T_3 = -a^3 + 4ab - 8c \;.
\label{t123}
\end{equation}

\section{Endnu en substitution}
Vi kan eliminere fortegnsskiftene ved at kvadrere de nye variable. Dermed
kommer vi frem til resultatet, at permutationer af rødderne fører til
permutationer af tallene $T_1^2$, $T_2^2$ og $T_3^2$. Det kan udtrykkes på den
måde, at følgende mængde af tal er invariant over for vilkårlige permutationer
af rødderne:
\begin{equation}
    \{T_1^2,\, T_2^2,\, T_3^2\}\;.
    \label{n_set}
\end{equation}
Vi foretager nu følgende udregning:
\bas
&& T_1^2 \\
&=& (x_1+x_2-x_3-x_4)^2 \\
&=& x_1^2+x_2^2+x_3^2+x_4^2 + 2(x_1x_2+x_3x_4-x_1x_3-x_1x_4-x_2x_3-x_2x_4) \\
&=& a^2-2b + 4(x_1x_2+x_3x_4) - 2b \\
&=& a^2-4b + 4u_1 \;.
\eas
hvor $u_1 = x_1x_2+x_3x_4$.
På tilsvarende måde får vi følgende resultat
\bea
T_1^2 &=& a^2-4b + 4u_1 \label{eq_t1}\\
T_2^2 &=& a^2-4b + 4u_2 \label{eq_t2}\\
T_3^2 &=& a^2-4b + 4u_3 \label{eq_t3}\;,
\eea
hvor vi har indført de nye variabler
\bea
u_1 &=& x_1x_2 + x_3x_4 \label{eq_u1}\\
u_2 &=& x_1x_3 + x_2x_4 \label{eq_u2}\\
u_3 &=& x_1x_4 + x_2x_3 \label{eq_u3}\;.
\eea

I det følgende vil vi udlede en ligning til at bestemme værdierne af $u_1$,
$u_2$ og $u_3$. Ved hjælp af ligningerne~(\ref{eq_t1})-(\ref{eq_t3}) kan vi 
så bestemme værdierne af $T_1$, $T_2$ og $T_3$ og derefter rødderne $x_1$, 
$x_2$, $x_3$ og $x_4$.

Dog er der ved løsning af ligningerne~(\ref{eq_t1})-(\ref{eq_t3}) tre fortegn,
der ikke er bestemt, og dermed i alt otte kombinationer af fortegn.
For at undgå tvetydighed er det nødvendigt med en ekstra betingelse på 
værdierne af $T_1$, $T_2$ og $T_3$. Denne betingelse er netop ligning~(\ref{t123}).

\section{Diskriminanten}
For at komme videre med diskriminanten udregner vi faktorerne parvis.
Først regner vi:
\bea
&& (x_1-x_2)(x_3-x_4) \\
&=& x_1x_3-x_1x_4-x_2x_3+x_2x_4 \\
&=& u_2 - u_3\;.
\eea
På tilsvarende vis finder vi, at 
\bea
(x_1-x_3)(x_2-x_4) &=& u_1 - u_3 \\
(x_1-x_4)(x_2-x_3) &=& u_1 - u_2 \;.
\eea

Dette giver følgende formel for diskriminanten
\begin{equation}
    D = \left[(u_1-u_2)(u_1-u_3)(u_2-u_3)\right]^2 \;.
    \label{eq_dis3}
\end{equation}


\section{Omskrivning til tredjegradsligning} \label{sec_omskriv}
Vi opskriver nu en tredjegradsligning i $u$, som har rødderne $u_1$, $u_2$ og 
$u_3$.
\begin{equation}
    (u-u_1)(u-u_2)(u-u_3)=0 \;.
    \label{tredje_1}
\end{equation}
Ved at gange parenteserne ud kan vi også skrive det på følgende måde:
\begin{equation}
    u^3 + pu^2+ qu + r = 0 \;,
    \label{tredje_2}
\end{equation}
hvor koefficienterne $p$, $q$ og $r$ kan udtrykkes ved rødderne $u_1$, $u_2$ og $u_3$ på følgende måde:
\bea
-p &=& u_1 + u_2 + u_3 \\
 q &=& u_1u_2 + u_1u_3 + u_2u_3 \\
-r &=& u_1u_2u_3 \,.
\eea
Nu ser vi, at koefficienterne $p$, $q$ og $r$ er invariante over for
permutationer af $u_1$, $u_2$ og $u_3$. Dermed er de også invariante over for
permutaioner af $x_1$, $x_2$, $x_3$ og $x_4$.  Så må der gælde, at de kan
udtrykkes ved de oprindelige koefficienter. Og derfor er substitutionen i
afsnit~\ref{sec_subst} smart!

Vi foretager nu følgende beregninger:
\bas
&& -p \\
&=& u_1+u_2+u_3 \\
&=& x_1x_2+x_3x_4 + x_1x_3+x_2x_4 + x_1x_4+x_2x_3 \\
&=& b\;.
\eas
\bas
&& q \\
&=& u_1u_2 + u_1u_3 + u_2u_3 \\
&=& x_1^2x_2x_3 + x_2^2x_1x_4 + x_3^2x_1x_4 + x_4^2x_2x_3 \\
&& \quad +x_1^2x_2x_4 + x_2^2x_1x_3 + x_3^2x_2x_4 + x_4^2x_1x_3 \\
&& \quad +x_1^2x_3x_4 + x_2^2x_3x_4 + x_3^2x_1x_2 + x_4^2x_1x_2 \\
&=& x_1(-c-x_2x_3x_4) + x_2(-c-x_1x_3x_4) \\
&& \quad + x_3(-c-x_1x_2x_4) + x_4(-c-x_1x_2x_3) \\
&=& ac-4d \;.
\eas
\bas
&& -r \\
&=& u_1u_2u_3 \\
&=& x_1^3x_2x_3x_4 + x_1^2x_2^2x_4^2 + x_1^2x_3^2x_4^2 + x_1x_2x_3x_4^3 \\
&& \quad + x_1^2x_2^2x_3^2 + x_2^3x_1x_3x_4 + x_3^3x_1x_2x_4 + x_2^2x_3^2x_4^2 \\
&=& d(x_1^2+x_2^2+x_3^2+x_4^2) + c^2 - 2\left(x_1^2x_2^2x_3x_4 + x_1^2x_3^2x_2x_4 \right.\\
&& \quad \left. + x_2^2x_3^2x_1x_4 + x_1^2x_4^2x_2x_3 + x_2^2x_4^2x_1x_3 + x_3^2x_4^2x_1x_2\right) \\
&=& d(a^2-2b) + c^2 - 2db \\
&=& a^2d - 4bd + c^2 \;.
\eas
Vi har således vist, at
\bea
-p &=& b \label{eq_p}\\
 q &=& ac-4d \label{eq_q}\\
-r &=& a^2d-4bd+c^2 \label{eq_r}\;.
\eea

På denne måde kan man, ved at løse en tredjegradsligningen~(\ref{tredje_2}),
bestemme værdierne for $u_1$, $u_2$ og $u_3$. Ved at tage kvadratroden af disse
tal bliver der i alt otte kombinationer af fortegn for $T_1$, $T_2$ og $T_3$,
som dog også skal opfylde~(\ref{t123}).


\section{Eksempel}
Som eksempel på anvendelsen af denne teori, bliver følgende
ligning gennemregnet:
\begin{equation}
    x^4 - 3,3 x^3 - 4,21 x^2 + 12,633 x - 4,914 = 0 \;.
\end{equation}

I vort eksempel giver ligningerne~(\ref{eq_p})-(\ref{eq_r}) følgende:
\bea
-p &=& 66,35 \\
 q &=& 1020,3859 \\
-r &=& 91,298025\;,
\eea
og vi skal derfor løse følgende tredjegradsligning:
\[
    u^3 - 66,35 u^2 + 1020,3859 u - 91,298025 = 0\;.
\]
Her finder vi rødderne
\bea
u_1 &=& 0,09 \\
u_2 &=& 24,01 \\
u_3 &=& 42,25
\eea
Endvidere har vi fra ligning~(\ref{t123}), at
\begin{equation}
T_1 T_2 T_3 = -9,555\;.
\label {t123_e}
\end{equation}
Som kontrol har vi, at 
\begin{equation}
    u_1u_2u_3 = 91,298025 = (-9,555)^2
\end{equation}
Vi vælger den positive kvadratrod for $T_1$ og $T_2$ og må derfor, på grund af
ligning~(\ref{t123_e}) vælge den negative rod for $T_3$.
Det giver:
\bea
T_1 &=& 0,3 \\
T_2 &=& 4,9 \\
T_3 &=& -6,5
\eea
Dermed bliver rødderne ifølge~(\ref{x1})-(\ref{x4}):
\bea
x_1 &=& \frac14(3,3+0,3+4,9-6,5) = 0,5 \\
x_2 &=& \frac14(3,3+0,3-4,9+6,5) = 1,3 \\
x_3 &=& \frac14(3,3-0,3+4,9+6,5) = 3,6 \\
x_4 &=& \frac14(3,3-0,3-4,9-6,5) = -2,1
\eea
Ved indsættelse ses disse tal at passe.


\end{document}

