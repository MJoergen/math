\documentclass[12pt,oneside,a4paper]{article}

\usepackage[utf8]{inputenc} % Lærer LaTeX at forstå unicode - HUSK at filen skal
% være unicode (UTF-8), standard i Linux, ikke i
% Win.

\usepackage[danish]{babel} % Så der fx står Figur og ikke Figure, Resumé og ikke
% Abstract etc. (god at have).

%\usepackage{graphicx}
\usepackage{amsfonts}
\usepackage{amsthm}        % Theorems
\usepackage{amsmath}
\usepackage{tcolorbox}
%\usepackage{hyperref}

%\renewcommand{\mid}[1]{{\rm E}\!\left[#1\right]}
\newcommand{\bas}{\begin{eqnarray*}}
\newcommand{\eas}{\end{eqnarray*}}
\newcommand{\be}{\begin{equation}}
\newcommand{\ee}{\end{equation}}
\newcommand{\bea}{\begin{eqnarray}}
\newcommand{\eea}{\end{eqnarray}}

\newtheorem{thm}{Sætning}[section]
\newtheorem{mydef}[thm]{Definition}
\newtheorem{eks}[thm]{Eksempel}

\DeclareMathSymbol{,}{\mathord}{letters}{"3B}

\title{Løsning af fjerdegradsligninger}
\date{Marts 2017}
\author{Michael Jørgensen}

\begin{document}

\maketitle
Vi betragter det generelle fjerdegradspolynonium
\begin{equation}
x^4 + ax^3 + bx^2 + cx + d\;,
\label{eq_4}
\end{equation}
og kalder rødderne for $x_1$, $x_2$, $x_3$ og $x_4$.
Opgaven i det følgende er at bestemme rødderne ud fra kendskabet til
koefficienterne $a$, $b$, $c$ og $d$.
For at gøre dette, gør vi brug af gruppe-teori, teori om
symmetriske polynomier, samt flittig brug af CAS.
Endvidere vil vi også beregne diskriminanten $D$ af polynomiet, som 
er defineret ved
\begin{equation}
    D =
    \left[(x_1-x_2)(x_1-x_3)(x_1-x_4)(x_2-x_3)(x_2-x_4)(x_3-x_4)\right]^2\;.
    \label{eq_dis4}
\end{equation}
Vi slutter af med at gennemregne et eksempel.

\section{Sammenhæng mellem rødder og koefficienter}
Til at begynde med løser vi den modsatte opgave, nemlig at bestemme
koefficienterne ud fra rødderne. Dette gøres ved at faktorisere polynomiet:
\begin{equation}
    x^4 + ax^3 + bx^2 + cx + d = (x-x_1)(x-x_2)(x-x_3)(x-x_4)\;,
    \label{fakt}
\end{equation}
og så samle led med potenser af $x$.
Dette giver (vha CAS) følgende resultat:
\bea
  -a &=& x_1 + x_2 + x_3 + x_4 \label{koef_a} \\
   b &=& x_1x_2 + x_1x_3 + x_1x_4 + x_2x_3 + x_2x_4 + x_3x_4  \label{koef_b}\\
  -c &=& x_1x_2x_3 + x_1x_2x_4 + x_1x_3x_4 + x_2x_3x_4  \label{koef_c}\\
   d &=& x_1x_2x_3x_4 \label{koef_d}
\eea
I stedet for at løse det oprindelige fjerdegradspolynomium, så kan vi i stedet løse ovenstående fire ligninger med fire ubekendte. Desværre er ligningerne ikke lineære, så umiddelbart hjælper det os ikke.

\section{Permutationer af rødder}
Faktoriseringen~(\ref{fakt}) er tydeligvis invariant over for permutationer af
rødderne. Dvs hvis f.eks. $x_1$ og $x_2$ bytter plads, så vil faktorisering være
uændret, fordi faktorernes orden ikke betyder noget.
Heraf følger, at de fire koefficienter $a$, $b$, $c$ og $d$ således også er
invariante over for alle 24 permutationer af rødderne. Ved inspektion af
ligningerne~(\ref{koef_a})-(\ref{koef_d}) ser vi således, at vilkårlige
ombytninger af rødderne ikke ændrer værdierne af $a$, $b$, $c$ eller $d$.

Omvendt gælder der, at ethvert udtryk, som er invariant over for vilkårlige
permutationer af rødderne, vil kunne omskrives til et algebraisk udtryk i 
koefficienterne $a$, $b$, $c$ og $d$, og dette endda på en éntydig måde.
Vi kan f.eks. foretage følgende beregning:
\bas
 && a^2 \\
 &=& (x_1+x_2+x_3+x_4)^2  \\
 &=& x_1^2+x_2^2+x_3^2+x_4^2 + 2(x_1x_2+x_1x_3+x_1x_4 +x_2x_3+x_2x_4+x_3x_4)\\
 &=& x_1^2+x_2^2+x_3^2+x_4^2 + 2b
\eas
Dette viser, at 
\begin{equation}
    x_1^2 + x_2^2 + x_3^2 + x_4^2 = a^2 - 2b\;.
\end{equation}
Venstresiden er tydeligvis invariant over for permutationer af rødderne, og
kan derfor, som det ses, udtrykkes ved koefficienterne $a$, $b$, $c$ og $d$.

På tilsvarende måde udregner vi
\bas
 && -ab \\
 &=& -3c + x_1^2(x_2+x_3+x_4) + x_2^2(x_1+x_3+x_4) \\
 && \quad + x_3^2(x_1+x_2+x_4) + x_4^2(x_1+x_2+x_3) \\
 &=& -3c + x_1^2(-a-x_1) + x_2^2(-a-x_2) + x_3^2(-a-x_3) + x_4^2(-a-x_4) \\
 &=& -3c - a(x_1^2+x_2^2+x_3^2+x_4^2) - (x_1^3+x_2^3+x_3^3+x_4^3) \\
 &=& -3c - a(a^2-2b) - (x_1^3+x_2^3+x_3^3+x_4^3) \;.
\eas
Heraf følger, at 
\begin{equation}
x_1^3+x_2^3+x_3^3+x_4^3 = -a^3 + 3ab - 3c \;.
\end{equation}

\section{Substitution} \label{sec_subst}
For at komme videre, indfører vi en smart substitution:
\bea
    T_1 &=& x_1 + x_2 - x_3 - x_4 \label{subst_t1}\\
    T_2 &=& x_1 - x_2 + x_3 - x_4 \label{subst_t2}\\
    T_3 &=& x_1 - x_2 - x_3 + x_4 \label{subst_t3}\;.
\eea
At dette skulle være en smart substitution bliver først klart i
afsnit~\ref{sec_omskriv}.  Det afgørende er, at hvert udtryk består af fire
led; to negative og to positive.

Hver af de tre nye variabler udgør, sammen med ligning~(\ref{koef_a}), et
system af fire lineære ligninger med fire ubekendte.  Med andre ord, det er
muligt at udtrykke rødderne ud fra kendskab til $T_1$, $T_2$ og $T_3$.
Udregning på CAS giver følgende svar:
\begin{tcolorbox}
\bea
    4x_1 &=& -a + T_1 + T_2 + T_3 \label{x1} \\
    4x_2 &=& -a + T_1 - T_2 - T_3 \label{x2} \\
    4x_3 &=& -a - T_1 + T_2 - T_3 \label{x3} \\
    4x_4 &=& -a - T_1 - T_2 + T_3 \label{x4} \;.
\eea
\end{tcolorbox}
I det følgende vil vi beskrive, hvorledes vi kan bestemme værdierne af de nye
variable $T_1$, $T_2$ og $T_3$, og således med ovenstående formler bestemme
rødderne.

\section{Resultat af permutationer af rødderne}
Vi undersøger, hvorledes permutationer af rødderne påvirker de nye variable
$T_1$, $T_2$ og $T_3$.

Vi ser først på den simple ombytning $x_1 \leftrightarrow x_2$. Det ses
umiddelbart ud fra ligningerne~(\ref{subst_t1})-(\ref{subst_t3}), at $T_1$ er uændret, mens $T_2 \rightarrow -T_3$ og $T_3 \rightarrow -T_2$.

Dernæst undersøger vi den cykliske ombytning af de første tre rødder $x_1$,
$x_2$ og $x_3$, givet ved: $x_1 \rightarrow x_2 \rightarrow x_3 \rightarrow
x_1$.  Ved inspektion ser vi, at $T_1 \rightarrow -T_3$, $T_2 \rightarrow T_1$
og $T_3 \rightarrow -T_2$.

Ved at fortsætte på samme måde og undersøge samtlige 24 permutationer af
rødderne kommer vi frem til følgende skema:

\begin{tabular}{|c|r|r|r|}
    \hline 
    -    & $ T_1$ & $ T_2$ & $ T_3$ \\
    \hline 
    (12) & $ T_1$ & $-T_3$ & $-T_2$ \\
    (13) & $-T_3$ & $ T_2$ & $-T_1$ \\
    (14) & $-T_2$ & $-T_1$ & $ T_3$ \\
    (23) & $ T_2$ & $ T_1$ & $ T_3$ \\
    (24) & $ T_3$ & $ T_2$ & $ T_1$ \\
    (34) & $ T_1$ & $ T_3$ & $ T_2$ \\
    \hline 
    (12)(34) & $ T_1$ & $-T_2$ & $-T_3$ \\
    (13)(24) & $-T_1$ & $ T_2$ & $-T_3$ \\
    (14)(23) & $-T_1$ & $-T_2$ & $ T_3$ \\
    \hline 
    (123) & $-T_3$ & $ T_1$ & $-T_2$ \\
    (124) & $ T_3$ & $-T_1$ & $-T_2$ \\
    (134) & $-T_2$ & $ T_3$ & $-T_1$ \\
    (234) & $ T_3$ & $ T_1$ & $ T_2$ \\
    (321) & $ T_2$ & $-T_3$ & $-T_1$ \\
    (421) & $ T_3$ & $-T_1$ & $-T_2$ \\
    (431) & $-T_2$ & $ T_3$ & $-T_1$ \\
    (432) & $ T_3$ & $ T_1$ & $ T_2$ \\
    \hline 
    (1234) & $-T_3$ & $-T_2$ & $ T_1$ \\
    (1243) & $-T_2$ & $ T_1$ & $-T_3$ \\
    (1324) & $-T_1$ & $-T_3$ & $ T_2$ \\
    (1342) & $ T_2$ & $-T_1$ & $-T_3$ \\
    (1423) & $-T_1$ & $ T_3$ & $-T_2$ \\
    (1432) & $ T_3$ & $-T_2$ & $-T_1$ \\
    \hline
\end{tabular}

Vi ser således, at enhver permutation af de fire rødder fører til permutationer
af de nye variabler, med eventuelle fortegnsskift. Endvidere ser vi, at
produktet $T_1 T_2 T_3$ er invariant over for alle permutationer. Det er derfor
muligt at udtrykke dette produkt udelukkende ved de oprindelige koefficienter.
Vi udregner derfor:
\bas
&& T_1 T_2 \\
&=& (x_1-x_4+x_2-x_3)(x_1-x_4-x_2+x_3) \\
&=& (x_1-x_4)^2 - (x_2-x_3)^2 \;.
\eas
Dernæst udregner vi:
\bas
&& T_1T_2T_3 \\
&=& (x_1+x_4-x_2-x_3)((x_1-x_4)^2 - (x_2-x_3)^2) \\
&=& (x_1-x_4)(x_1^2-x_4^2) + (x_2-x_3)(x_2^2-x_3^2) \\
&& \quad - (x_1+x_4)(x_2-x_3)^2 - (x_2+x_3)(x_1-x_4)^2 \\
&=& x_1^3 - x_1^2x_4 - x_1x_4^2 + x_4^3 + x_2^3 - x_2^2x_3 - x_2x_3^2 + x_3^3 \\
&& \quad - x_1x_2^2 + 2x_1x_2x_3 - x_1x_3^2 - x_4x_2^2 + 2x_2x_3x_4 - x_4x_3^2 \\
&& \quad - x_2x_1^2 + 2x_1x_2x_4 - x_2x_4^2 - x_3x_1^2 + 2x_1x_3x_4 - x_3x_4^2 \\
&=& -a^3 + 3ab - 3c - 2c - x_1^2(x_2+x_3+x_4) - x_2^2(x_1+x_3+x_4) \\
&& \quad - x_3^2(x_1+x_2+x_4) - x_4^2(x_1+x_2+x_3) \\
&=& -a^3 + 3ab - 5c - x_1^2(-a-x_1) - x_2^2(-a-x_2) \\
&& \quad - x_3^2(-a-x_3) - x_4^2(-a-x_4) \\
&=& -a^3 + 3ab - 5c + a(x_1^2+x_2^2+x_3^2+x_4^2) + x_1^3+x_2^3+x_3^3+x_4^3 \\
&=& -a^3 + 3ab - 5c + a(a^2-2b) - a^3 + 3ab - 3c \\
&=& -a^3 + 4ab - 8c \;.
\eas
Hermed har vi vist, at
\begin{tcolorbox}
\begin{equation}
T_1T_2T_3 = -a^3 + 4ab - 8c \;.
\label{t123}
\end{equation}
\end{tcolorbox}

\section{Endnu en substitution}
Vi kan eliminere fortegnsskiftene ved at kvadrere de nye variable. Dermed
kommer vi frem til resultatet, at permutationer af rødderne fører til
permutationer af tallene $T_1^2$, $T_2^2$ og $T_3^2$. Det kan udtrykkes på den
måde, at følgende mængde af tal er invariant over for vilkårlige permutationer
af rødderne:
\begin{equation}
    \{T_1^2,\, T_2^2,\, T_3^2\}\;.
    \label{n_set}
\end{equation}
Vi foretager derfor følgende udregning:
\bas
&& T_1^2 \\
&=& (x_1+x_2-x_3-x_4)^2 \\
&=& x_1^2+x_2^2+x_3^2+x_4^2 + 2(x_1x_2+x_3x_4-x_1x_3-x_1x_4-x_2x_3-x_2x_4) \\
&=& a^2-2b + 4(x_1x_2+x_3x_4) - 2b \\
&=& a^2-4b + 4u_1 \;.
\eas
hvor $u_1 = x_1x_2+x_3x_4$.
På tilsvarende måde får vi følgende resultat
\begin{tcolorbox}
\bea
T_1^2 &=& a^2-4b + 4u_1 \label{eq_t1}\\
T_2^2 &=& a^2-4b + 4u_2 \label{eq_t2}\\
T_3^2 &=& a^2-4b + 4u_3 \label{eq_t3}\;,
\eea
\end{tcolorbox}
hvor vi har indført de nye variabler
\bea
u_1 &=& x_1x_2 + x_3x_4 \label{eq_u1_def}\\
u_2 &=& x_1x_3 + x_2x_4 \label{eq_u2_def}\\
u_3 &=& x_1x_4 + x_2x_3 \label{eq_u3_def}\;.
\eea

I det følgende vil vi udlede en ligning til at bestemme værdierne af $u_1$,
$u_2$ og $u_3$. Ved hjælp af ligningerne~(\ref{eq_t1})-(\ref{eq_t3}) kan vi 
så bestemme værdierne af $T_1$, $T_2$ og $T_3$ og derefter rødderne $x_1$, 
$x_2$, $x_3$ og $x_4$.

Dog er der ved løsning af ligningerne~(\ref{eq_t1})-(\ref{eq_t3}) tre fortegn,
der ikke er bestemt, og dermed i alt otte kombinationer af fortegn.
For at undgå tvetydighed er det nødvendigt at benytte ligning~(\ref{t123}).

\section{Omskrivning til tredjegradsligning} \label{sec_omskriv}
Vi opskriver nu en tredjegradsligning i $u$, som har rødderne $u_1$, $u_2$ og 
$u_3$.
\begin{equation}
    (u-u_1)(u-u_2)(u-u_3)=0 \;.
    \label{tredje_1}
\end{equation}
Ved at gange parenteserne ud kan vi også skrive det på følgende måde:
\begin{equation}
    u^3 + pu^2+ qu + r = 0 \;,
    \label{tredje_2}
\end{equation}
hvor koefficienterne $p$, $q$ og $r$ kan udtrykkes ved rødderne $u_1$, $u_2$ og $u_3$ på følgende måde:
\bea
-p &=& u_1 + u_2 + u_3 \label{equ_p}\\
 q &=& u_1u_2 + u_1u_3 + u_2u_3 \label{equ_q}\\
-r &=& u_1u_2u_3 \label{equ_r}\,.
\eea
Nu ser vi, at koefficienterne $p$, $q$ og $r$ er invariante over for
permutationer af $u_1$, $u_2$ og $u_3$. Dermed er de også invariante over for
permutaioner af $x_1$, $x_2$, $x_3$ og $x_4$, og der må derfor gælde, at $p$,
$q$ og $r$ kan udtrykkes ved de oprindelige koefficienter. Og derfor er
substitutionen i afsnit~\ref{sec_subst} smart!

Vi foretager nu følgende beregninger:
\bas
&& -p \\
&=& u_1+u_2+u_3 \\
&=& x_1x_2+x_3x_4 + x_1x_3+x_2x_4 + x_1x_4+x_2x_3 \\
&=& b\;.
\eas
\bas
&& q \\
&=& u_1u_2 + u_1u_3 + u_2u_3 \\
&=& x_1^2x_2x_3 + x_2^2x_1x_4 + x_3^2x_1x_4 + x_4^2x_2x_3 \\
&& \quad +x_1^2x_2x_4 + x_2^2x_1x_3 + x_3^2x_2x_4 + x_4^2x_1x_3 \\
&& \quad +x_1^2x_3x_4 + x_2^2x_3x_4 + x_3^2x_1x_2 + x_4^2x_1x_2 \\
&=& x_1(-c-x_2x_3x_4) + x_2(-c-x_1x_3x_4) \\
&& \quad + x_3(-c-x_1x_2x_4) + x_4(-c-x_1x_2x_3) \\
&=& ac-4d \;.
\eas
\bas
&& -r \\
&=& u_1u_2u_3 \\
&=& x_1^3x_2x_3x_4 + x_1^2x_2^2x_4^2 + x_1^2x_3^2x_4^2 + x_1x_2x_3x_4^3 \\
&& \quad + x_1^2x_2^2x_3^2 + x_2^3x_1x_3x_4 + x_3^3x_1x_2x_4 + x_2^2x_3^2x_4^2 \\
&=& d(x_1^2+x_2^2+x_3^2+x_4^2) + c^2 - 2\left(x_1^2x_2^2x_3x_4 + x_1^2x_3^2x_2x_4 \right.\\
&& \quad \left. + x_2^2x_3^2x_1x_4 + x_1^2x_4^2x_2x_3 + x_2^2x_4^2x_1x_3 + x_3^2x_4^2x_1x_2\right) \\
&=& d(a^2-2b) + c^2 - 2db \\
&=& a^2d - 4bd + c^2 \;.
\eas
Vi har således vist, at
\begin{tcolorbox}
\bea
p &=& -b \label{eq_p}\\
q &=& ac-4d \label{eq_q}\\
r &=& -a^2d+4bd-c^2 \label{eq_r}\;.
\eea
\end{tcolorbox}

På denne måde kan man, ved at løse en tredjegradsligningen~(\ref{tredje_2}),
bestemme værdierne for $u_1$, $u_2$ og $u_3$. Ved at tage kvadratroden af disse
tal bliver der i alt otte kombinationer af fortegn for $T_1$, $T_2$ og $T_3$,
som dog også skal opfylde~(\ref{t123}).

Til senere brug udregner vi nu også:
\bas
p^2 &=& (u_1 + u_2 + u_3)^2 \\
&=& u_1^2 + u_2^2 + u_3^2 + 2(u_1u_2 + u_1u_3 + u_2u_3) \\
&=& u_1^2 + u_2^2 + u_3^2 + 2q \;.
\eas
Heraf fåes:
\begin{equation}
u_1^2 + u_2^2 + u_3^2 = p^2 - 2q\;.
\end{equation}

\bas
&& u_1u_2^2 + u_2u_3^2 + u_3u_1^2 + u_1^2u_2 + u_2^2u_3 + u_3^2u_1 \\
&=& u_1u_2(u_1+u_2) + u_1u_3(u_1+u_3) + u_2u_3(u_2+u_3) \\
&=& -u_1u_2(p+u_3) - u_1u_3(p+u_2) - u_2u_3(p+u_1) \\
&=& -p(u_1u_2 + u_1u_3 + u_2u_3) - 3u_1u_2u_3 \\
&=& -pq + 3r \;.
\eas
Dvs vi har altså
\begin{equation}
u_1u_2^2 + u_2u_3^2 + u_3u_1^2 + u_1^2u_2 + u_2^2u_3 + u_3^2u_1 
= -pq + 3r \;.
\label{eq_u1_u22}
\end{equation}

\bas
-p^3 &=& (u_1+u_2+u_3)(u_1^2+u_2^2+u_3^2 + 2q) \\
&=& u_1^3 + u_2^3 + u_3^3 + u_1^2u_2 + u_2^2u_3 + u_3^2u_1 \\
&& \quad + u_1u_2^2 + u_2u_3^2 + u_3u_1^2 - 2pq \\
&=& u_1^3 + u_2^3 + u_3^3 - pq+3r - 2pq \;.
\eas
Heraf fåes:
\begin{equation}
u_1^3 + u_2^3 + u_3^3 = -p^3 + 3pq - 3r \;.
\end{equation}

\section{Løsning af tredjegradsligningen}
For at løse ligning~(\ref{tredje_2}) indfører vi endnu en substitution:
\bea
A_1 &=& u_1 + \alpha u_2 + \alpha^2 u_3 \label{eq_a1}\\
A_2 &=& u_1 + \alpha^2 u_2 + \alpha u_3  \label{eq_a2}\;,
\eea
hvor $\alpha$ er et komplekst tal, som opfylder:
\begin{equation}
\alpha^2 = -1 - \alpha \;.
\end{equation}
Sammen med ligning~(\ref{equ_p}) udgør ligning~(\ref{eq_a1}) og
ligning~(\ref{eq_a2}) et system af tre lineære ligninger i tre ubekendte.
De kan derfor løses, og svaret er:
\begin{tcolorbox}
\bea
3u_1 &=& -p + A_1 + A_2 \label{eq_u1}\\
3u_2 &=& -p + \alpha^2 A_1 + \alpha A_2 \label{eq_u2}\\
3u_3 &=& -p + \alpha A_1 + \alpha^2 A_2 \label{eq_u3}\;.
\eea
\end{tcolorbox}
Når vi kender $A_1$ og $A_2$ kan ovenstående ligninger altså give rødderne
$u_1$, $u_2$ og $u_3$.

\section{Permutationer af $u_1$, $u_2$ og $u_3$}
Vi undersøger nu, hvorledes variablene $A_1$ og $A_2$ påvirkes af de seks
mulige permutationer af rødderne $u_1$, $u_2$ og $u_3$. Resultatet er

\begin{tabular}{|c|r|r|}
    \hline 
    -    & $A_1$ & $ A_2$ \\
    \hline 
    (12) & $\alpha A_2$ & $\alpha^2 A_1$ \\
    (13) & $\alpha^2 A_2$ & $\alpha A_1$ \\
    (23) & $A_2$ & $A_1$ \\
    \hline 
    (123) & $\alpha^2 A_1$ & $\alpha A_2$ \\
    (132) & $\alpha A_1$ & $\alpha^2 A_2$ \\
    \hline
\end{tabular}

Idet $\alpha^3 = 1$ så ser vi, at såvel $A_1A_2$ som $A_1^3 + A_2^3$ er
uændret efter enhver permutation. Derfor må disse kunne udtrykkes 
ved koefficienterne $p$, $q$ og $r$.

Ved brug af CAS udregner vi følgende:
\bas
A_1 A_2 &=& (u_1+\alpha u_2 +\alpha^2 u_3)(u_1 +\alpha^2 u_2 + \alpha u_3) \\
&=& u_1^2 + u_2^2 + u_3^2 - (u_1u_2 + u_1u_3 + u_2u_3) \\
&=& p^2 - 2q - q \\
&=& p^2 - 3q \;.
\eas

Og igen ved CAS finder vi
\bas
A_1^3 + A_2^3 &=& 2(u_1^3 + u_2^3 + u_3^3) + 12 u_1u_2u_3 \\
&& \quad -3(u_1^2u_2 + u_2^2u_3 + u_3^2u_1 + u_1u_2^2 +u_2u_3^2 + u_3u_1^2) \\
&=& 2(-p^3+3pq-3r) - 12r - 3(-pq+3r) \\
&=& -2p^3 + 9pq - 27r \;.
\eas

Vi har altså ved direkte udregning vist følgende:
\bea
A_1A_2 &=& p^2 - 3q \label{eq_a1a2}\\
A_1^3 + A_2^3 &=& -2p^3 + 9pq - 27r \;.
\eea

\section{Endnu en ny substitution}
For at bestemme $A_1$ og $A_2$ indfører vi endnu en substitution
\bea
y_1 &=& A_1^3 \label{eq_y1}\\
y_2 &=& A_2^3 \label{eq_y2}
\eea
Ovenstående overvejelser viser, at permutationer af $u_1$, $u_2$ og $u_3$
fører til permutationer af $y_1$ og $y_2$. Derfor kan vi nu 
opskrive en andengradsligning
\begin{equation}
y^2 + sy + t = 0,
\label{eq_y}
\end{equation}
hvor $s$ og $t$ er endnu ukendte koefficienter.
Idet vi faktoriserer ligningen som $(y-y_1)(y-y_2) = 0$ må der gælde:
\bas
-s &=& y_1 + y_2 \\
&=& A_1^3 + A_2^3 \;,
\eas
og
\bas
t &=& y_1 y_2 \\
&=& (A_1 A_2)^3 \;.
\eas

Dermed har vi
\bea
s &=& 2p^3 - 9pq + 27r \label{eq_s}\\
t &=& (p^2-3q)^3 \label{eq_t}  \;.
\eea

Andengradsligningen~(\ref{eq_y}) har diskriminanten
\bas
D &=& (y_1-y_2)^2 \\
&=& (y_1+y_2)^2 - 4y_1y_2 \\
&=& s^2 - 4t \;.
\eas
Vi har altså:
\begin{equation}
    D = s^2 - 4t \;.
\label{eq_dis}
\end{equation}

Nu udregner vi
\bas
y_1-y_2 &=& A_1^3 - A_2^3 \\
&=& 3(\alpha-\alpha^2)(u_1^2u_2 + u_2^2u_3 + u_3^2u_1 - u_1u_2^2 - u_2u_3^2 - u_3u_1^2) \\
&=& \sqrt{-27}(u_1-u_2)(u_1-u_3)(u_2-u_3) \;.
\eas
Vi har således vist, at
\begin{equation}
    D = -27[(u_1-u_2)(u_1-u_3)(u_2-u_3)]^2 \;.
    \label{eq_dis3}
\end{equation}
Heraf ser vi nu, at hvis alle fire rødder $x_1$, $x_2$, $x_3$ og $x_4$ er 
reelle, så vil rødderne $u_1$, $u_2$ og $u_3$ også være reelle, og
dermed vil $D$ være negativ.

Det viser sig derfor nemmest at  indføre komplekse tal. Løsningerne til
andengradsligningen~(\ref{eq_y}) er derfor
\begin{equation}
y = \frac{-s \pm i \sqrt{-D}}{2} \;.
\end{equation}
Af hensyn til senere udregninger, omskriver vi til polær form
\[
y = |y| (\cos v \pm i \sin v) \;,
\]
hvor 
\begin{tcolorbox}
\bea
|y| &=& \sqrt{y_1y_2} = \sqrt{t} \label{eq_y_abs}\\
v &=& \arctan\left(\frac{\sqrt{-D}}{-s}\right)
= \arctan\left(\frac{s}{\sqrt{-D}}\right) + \frac{\pi}{2} \;.
\label{eq_v}
\eea
\end{tcolorbox}

\section{Endelig løsning}
Vi er nu klar til at stykke alle stumperne sammen.
Ud fra koefficienterne i den oprindelige fjerdegradsligning~(\ref{eq_4}) 
\[
    x^4 + ax^3 + bx^2 + cx + d = 0\;,
\]
danner vi nogle nye koefficienter $p$, $q$ og $r$ givet ved
ligningerne~(\ref{eq_p}), (\ref{eq_q}) og~(\ref{eq_r}):
\bas
p &=& -b \\
q &=& ac-4d \\
r &=& -a^2d+4bd-c^2 \;.
\eas
Derefter dannes endnu nogle nye kofficienter $s$ og $t$ givet ved
ligningerne~(\ref{eq_s}) og~(\ref{eq_t}):
\bas
s &=& 2p^3 - 9pq + 27r \\
t &=& (p^2-3q)^3 \;.
\eas
Til sidste udregner vi diskriminanten $D$ ud fra ligning~(\ref{eq_dis}):
\[
D = s^2 - 4t \;.
\]

Nu benyttes ligningerne~(\ref{eq_y_abs}) og~(\ref{eq_v}):
\bas
|y| &=& \sqrt{t} \\
v &=& \arctan\left(\frac{s}{\sqrt{-D}}\right) + \frac{\pi}{2} \;.
\eas
Dernæst finder vi værdierne af $A_1$ og $A_2$ ud fra ligningerne~(\ref{eq_y1})
og~(\ref{eq_y2}):
\bas
A_1 &=& \sqrt{t^{1/3}} \left(\cos\frac{v}{3} + i\sin\frac{v}{3}\right) \\
A_2 &=& \sqrt{t^{1/3}} \left(\cos\frac{v}{3} - i\sin\frac{v}{3}\right)  \;.
\eas
Summen af disse er:
\[
    A_1 + A_2 = 2\sqrt{p^2-3q} \cos\frac{v}{3} \;,
\]
og herefter finder vi $u_1$ ved ligning~(\ref{eq_u1}):
\[
    u_1 = \frac{1}{3} \left(-p + 2\sqrt{p^2-3q} \cos\frac{v}{3} \right) \;.
\]
De øvrige to værdier $u_2$ og $u_3$ findes ved:
\bas
u_2 &=& \frac{1}{3} \left(-p + 2\sqrt{p^2-3q} \cos\frac{v+2\pi}{3} \right) \\
u_3 &=& \frac{1}{3} \left(-p + 2\sqrt{p^2-3q} \cos\frac{v+4\pi}{3} \right) \;.
\eas

Disse indsættes i ligningerne~(\ref{eq_t1}), (\ref{eq_t2}) og~(\ref{eq_t3})
\bas
T_1 &=& \sqrt{a^2-4b + 4u_1} \\
T_2 &=& \sqrt{a^2-4b + 4u_2} \\
T_3 &=& \frac{-a^3 + 4ab - 8c}{T_1T_2} \;,
\eas
hvor vi har benyttet ligning~(\ref{t123}).

Til sidst benyttes ligningerne~(\ref{x1})-~(\ref{x4}), til at give de
endelige rødder $x_1$, $x_2$, $x_3$ og $x_4$.
\bas
x_1 &=& \frac{-a + T_1 + T_2 + T_3}4 \\
x_2 &=& \frac{-a + T_1 - T_2 - T_3}4 \\
x_3 &=& \frac{-a - T_1 + T_2 - T_3}4 \\
x_4 &=& \frac{-a - T_1 - T_2 + T_3}4 \;.
\eas

\section{Diskriminanten}
For at komme videre med diskriminanten udregner vi faktorerne parvis.
Først regner vi:
\bea
&& (x_1-x_2)(x_3-x_4) \\
&=& x_1x_3-x_1x_4-x_2x_3+x_2x_4 \\
&=& u_2 - u_3\;.
\eea
På tilsvarende vis finder vi, at 
\bea
(x_1-x_3)(x_2-x_4) &=& u_1 - u_3 \\
(x_1-x_4)(x_2-x_3) &=& u_1 - u_2 \;.
\eea

Dette giver følgende formel for diskriminanten
\[
D_4 = \left[(u_1-u_2)(u_1-u_3)(u_2-u_3)\right]^2 \;.
\]
Ved at kombinere ligningerne~(\ref{eq_dis3}) med~(\ref{eq_dis}) får vi:
\bas
    -27 D_4 &=& s^2 - 4t \\
    &=& (2p^3-9pq+27r)^2 - 4(p^2-3q)^3 \\
    &=& -27(-4p^3r + p^2q^2 + 18pqr - 4q^3 - 27r^2)\;.
\eas
Heraf fås:
\begin{equation}
D_4 = \frac{s^2-4t}{-27} = -4p^3r + p^2q^2 + 18pqr - 4q^3 - 27r^2\;.
\end{equation}

Vi kan finde et andet udtryk for diskriminanten ved
at indføre følgende to hjælpevariable:
\bas
B_1 &=& u_1^2u_2 + u_2^2u_3 + u_3^2u_1 \\
B_2 &=& u_1u_2^2 + u_2u_3^2 + u_3u_1^2 \;.
\eas
Vi har derfor
\begin{equation}
D_4 = (B_1-B_2)^2 = (B_1+B_2)^2 - 4B_1B_2\;.
\end{equation}

Vi har i ligning~(\ref{eq_u1_u22}) vist, at
\[
B_1 + B_2 = -pq + 3r\;.
\]
Indsættes dette i ovenstående giver det:
\bas
4B_1B_2 &=& (B_1+B_2)^2 - D_4 \\
&=& (-pq+3r)^2 - (-4p^3r + p^2q^2 + 18pqr - 4q^3 - 27r^2) \\
&=& 4p^3r - 24pqr + 4q^3 + 36r^2 \;.
\eas
Dermed er
\begin{equation}
B_1B_2 = p^3r - 6pqr + q^3 + 9r^2\;.
\end{equation}
Så kan diskriminanten også skrives som:
\bas
D_4 &=& (B_1+B_2)^2 - 4B_1B_2 \\
&=& (3r-pq)^2 - 4(q^3 + rp^3 - 6pqr + 9r^2) 
\eas


\section{Eksempel}
Som eksempel på anvendelsen af denne teori, bliver følgende
ligning gennemregnet:
\begin{equation}
    x^4 - 3,3 x^3 - 4,21 x^2 + 12,633 x - 4,914 = 0 \;.
\end{equation}
Vi har altså de fire koefficienter
\bas
a &=& -3,3 \\
b &=& -4,21 \\
c &=& 12,633 \\
d &=& -4,914 \;.
\eas

Først udregner vi
\bas
p &=& 4,21 \\
q &=& -3,3\cdot 12,633 + 4\cdot 4,914 = -22,0329 \\
r &=& 3,3^2\cdot 4,914 + 4\cdot 4,21 \cdot 4,914 - 12,633^2 = -23,327469 \;.
\eas
Dernæst udregner vi
\bas
s &=& 2\cdot 4,21^3 + 9\cdot 4,21 \cdot 22,0329 - 27\cdot 23,327469
    = 354,22184\\
t &=&  (4,21^2 + 3\cdot 22,0329)^3 = 83,8228^3
    = 588960,1937596 \;.
\eas
Vi udregner nu diskriminanten 
\bas
D &=& s^2-4t = 276,603379^2 - 4\cdot 588960,937595628 = -2230370,63845 \;.
\eas

Nu udregner vi
\bas
v &=& \arctan\left(\frac{354,22184}{\sqrt{223370,63845}}\right) + \frac{\pi}{2} \\
&=& 1,80367771621 \;.
\eas

Det giver:
\bas
u_1 &=& \frac13 \left(-4,21 + 2\sqrt{83,8228} \cos\left(\frac{1,80367771621}{3}\right)\right) \\
 &=& 3,63 \\
u_2 &=& \frac13 \left(-4,21 + 2\sqrt{83,8228} \cos\left(\frac{1,80367771621+2\pi}{3}\right)\right) \\
 &=& -6,91 \\
u_3 &=& \frac13 \left(-4,21 + 2\sqrt{83,8228} \cos\left(\frac{1,80367771621+4\pi}{3}\right)\right) \\
 &=& -0,93 \;.
\eas

Dernæst udregner vi
\bas
    a^2-4b &=& 27,73 \\
    T_1T_2T_3 &=& -a^3 + 4ab - 8c = -9,555 \;,
\eas
og derefter
\bas
T_1 &=& \sqrt{27,73 + 4\cdot 3,63} = 6,5 \\
T_2 &=& \sqrt{27,73 - 4\cdot 6,91} = 0,3 \\
T_3 &=& \frac{-9,555}{6,5\cdot 0,3} = -4,9 \;.
\eas
Dermed bliver rødderne ifølge~(\ref{x1})-(\ref{x4}):
\bea
x_1 &=& \frac14(3,3+6,5+0,3-4,9) = 1,3 \\
x_2 &=& \frac14(3,3+6,5-0,3+4,9) = 3,6 \\
x_3 &=& \frac14(3,3-6,5+0,3+4,9) = 0,5 \\
x_4 &=& \frac14(3,3-6,5-0,3-4,9) = -2,1 \;.
\eea
Ved indsættelse ses disse tal at passe.

\end{document}

