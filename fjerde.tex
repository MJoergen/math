\documentclass[12pt,oneside,a4paper]{article}

\usepackage[utf8]{inputenc} % Lærer LaTeX at forstå unicode - HUSK at filen skal
% være unicode (UTF-8), standard i Linux, ikke i
% Win.

\usepackage[danish]{babel} % Så der fx står Figur og ikke Figure, Resumé og ikke
% Abstract etc. (god at have).

%\usepackage{graphicx}
\usepackage{amsfonts}
\usepackage{amsthm}        % Theorems
\usepackage{amsmath}
%\usepackage{hyperref}

%\renewcommand{\mid}[1]{{\rm E}\!\left[#1\right]}
\newcommand{\bas}{\begin{eqnarray*}}
\newcommand{\eas}{\end{eqnarray*}}
\newcommand{\be}{\begin{equation}}
\newcommand{\ee}{\end{equation}}
\newcommand{\bea}{\begin{eqnarray}}
\newcommand{\eea}{\end{eqnarray}}

\newtheorem{thm}{Sætning}[section]
\newtheorem{mydef}[thm]{Definition}
\newtheorem{eks}[thm]{Eksempel}

\DeclareMathSymbol{,}{\mathord}{letters}{"3B}

\title{Løsning af fjerdegradsligninger}
\date{Marts 2017}
\author{Michael Jørgensen}

\begin{document}

\maketitle
Vi betragter det generelle fjerdegradspolynonium
\begin{equation}
    x^4 + ax^3 + bx^2 + cx + d = 0\;,
\end{equation}
og kalder rødderne for $x_1$, $x_2$, $x_3$ og $x_4$.
Opgaven i det følgende er at bestemme rødderne ud fra kendskabet til
koefficienterne $a$, $b$, $c$ og $d$.
For at gøre dette, gør vi brug af gruppe-teori, teori om
symmetriske polynomier, samt flittig brug af CAS.

\section{Sammenhæng mellem rødder og koefficienter}
Til at begynde med løser vi den modsatte opgave, nemlig at bestemme
koefficienterne ud fra rødderne. Dette gøres ved at faktorisere polynomiet:
\begin{equation}
    x^4 + ax^3 + bx^2 + cx + d = (x-x_1)(x-x_2)(x-x_3)(x-x_4)\;,
    \label{fakt}
\end{equation}
og så samle led med potenser af $x$.
Dette giver (vha CAS) følgende resultat:
\bea
  -a &=& x_1 + x_2 + x_3 + x_4 \label{koef_a} \\
   b &=& x_1x_2 + x_1x_3 + x_1x_4 + x_2x_3 + x_2x_4 + x_3x_4  \label{koef_b}\\
  -c &=& x_1x_2x_3 + x_1x_2x_4 + x_1x_3x_4 + x_2x_3x_4  \label{koef_c}\\
   d &=& x_1x_2x_3x_4 \label{koef_d}
\eea
I stedet for at løse det oprindelige fjerdegradspolynomium, så kan vi i stedet løse ovenstående fire ligninger med fire ubekendte. Desværre er ligningerne ikke lineære, så umiddelbart hjælper det os ikke.

\section{Permutationer af rødder}
Faktoriseringen~(\ref{fakt}) er tydeligvis invariant over for permutationer af
rødderne. Dvs hvis f.eks. $x_1$ og $x_2$ bytter plads, så vil faktorisering være
uændret, fordi faktorernes orden ikke betyder noget.
Heraf følger, at de fire koefficienter $a$, $b$, $c$ og $d$ således også er
invariante over for alle permutationer af rødderne. Ved inspektion af
ligningerne~(\ref{koef_a})-(\ref{koef_d}) ser vi således, at vilkårlige
ombytninger af rødderne ikke ændrer værdierne af $a$, $b$, $c$ eller $d$.

Omvendt gælder der, at ethvert udtryk, som er invariant over for vilkårlige
permutationer af rødderne, vil kunne omskrives til et algebraisk udtryk i 
koefficienterne $a$, $b$, $c$ og $d$, og dette endda på en éntydig måde. Vi ser
på et eksempel:
Udtrykket
\begin{equation}
    x_1^2 + x_2^2 + x_3^2 + x_4^2\;,
\end{equation}
er tydeligvis invariant over for alle permutationer af rødderne. Det kan derfor
omskrives til et algebraisk udtryk i $a$, $b$, $c$ og $d$.
Ved hjælp af CAS kan vi verificere følgende identiteter:
\bea
    x_1^2 + x_2^2 + x_3^2 + x_4^2 &=& a^2 - 2b \\
    x_1^3 + x_2^3 + x_3^3 + x_4^3 &=& -a^3 + 3ab - 3c \\
    x_1^4 + x_2^4 + x_3^4 + x_4^4 &=& a^4 - 4a^2b + 4ac + 2b^2 - 4d \;.
\eea
Ovenstående formler skal ikke bruges til noget i det følgende, men er 
udelukkende ment som eksempler.

\section{Substitution}
For at komme videre, indfører en smart substitution:
\bea
    T_1 &=& x_1 + x_2 - x_3 - x_4 \label{subst_t1}\\
    T_2 &=& x_1 - x_2 + x_3 - x_4 \label{subst_t2}\\
    T_3 &=& x_1 - x_2 - x_3 + x_4 \label{subst_t3}.
\eea
Hver af de tre nye variabler udgør, sammen med~(\ref{koef_a}), et system af
fire lineære ligninger med fire ubekendte.  Med andre ord, det er muligt at
udtrykke rødderne ud fra kendskab til $T_1$, $T_2$ og $T_3$.  Udregning på CAS
giver følgende svar:
\bea
    4x_1 &=& -a + T_1 + T_2 + T_3 \label{x1} \\
    4x_2 &=& -a + T_1 - T_2 - T_3 \label{x2} \\
    4x_3 &=& -a - T_1 + T_2 - T_3 \label{x3} \\
    4x_4 &=& -a - T_1 - T_2 + T_3 \label{x4} \;.
\eea
I det følgende vil vi beskrive, hvorledes vi kan bestemme værdierne af de nye
variabler $T_1$, $T_2$ og $T_3$, og således med ovenstående formler bestemme
    rødderne.

\section{Egenskaber ved substitutionen}
Vi undersøger nu, hvorledes permutationer af rødderne påvirker de nye variabler
$T_1$, $T_2$ og $T_3$.

Vi ser først på den simple ombytning $x_1 \leftrightarrow x_2$. Det ses
umiddelbart ud fra ligningerne~(\ref{subst_t1})-(\ref{subst_t3}), at $T_1$ er
uændret, mens $T_2$ og $T_3$ skifter fortegn.

Dernæst undersøger vi den cykliske ombytning af de første tre rødder $x_1$,
$x_2$ og $x_3$, givet ved: $x_1 \rightarrow x_2 \rightarrow x_3 \rightarrow
x_1$.  Ved inspektion ser vi, at $T_1 \rightarrow -T_3$, $T_2 \rightarrow T_1$
og $T_3 \rightarrow -T_2$.

Ved at fortsætte på samme måde og undersøge samtlige 24 permutationer af
rødderne, så ser vi, at det hver gang fører til permutationer af de nye
variabler, samt eventuelle fortegnsskift.

Vi kan eliminere fortegnsskiftene ved at kvadrere de nye variable. Dermed
kommer vi frem til resultatet, at permutationer af rødderne fører til
permutationer er tallene $T_1^2$, $T_2^2$ og $T_3^2$. Det kan udtrykkes på den
måde, at følgende mængde af tal er invariant over for vilkårlige permutationer
af rødderne:
\begin{equation}
    \{T_1^2,\, T_2^2,\, T_3^2\}\;.
    \label{n_set}
\end{equation}

\section{Omskrivning til tredjegradsligning}
Vi opskriver nu en ny tredjegradsligning i $u$, hvor elementerne
i~(\ref{n_set}) er rødder. Dette gøres ved simpel faktorisering:
\begin{equation}
    (u-T_1^2)(u-T_2^2)(u-T_3^2)=0 \;.
    \label{tredje_1}
\end{equation}
Ved at gange parenteserne ud kan vi også skrive det på følgende måde:
\begin{equation}
    u^3 + pu^2+ qu + r = 0 \;,
    \label{tredje_2}
\end{equation}
hvor koefficienterne $p$, $q$ og $r$ kan udtrykkes med de nye rødder $T_1^2$, $T_2^2$ og $T_3^2$.

Vi vil i det følgende vise, at de nye koefficienter $p$, $q$ og $r$ også kan
udtrykkes ved de oprindelige koefficienter $a$, $b$, $c$ og $d$ på følgende
måde:
\bea
-p &=& 3a^2 - 8b \label{eq_p}\\
 q &=& 3a^4 - 16a^2b^2 + 16b^2 + 16ac - 64d \label{eq_q}\\
-r &=& a^6 - 8a^4b + 16a^2b^2 + 16a^3c - 64abc + 64c^2 \label{eq_r}
\eea

For at vise det, skal vi først gange ud i ligning~(\ref{tredje_1}) og
sammenligne led for led med ligning~(\ref{tredje_2}). Det giver
\bea
-p &=& T_1^2 + T_2^2 + T_3^2 \\
 q &=& T_1^2T_2^2 + T_1^2T_3^2 + T_2^2T_3^2 \\
-r &=& T_1^2T_2^2T_3^2 \;.
\eea
Derefter er det en simpel opgave på CAS at indsætte
udtrykkene~(\ref{subst_t1})-(\ref{subst_t3}) i ovenstående og bede CAS om at
reducere.

På denne måde kan man - ved at løse en tredjegradsligning - bestemme værdierne
for $T_1^2$, $T_2^2$ og $T_3^2$. Ved at tage kvadratroden af disse tal bliver
der i alt otte kombinationer af fortegn for $T_1$, $T_2$ og $T_3$. Disse kan dog ikke variere vilkårligt, idet der gælder følgende:
\begin{equation}
    T_1 T_2 T_3 = -a^3 + 4ab - 8c \;,
    \label{t123}
\end{equation}
som igen vises på CAS.


\section{Eksempel}
Som eksempel på anvendelsen af denne teori, bliver følgende
ligning gennemregnet:
\begin{equation}
    x^4 - 3,3 x^3 - 4,21 x^2 + 12,633 x - 4,914 = 0 \;.
\end{equation}

I vort eksempel giver ligningerne~(\ref{eq_p})-(\ref{eq_r}) følgende:
\bea
-p &=& 66,35 \\
 q &=& 1020,3859 \\
-r &=& 91,298025\;,
\eea
og vi skal derfor løse følgende tredjegradsligning:
\[
    u^3 - 66,35 u^2 + 1020,3859 u - 91,298025 = 0\;.
\]
Her finder vi rødderne
\bea
u_1 &=& 0,09 \\
u_2 &=& 24,01 \\
u_3 &=& 42,25
\eea
Endvidere har vi fra ligning~(\ref{t123}), at
\begin{equation}
T_1 T_2 T_3 = -9,555\;.
\label {t123_e}
\end{equation}
Som kontrol har vi, at 
\begin{equation}
    u_1u_2u_3 = 91,298025 = (-9,555)^2
\end{equation}
Vi vælger den positive kvadratrod for $T_1$ og $T_2$ og må derfor, på grund af
ligning~(\ref{t123_e}) vælge den negative rod for $T_3$.
Det giver:
\bea
T_1 &=& 0,3 \\
T_2 &=& 4,9 \\
T_3 &=& -6,5
\eea
Dermed bliver rødderne ifølge~(\ref{x1})-(\ref{x4}):
\bea
x_1 &=& \frac14(3,3+0,3+4,9-6,5) = 0,5 \\
x_2 &=& \frac14(3,3+0,3-4,9+6,5) = 1,3 \\
x_3 &=& \frac14(3,3-0,3+4,9+6,5) = 3,6 \\
x_4 &=& \frac14(3,3-0,3-4,9-6,5) = -2,1
\eea
Ved indsættelse ses disse tal at passe.


\end{document}

