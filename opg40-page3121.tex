\documentclass[12pt,oneside,a4paper]{article}

\usepackage[utf8]{inputenc} % Lærer LaTeX at forstå unicode - HUSK at filen skal
% være unicode (UTF-8), standard i Linux, ikke i
% Win.

\usepackage[danish]{babel} % Så der fx står Figur og ikke Figure, Resumé og ikke
% Abstract etc. (god at have).

%\renewcommand{\mid}[1]{{\rm E}\!\left[#1\right]}
\newcommand{\bas}{\begin{eqnarray*}}
\newcommand{\eas}{\end{eqnarray*}}

\begin{document}

\section{Opgave 40}
Løs inden for de reelle tal ligningssystemet
\bas
x+y &=& z \\
x^2+y^2 &=& z \\
x^3+y^3 &=& z
\eas

\section{Løsning}
Vi ser umiddelbart, at $z \ge 0$. Endvidere ser vi, at hvis $z=0$, så er der netop én løsning $(0, 0, 0)$.

Antag derfor nu, at $z \ne 0$. Så udregner vi
$$
z^2 = (x+y)^2 = x^2+y^2+2xy = z+2xy
$$
og 
$$
z^3 = (x+y)^3 = x^3+y^3+3xy(x+y) = z(1+3xy)
$$
Da $z\ne 0$ kan den sidste ligning forkortes med $z$. Det giver 
$$
z^2=1+3xy
$$
Isoleres nu $xy$ i de to ovenstående ligninger for $z^2$ får vi følgende ligning:
$$
xy = \frac{z^2-1}{3} = \frac{z^2-z}{2}
$$
Det sidste lighedstegn reducerer til følgende andengradsligning
$$
z^2-3z+2=0
$$
som har løsningerne $z=1$ og $z=2$.

Nu undersøger vi først $z=1$: Det giver $xy=0$, som sammen med $x+y=1$ giver to løsninger $(0, 1, 1)$ og $(1, 0, 1)$.

Nu undersøger vi dernæst $z=2$: Det giver $xy=1$, som sammen med $x+y=2$ giver én løsning $(1, 1, 2)$.

I alt er der fundet fire forskellige løsninger.

\end{document}

