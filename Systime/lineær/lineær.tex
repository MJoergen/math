\documentclass[12pt,oneside,a4paper]{article}

\usepackage[utf8]{inputenc} % Lærer LaTeX at forstå unicode - HUSK at filen skal
% være unicode (UTF-8), standard i Linux, ikke i
% Win.

\usepackage[danish]{babel} % Så der fx står Figur og ikke Figure, Resumé og ikke
% Abstract etc. (god at have).

\usepackage{graphicx}
\usepackage{amsfonts}
\usepackage{amsthm}        % Theorems
\usepackage{amsmath}
\usepackage{float}         % Så kan man bedre styre, hvor figurerne havner henne
                           % vha [H].
%\usepackage{hyperref}

%\renewcommand{\mid}[1]{{\rm E}\!\left[#1\right]}
\newcommand{\bas}{\begin{eqnarray*}}
\newcommand{\eas}{\end{eqnarray*}}
\newcommand{\be}{\begin{equation}}
\newcommand{\ee}{\end{equation}}
\newcommand{\bea}{\begin{eqnarray}}
\newcommand{\eea}{\end{eqnarray}}

\newtheorem{thm}{Sætning}[section]
\newtheorem{mydef}[thm]{Definition}
\newtheorem{eks}[thm]{Eksempel}

\DeclareMathSymbol{,}{\mathord}{letters}{"3B}

\title{Lineære modeller}
\date{\vspace{-5ex}}

\begin{document}

\maketitle

%%%%%%%%%%%%%%%%%%%%%%%%%%%%%%%%%%%%%%%%%%%%%%%%%%%%%%%%%%%%%%%%%

\section{Indledning}
Lineære modeller bruges dels til at foretage eksakte omregninger, f.eks. mellem Fahrenheit
Celsius, og dels til at lave prognoser ud fa kendte data.

Vi begynder med at se på nogle eksempler.

\subsection{Eksempel}
Et teleselskab har et simpelt takstsystem:
\begin{itemize}
    \item Abonnement 20 kr. pr. måned, samt 0,50 kr. pr. talt minut.
\end{itemize}
Hvis en abonnent på en måned taler 25 minutter, skal han betale:
$$
20 + 0,50\cdot25 = 32,50 \, {\rm kr}.
$$
Hvis der tales $x$ minutter en bestemt måned, er prisen $y$ kroner bestemt ved:
$$
y = 20 + 0,50\cdot x
$$
Man siger, at der er en {\em lineær sammenhæng} mellem samtaletiden $x$ og prisen $y$.

Hvis vi afbilder sammenhængen i et koordinatsystem med antallet af minutter på
$x$-aksen og prisen på $y$-aksen, vil sammenhørende værdier af $x$ og $y$ (dvs.
af taletid og pris) udgøre en ret linje med hældning $0,50$, som skærer
$y$-aksen i tallet $20$ (det koster 20 kroner at tale i 0 minutter), se figur
1.1.

\begin{figure}[H]
    \centering
    \includegraphics[width=0.5\textwidth]{lin-1}
    \caption{Figur 1.1}
\end{figure}

Hældningen $0,50$ angiver prisstigningen for hvert ekstra minut, der tales --
den er 0,50 kroner.  Prisen for 20 minutters samtale er 30 kr., og for 21
minutters samtale er den 30,50 kr.

\subsection{Eksempel}
I Europa bruger vi temperaturskalaen Celsius, hvor vand fryser ved 0 grader og
koger ved 100 grader.  I USA bruger de derimod temperaturskalaen Fahrenheit,
hvor vand fryser ved 32 F og koger ved 212 F.
Man kan omregne mellem de to temperaturskalaer ved følgende formel:
\[
F = 1,8 C + 32 \,.
\]
Denne sammenhæng er vist grafisk i figuren:

\begin{figure}[H]
    \centering
    \includegraphics[width=0.5\textwidth]{lin-1a}
    \caption{Figur 1.1a}
\end{figure}

Ud fra ligningen er det muligt at omregne mellem temperaturer i Celsius og
temperaturer i Fahrenheit. F.eks.  er kroppens normaltemperatur 37 grader C. I
Fahrenheit vil den samme temperatur være
\[
    32 + 1,8\cdot 37 = 98,6 F \,.
\]


%%%%%%%%%%%%%%%%%%%%%%%%%%%%%%%%%%%%%%%%%%%%%%%%%%%%%%%%%%%%%%%%%

\section{Lineære sammenhænge}
\begin{mydef}
    En variabelsammenhæng er en sammenhæng mellem en uafhængig variabel, ofte
    kaldet $x$, og en afhængig variabel, ofte kaldet $y$.
\end{mydef}

En variabelsammenhæng er ofte beskrevet ved en ligning, men kan også tegnes grafisk i et
koordinatsystem.

\begin{mydef}
    En lineær sammenhæng er en variabelsammenhæng givet ved følgende ligning:
    $$
    y = a\cdot x + b \,,
    $$
    hvor $a$ og $b$ er tal.
\end{mydef}

Vi interesserer os for de koordinatsæt $(x,\,y)$, der passer i en sådan ligning.
Fra folkeskolen ved vi, at sådanne koordinatsæt udgør en ret linje.

Eksempler på sådanne lineære sammenhænge er (se figur 1.2):

\begin{tabular}{ll}
    $\bullet\quad y=3x-5$  & Her er $a=3$ og $b=-5$. \\
    $\bullet\quad y=-2x+1$ & Her er $a=-2$ og $b=1$. \\
    $\bullet\quad y=4-x$   & Her er $a=-1$ og $b=4$. \\
    $\bullet\quad y=3x$    & Her er $a=3$ og $b=0$.
\end{tabular}

\begin{figure}[H]
    \centering
    \includegraphics[width=0.5\textwidth]{lin-2}
    \caption{Figur 1.2}
\end{figure}

\begin{thm}
    For den lineære sammenhæng
    $$
    y = a\cdot x + b
    $$
    gælder, at grafen er en ret linje, hvor $a$ er linjens {\em hældning}.  Det
    betyder, at linjen vokser med $a$ enheder, når $x$ vokser med 1 enhed.
    Punktet $(0,\,b)$ er linjens {\em skæringspunkt med $y$-aksen}.
\end{thm}

Vi kan opstille en tabel ("sildeben") over koordinater $(x,\,y)$, der passer i
ligningen $y=3x-5$:
$$
\begin{tabular}{c|c|c|c|c|c|c}
    x &  -2 & -1 &  0 &  1 & 2 & 3 \\
    \hline
    y & -11 & -8 & -5 & -2 & 1 & 4
\end{tabular}
$$
Linjen går altså gennem punkterne $(0,\-5)$, $(1,\,-2)$ osv., se figur 1.3.

\begin{figure}[H]
    \centering
    \includegraphics[width=0.5\textwidth]{lin-3}
    \caption{Figur 1.3}
\end{figure}

Linjer med samme hældning er parallelle. Linjer med positiv hældning forløber
opad mod højre, linjer med negativ hældning nedad mod højre. Linjer med
hældning 0 er vandrette.  Sådanne linjer har en ligning af formen $y=0x+b$
eller blot $y=b$.

At tallet $b$ er linjens skæringspunkt med $y$-aksen følger af, at
koordinatsættet $(0,\,b)$ passer i ligningen:
$$
b = 0\cdot x+b \,.
$$

Hældningen bestemmes altså ud fra, hvor meget $y$ vokser for hver enhed af $x$. Det
er derfor muligt at aflæse hældningen på grafen ved (fra et valgfrit startpunkt)
at gå 1 til højre og se, hvor meget $y$ vokser.

Hældningen er illustreret på figur 1.4. Hvis $x$ vokser til $x+1$, så vokser
$y$ fra $y_1$ til $y_2$ og forskellen $y_2-y_1$ er netop $a$. Bemærk, at $a$
også kan være negativ, nemlig når $y$ falder i værdi, når $x$ vokser.

\begin{figure}[H]
    \centering
    \includegraphics[width=0.5\textwidth]{lin-4}
    \caption{Figur 1.4}
    \label{fig33}
\end{figure}


\subsection{Eksempel}
\begin{figure}[H]
    \centering
    \includegraphics[width=0.5\textwidth]{lin-3a}
    \caption{Figur 1.3a}
\end{figure}

På grafen ovenfor har vi valgt startpunktet $(1,\,1)$. Når vi bevæger
1 til højre, så skal vi gå 2 op for at ramme grafen. Derfor er hældningen $a=2$.

$b$ er en konstant, der afgør, hvor grafen skærer $y$-aksen. Hvis $b$ er positiv finder
skæringen sted over $x$-aksen, og hvis $b$ er negativ er skæringen placeret under $x$-aksen.

På figuren ovenfor kan vi se, at grafen skærer $y$-aksen i punktet $(0,\,-1)$. Derfor
er $b=-1$.

\subsection{Quiz}
\begin{itemize}
    \item Aflæs koordinater af punkter på grafen.
    \item Afgør om et givet punkt ligger på grafen.
    \item Udregn $y$ ud fra $x$ vha ligningen.
    \item Afgør om et punkt opfylder ligningen.
    \item Aflæs hældningen af en ret linje.
    \item Aflæs skæringen med $y$-aksen.
    \item Aflæs forskriften for en ret linje.
    \item Opskriv ligningen for en linje ud fra en sproglig beskrivelse.
\end{itemize}


%%%%%%%%%%%%%%%%%%%%%%%%%%%%%%%%%%%%%%%%%%%%%%%%%%%%%%%%%%%%%%%%%

\section{Lineær regression}
Ofte, så har man givet en række datapunkter, som ikke ligger på en ret linje. Alligevel
vil man ofte benytte en lineær model som tilnærmelse. Vi ser på et eksempel:

Man har undersøgt højden af et stort antal piger og beregnet middel\-høj\-den for
hver årgang. Pigernes højde ved forskellige aldre fremgår af skemaet.

\[
\begin{array}{|c|c|c|c|c|c|c|c|}
    \hline
    \mbox{Alder (år)},\, x &  2 &  3 &  4 &  5 &  6 &  7 &  8 \\
    \hline
    \mbox{Højde (cm)},\, y &  89,2 &   98,3 &   104,9 &  112,0 &  118,1 &  123,4 &  131,3 \\
    \hline
\end{array}
\]

\[
\begin{array}{|c|c|c|c|c|c|c|}
    \hline
    \mbox{Alder (år)},\, x  & 9  & 10 \\
    \hline
    \mbox{Højde (cm)},\, y  & 136,4 &  142,5 \\
    \hline
\end{array}
\]

\begin{figure}[H]
    \centering
    \includegraphics[width=0.5\textwidth]{lin-12}
    \caption{Figur 1.12}
    \label{fig56}
\end{figure}

På figuren ovenfor er punkterne indtegnet, og endvidere er tegnet en {\em tendenslinje} med ligningen:
\[
    y = 6,53 x + 78,2 \,,
\]
hvor $x$ er alderen i år, og $y$ er pigernes gennemsnitlige højde i cm.

Det ses, at linjen følger punkterne med rimelighed. Det er derfor muligt at bruge ligningen
for den rette linje til at foretage beregninger. F.eks. kan man beregne en piges gennemsnitlige højde ved alderen 12 år ved at indsætte $x=12$ i ligningen. Det giver
\[
    y=6,53\cdot 12 + 78,2 = 156,6 \mbox{ cm. }
\]
Altså forudsiger denne model, at piger i alderen 12 år har en gennemsnitlig
højde på 156,6 cm.  Det skal understreges, at der en tale om en prognose som
bygger på en række forudsætninger.  I dette tilfælde forudsætter modellen,
fordi det er en lineær model, at piger vokser med det samme antal cm hvert år,
i dette tilfælde 6,53 cm. Dette er selvsagt ikke længere gyldigt, når pigernes
alder har passeret puberteten. Endvidre er der tale om et gennemsnit af alle
danske piger.
Alligevel bliver sådanne modeller brugt i stor udstrækning, og det er derfor vigtigt
at kende modellenes forudsætningner. 
\begin{thm}
    I en lineær model forudsættes det, at væksthastigheden er konstant.
\end{thm}

\subsection{Mindste kvadraters metode}

For at finde frem til tendenslinjen, som er den linje, der bedst bestemmer
punkterne, benytter man sædvanligvis en metode, der kaldes {\em mindste
kvadraters metode}.

\begin{figure}[H]
    \centering
    \includegraphics[width=0.5\textwidth]{lin-11}
    \caption{Figur 1.11}
    \label{fig55}
\end{figure}

På fig. 1.11 er en række punkter afsat, og en vilkårlig linje $m$ er tegnet. Desuden er de
lodrette afstande $d_1$, $d_2$, $d_3$, ..., $d_n$ fra målepunkterne til linjen afsat. Man
kan så udregne summen $D$ af kvadraterne (kvadratsummen) på afstandene:

\[
D = {d_1}^2 + {d_2}^2 + {d_3}^2 + \ldots + {d_n}^2 \; . 
\]

Hvis man vælger en anden linje, får man selvfølgelig i reglen en anden kvadratsum.

Man har vedtaget at den linje, der gør kvadratsummen $D$ mindst mulig, er den
'bedste' linje. Metoden kaldes {\em de mindste kvadraters metode}, og den linje, der
fremkommer, kaldes {\em regressionslinjen} svarende til punkterne. Vi går ikke ind
på, hvordan linjen matematisk bestemmes. 

\subsection{Korrelation}
CAS giver foruden regressionslinjens ligning den såkaldte {\em
korrelations\-koef\-fi\-cient} $r$. Vi kommer ikke her ind på teorien bag
beregningen af dette tal, men nøjes med at bemærke, at
korrelationskoefficienten er et mål for den lineære sammenhæng mellem
punkterne.

Vi bemærker følgende om korrelationskoefficienten $r$:

\begin{itemize}
    \item Hvis $r = 1$ ligger punkterne præcis på ret linje med positiv hældning.
    \item Hvis $0 < r < 1$ har regressionslinjen positiv hældning. Jo tættere $r$
        er på 1, desto tættere ligger punkterne på linjen.
    \item Hvis $r = 0$ eller tæt ved 0, er der ingen eller kun svag lineær sammenhæng.
    \item Hvis $-1 < r < 0$ har regressionslinjen negativ hældning. Jo tættere $r$
        er på -1, desto tættere ligger punkterne på linjen.
    \item Hvis $r = -1$ ligger punkterne præcis på ret linje med negativ hældning. 
\end{itemize}
På fig. 5.7 er disse forhold illustreret.

\begin{figure}[H]
    \centering
    \includegraphics[width=0.5\textwidth]{fig57}
    \caption{Figur 5.7}
    \label{fig57}
\end{figure}


%%%%%%%%%%%%%%%%%%%%%%%%%%%%%%%%%%%%%%%%%%%%%%%%%%%%%%%%%%%%%%%%%

\section{Løsning af førstegradsligninger}
En ligning er af første grad, hvis den er af typen
$$
ax + b = 0\; ,
$$
eller umiddelbart kan omskrives til en sådan ligning. Her er et par eksempler:
$$
3x - 4 = 8 + 2x \quad , \quad 5(x - 4) + 2x = 6x - 2(3 - 4x)\; .
$$

\subsection{Ensbetydende ligninger}
Hvis man foretager en række omformninger af en ligning for at finde en løsning,
bruger man symbolet $\Leftrightarrow$ , en dobbeltpil (eller en {\em biimplikation}), mellem
ligninger, der har de samme løsninger. Fx har vi
\bas
&& 2(3x - 1) = x - 4(2 - x)\\
&\Leftrightarrow& 6x - 2 = x - 8 + 4x\\
&\Leftrightarrow& 6x - 2 = 5x - 8 \\
&\Leftrightarrow& x = -6 
\eas
Hver af de fire ligninger har den samme løsning. Man siger, at sådanne
ligninger er {\em ensbetydende}. Man kan også sige, at ensbetydende ligninger fremgår
af hinanden ved ’lovlige’ omformninger.

Vi anfører de vigtigste regler for løsning af ligninger.

[ANIMATION OM REGNEREGLER 1--4]


%%%%%%%%%%%%%%%%%%%%%%%%%%%%%%%%%%%%%%%%%%%%%%%%%%%%%%%%%%%%%%%%%

\section{Skæring mellem linjer -- omskrevet fra tilsvarende afsnit i Mat-C bogen}
Hvis to linjers ligninger er kendt, vil vi finde koordinaterne til linjernes
skæringspunkt (hvis de ikke er parallelle).

Linjerne $l$ og $m$ har f.eks. ligningerne (se figur 2.9)
\[
    l: y=0,5x-2\quad{\mbox{og}}\quad m: y=-3x+5
\]

\begin{figure}[H]
    \centering
    \includegraphics[width=0.5\textwidth]{lin-9}
    \caption{Figur 2.9}
    \label{fig6}
\end{figure}

I skæringspunktet mellem linjerne skal $y$-værdien for de to linjer være den
samme, så der må gælde, at
\[
0,5x-2 = -3x+5 \,,
\]
og denne ligning løses på følgende måde:
\bas
&& 0,5x+3x=5+2\\
&\iff& 3,5x=7\\
&\iff& x=2
\eas
Denne værdi af $x$ indsættes i en af ligningerne, ligegyldig hvilken:
\[
y=0,5\cdot2-2=-1 \,.
\]
Skæringspunktet har altså koordinaterne $(2,\,-1)$, hvilket ser ud til at
stemme med figuren.


%%%%%%%%%%%%%%%%%%%%%%%%%%%%%%%%%%%%%%%%%%%%%%%%%%%%%%%%%%%%%%%%%

\section{Ligefrem proportionalitet (kopi af afsnit i Mat-C bogen)}
Man bruger den talemåde, at to variabler $x$ og $y$ er {\em ligefrem
proportionale} (eller blot: {\em proportionale}), hvis de "stiger og falder i
samme takt". Vi skal se på, hvad denne lidt løse udtalelse dækker over.

Hvis en bil kører med konstant hastighed, f.eks. 90 km/t, kan vi opstille en
tabel over sammenhængen mellem tiden $x$ og den tilbagelagte afstand $y$:
\[
\begin{tabular}{c|c|c|c|c|c|c}
    x ({\rm min}) & 10 & 20 & 30 & 40 & 60 & 120 \\
    \hline
    y ({\rm km})  & 15 & 30 & 45 & 60 & 90 & 180  
\end{tabular}
\]
Vi ser, at $y$ netop er $1,5$ gange så stor som $x$, så sammenhængen mellem $x$
og $y$ kan udtrykkes som
\[
y = 1,5\cdot x
\]
Talemåden "$y$ ændrer sig i samme takt som $x$" betyder netop, at $x$ skal
ganges med et fast tal (her $1,5$) for at give $y$. Man kan også udtrykke det
sådan:
\begin{itemize}
    \item Når $x$ bliver dobbelt så stor, bliver $y$ også dobbelt så stor.
    \item Når $x$ bliver tre gange så stor, bliver $y$ også tre gange så stor.
    \item Når $x$ bliver halvt så stor, bliver $y$ også halvt så stor.
    \item osv.
\end{itemize}
Hvis man med konstant hastighed kører dobbelt så lang tid, tilbagelægger man
også dobbelt så lang afstand, se figur 3.6.

\begin{figure}[H]
    \centering
    \includegraphics[width=0.5\textwidth]{lin-5}
    \caption{Figur 2.5}
    \label{fig36}
\end{figure}

I dette tilfælde er tidsrummet $x$ og afstanden $y$ proportionale, og tallet
$1,5$ kaldes {\em proportionalitetsfaktoren}, fordi det netop er en faktor foran $x$.

Man kan også sige, at forholdet mellem $y$ og $x$ er konstant, fordi vi får:
\[
y = 1,5\cdot x \, \Leftrightarrow \, \frac{y}{x} = 1,5
\]
Forholdet mellem afstand og tid kaldes hastighed, og tallet angiver her netop
hastigheden: bilen kører med en hastighed på $1,5$ km/min, svarende til $90$
km/t.

Sammenhængen illustreres som en ret linje gennem $(0,\,0)$ med hældningen $1,5$.

\begin{mydef}
    To variabler $x$ og $y$ kaldes proportionale, hvis der findes et tal $k$, så
    $$
    y = k\cdot x \,\, {\rm eller} \,\, \frac{y}{x} = k
    $$
    Tallet $k$ kaldes proportionalitetsfaktoren. Sammenhængen mellem $x$ og $y$
    illustreres i koordinatsystemet af en ret linje gennem $(0,\,0)$ med 
    hældning $k$.
\end{mydef}

\subsection{Eksempel (sammenhæng mellem masse og volumen)}

Et eksempel på ligefrem proportionalitet er sammenhængen mellem masse og rumfang af olie.
Der gælder nemlig, at en liter olie vejer 0,8 kg. To liter olie vejer derfor 1,6 kg,
og generelt vejer $x$ liter olie $y$ kg, hvor
\[
y = 0,8 \cdot x \,.
\]
Tallet $0,8$ er således vægten af én liter olie og det kaldes for oliens
massefylde.  At olie flyder oven på vand skyldes netop, at oliens massefylde er
mindre end vands.

\subsection{Eksempel}
I eksempel 1.1 så vi på teleselskabet, hvor prisen $y$ (naturligvis) afhænger af
taletiden $x$ i minutter:
\[
y=20+0,5x \,.
\]
Her er prisen ikke proportional med taletiden: Dobbelt så lang taletid koster ikke
det dobbelte:
\begin{itemize}
    \item 15 minutter koster $20+0,5\cdot 15 = 27,50$ kroner.
    \item 30 minutter koster $20+0,5\cdot 30 = 35,00$ kroner.
\end{itemize}
I koordinatsystemet går linjen med ligningen $y=20+0,5x$ nemlig ikke gennem (0, 0),
sådan som kravet til proportionalitet er.


%%%%%%%%%%%%%%%%%%%%%%%%%%%%%%%%%%%%%%%%%%%%%%%%%%%%%%%%%%%%%%%%%

\section{Linje gennem to punkter}
Vi vil bestemme hældningen for en linje, der går gennem
to punkter med kendte koordinater, se figuren. Vi viser følgende sætning:
\begin{thm}
    Hvis $A(x_1,\,y_1)$ og $B(x_2,\,y_2)$ er to punkter på en ret linje, der ikke
    er lodret, dvs $x_1\neq x_2$, er hældningen $a$ af linjen bestemt ved
    $$
    a = \frac{y_2-y_1}{x_2-x_1}
    $$
\end{thm}

\begin{figure}[H]
    \centering
    \includegraphics[width=0.5\textwidth]{lin-6}
    \label{linear-1}
\end{figure}

\begin{proof}
    Linjens ligning er
    $$
    y = a\cdot x + b.
    $$
    De punkter, som ligger på linjen, har koordinater, der passer i
    ligningen.  Da $A$ og $B$ ligger på linjen, passer deres koordinater altså
    i ligningen, dvs.
    $$
    y_1 = a\cdot x_1 + b \quad {\rm og} \quad y_2 = a\cdot x_2 + b 
    $$
    Vi trækker den første ligning fra den sidste og får
    \bas
    y_2 - y_1 &=& a\cdot x_2 + b - (a\cdot x_1 + b) \\
              &=& a\cdot x_2 - a\cdot x_1 \\
              &=& a\cdot \left(x_2-x_1\right) 
    \eas
    altså $y_2-y_1 = a \cdot \left(x_2-x_1\right)$, hvoraf
    $$
    a = \frac{y_2-y_1}{x_2-x_1}
    $$
    og det var netop, hvad vi ville vise.
\end{proof}

Læg mærke til, at vi i den sidste ligning har divideret med tallet $x_2-x_1$ på
begge sider af lighedstegnet. Dette tal er nemlig ikke $0$, fordi $x_1$ og
$x_2$ er forskellige tal -- vi har jo netop forudsat, at linjen ikke er lodret, se
figuren.

\begin{figure}[H]
    \centering
    \includegraphics[width=0.5\textwidth]{lin-7}
    \label{linear-2}
\end{figure}

I formlen for hældningen $a$ angiver tælleren $y_2-y_1$ den lodrette afstand
mellem punkterne $A$ og $B$, mens nævneren $x_2-x_1$ angiver den vandrette
afstand. Vi kan derfor lidt populært sige, at
\[
a = \pm \frac{\mbox{lodret afstand}}{\mbox{vandret afstand}} \,.
\]
Her står $\pm$ for at minde om, at der skal anbringes et minus foran brøken,
hvis hældningen er negativ.
\begin{eks}
    Vi ser på linjen $l$ gennem $A(-1,\,3)$ og $B(4,\,5)$.
    Linjens hældning er
    $$
    a = \frac{5-3}{4-(-1)} = \frac{2}{5} = 0,4.
    $$
    Skæringspunktet med $y$-aksen finder vi ved at indsætte det ene punkts
    koordinater i ligningen:
    $$
    5 = 0,4\cdot 4 + b.
    $$
    Det giver
    $$
    b = 5 - 0,4\cdot 4 = 3,4.
    $$
    Linjens ligning er dermed
    $$
    y = 0,4 \cdot x + 3,4 
    $$
\end{eks}
\begin{figure}[H]
    \centering
    \includegraphics[width=0.5\textwidth]{lin-8}
    \label{linear-3}
\end{figure}


%Når man har beregnet hældningen $a$ så vil man også gerne bestemme
%skæringspunktet med $y$-aksen.
%\begin{thm}
%    Hvis $A(x_1,\,y_1)$ er et punkt på en ret linje, og linjen har hældningen
%    $a$, så er skæringen $b$ bestemt ved
%    $$
%    b = y_1 - a\cdot x_1
%    $$
%\end{thm}
%\begin{proof}
%    Linjens ligning er 
%    $$
%    y = a\cdot x + b
%    $$
%    og netop de punkter, som ligger på linjen, har koordinater, der passer i
%    ligningen.  Da $A$ ligger på linjen, passer dens koordinater altså i
%    ligningen, dvs.
%    $$
%    y_1 = a\cdot x_1 + b 
%    $$
%    Heri isoleres $b$:
%    $$
%    b = y_1 - a\cdot x_1
%    $$
%Dermed har vi vist det ønskede.
%\end{proof}

\begin{thm}
    Hvis $A(x_1,\,y_1)$ er et punkt på en ret linje, og linjen har hældningen
    $a$, så er linjens ligning givet ved
    $$
    y = a\cdot (x-x_1) + y_1 
    $$
\end{thm}
\begin{proof}
    Vi har generelt, at 
    $$
    y = a\cdot x + b
    $$
    og i punnket $(x_1, y_1)$ gælder:
    $$
    y_1 = a\cdot x_1 + b
    $$
    Trækkes den nederste ligning fra den øverste får vi:
    $$
    y-y_1 = a \cdot x + b - (a \cdot x_1 + b)
    $$
    $$
    \Leftrightarrow y-y_1 = a \cdot x - a \cdot x_1
    $$
    $$
    \Leftrightarrow y-y_1 = a \cdot (x - x_1)
    $$
    Dermed har vi vist det ønskede.
\end{proof}


%%%%%%%%%%%%%%%%%%%%%%%%%%%%%%%%%%%%%%%%%%%%%%%%%%%%%%%%%%%%%%%%%%
%
%\section{Intervaller -- kopi af afsnit fra Mat-A1 bogen}
%
%Vi kender den sædvanlige tallinje med begyndelsespunkt $O$ og det såkaldte
%enhedspunkt $E$, der svarer til tallet 1. Længden af linjestykket $OE$ er altså 1.
%
%\begin{figure}[ht]
%    \centering
%    \includegraphics[width=0.5\textwidth]{fig23}
%    \caption{Figur 2.3}
%    \label{fig23}
%\end{figure}
%
%Vi ser på visse dele af tallinjen, de såkaldte intervaller.
%
%\subsection{Begrænsede intervaller}
%Hvis $a$ og $b$ er to tal, hvor $a < b$, skriver vi (se fig. 2.3):
%\begin{itemize}
%    \item $[a;b]$ : alle tal mellem $a$ og $b$, $a$ og $b$ medregnet,
%    \item $[a;b[$ : alle tal fra og med $a$ til, men ikke med $b$,
%    \item $]a;b]$ : alle tal fra $a$ til $b$, men ikke med $a$,
%    \item $]a;b[$ : alle tal mellem $a$ og $b$, $a$ og $b$ ikke medregnet .
%\end{itemize}
%Disse afsnit af tallinjen kaldes {\em intervaller}, og $a$ og $b$ er deres {\em endepunkter}.
%
%Vi siger, at et interval er
%\begin{itemize}
%    \item lukket, hvis begge endepunkter er med i intervallet,
%    \item halvåbent, hvis det ene, men ikke det andet endepunkt er med i intervallet,
%    \item åbent, hvis ingen af endepunkterne er med i intervallet.
%\end{itemize}
%Intervaller af disse fire typer er alle begrænsede, og deres længde er $b-a$.
%
%\begin{eks}
%Intervallet $[3;7[$ består af alle tal på tallinjen mellem 3 og 7; 3 er
%medregnet og 7 er ikke medregnet. Intervallets længde er $7-3 = 4$ .
%
%Intervallet $[-5;8]$ består af alle tal mellem -5 og 8, og både -5 og 8
%tilhører intervallet. Dets længde er 13 fordi $8 - (-5) = 13$ .
%
%På tallinjen er et intervals længde altså afstanden mellem endepunkterne.
%\end{eks}
%[INDSÆT FIGUR HER]
%
%\subsection{Ubegrænsede intervaller}
%Vi får tit brug for at se på tal, der på tallinjen ligger til højre eller til
%venstre for et givet tal (fig. 2.4), fx alle tal der er større end 17, eller
%alle tal der er mindre end eller lig med 50.
%
%Det giver anledning til disse skrivemåder:
%
%\begin{itemize}
%    \item $[a;\infty[$  : alle tal større end eller lig med $a$
%    \item $]a;\infty[$  : alle tal større end $a$
%    \item $]-\infty;b]$ : alle tal mindre end eller lig med $b$
%    \item $]-\infty;b[$ : alle tal mindre end $b$
%\end{itemize}
%Tegnet $\infty$ læses ’uendelig’.
%
%Intervallerne af typerne $]a;\infty [$ og $]-\infty ;b[$ kaldes åbne, da $a$ og $b$
%ikke medregnes, og intervallerne af typerne $[a;\infty [$ og $]-\infty ;b]$
%kaldes lukkede, da $a$ og $b$ medregnes.
%
%Disse intervaller er ubegrænsede, fordi de ikke har nogen længde – det
%fortsætter ’i det uendelige’ i den ene ende af tallinjen.
%
%Endelig kan man også skrive intervallet $]-\infty ;\infty [$. Dette angiver
%alle tal på tallinjen.

%\section{Uligheder (kopi af Mat-B1 HTX)}
%Vi har tidligere set på en række forskellige ligningstyper. Fælles for disse
%ligninger er, at de alle indeholder et lighedstegn. Vi skal nu se på et emne,
%der under ét benævnes uligheder. Uligheder optræder også i teknikkens verden.
%Skibe, der skal kunne besejle Panama-kanalen, må højest være 32,3 meter bredde.
%
%$$
%B\leq 32,3 \, \text{m}
%$$
%Temperaturen skal være mindst $800^\circ$ i en given proces:
%
%
%$$
%T \geq 800 ^\circ C
%$$
%Som det ses, optræder der ulighedstegn. Dem findes der fire af, to skarpe og to svage:
%
%\begin{itemize}
%    \item Mindre end, (skarp) $<$
%    \item Større end, (skarp) $>$
%    \item Mindre end eller lig med, (svag) $\leq$
%    \item Større end eller lig med, (svag) $\geq$
%\end{itemize}
%
%Et par eksempler på uligheder:
%
%$$
%x-2>7 \;\;\; \text{og} \;\;\; 3\cdot x-12\geq 4
%$$
%
%Når vi skal løse uligheder, er teknikken stort set den samme som ved ligninger, dog med en enkelt vigtig undtagelse.
%
%Et led kan flyttes fra den ene side af ulighedstegnet til den anden side ved at skifte fortegn:
%$$
%x+3<7 \Leftrightarrow x<7-3 \Leftrightarrow x<4
%$$
%$$
%13-x>3 \Leftrightarrow 13>3+x \Leftrightarrow 10>x
%$$
%Bemærk, at en ulighed kan læses fra to sider:
%$10 < x$ er det samme som $x > 10$
%
%Ved multiplikation og division skal vi være forsigtige. Det er tydeligt for enhver, at:
%$$
%2<7
%$$
%Vi ganger med et positivt tal på begge sider:
%$$
%3 \cdot 2 <3 \cdot 7 \Leftrightarrow 6<21
%$$
%Det gik fint.
%
%Vi prøver at nu at gange med et negativt tal, her -1, på begge sider af ulighedstegnet:
%$$
%(-1) \cdot 2 < (-1) \cdot 7 \Leftrightarrow -2<-7
%$$
%Det går ikke! -2 er jo større end -7. Vi er åbenbart nødt til at vende ulighedstegnet om:
%$$
%2<7 \Leftrightarrow (-1) \cdot 2 > (-1) \cdot 7 \Leftrightarrow -2>-7
%$$
%Ud fra eksemplerne kan vi opstille følgende regler:
%\begin{enumerate}
%    \item Et led flyttes fra den ene side af ulighedstegnet til den anden side ved at skifte fortegn på leddet.
%    \item Man må gange og dividere med det samme positive tal, undtagen 0, på begge sider af ulighedstegnet.
%    \item Man må gange og dividere med det samme negative tal på begge sider af ulighedstegnet, hvis man samtidig vender ulighedstegnet om.
%\end{enumerate}
%
%\begin{eks}
%Vi løser uligheden:
%$$
%3 \cdot x - 7 \geq 8
%$$
%Led kan flyttes som i ligninger:
%$$
%3 \cdot x \geq 8+7 \Leftrightarrow 3 \cdot x \geq 15
%$$
%Vi må gange og dividere med positive tal:
%$$
%3 \cdot x \geq 15 \Leftrightarrow x \geq 5
%$$
%\end{eks}
%
%\begin{eks}
%Vi løser denne ulighed:
%$$
%-4 \cdot x + 1 \leq 21
%$$
%Et led flyttes:
%$$
%-4 \cdot x \leq 21 -1 \Leftrightarrow -4 \cdot x \leq 20
%$$
%Nu skal vi passe på! Vi isolerer x ved at dividere med -4 på begge sider og skal huske at vende ulighedstegnet:
%$$
%-4 \cdot x \leq 20 \Leftrightarrow x \geq-5
%$$
%
%Vi kunne have grebet opgaven anderledes an:
%$$
%-4 \cdot x + 1 \leq 21
%$$
%$$
%-4 \cdot x + 1 \leq 21 \Leftrightarrow 1 - 21 \leq 4 \cdot x
%$$
%Nu er koefficienten foran x positiv:
%$$
%1-21 \leq 4 \cdot x \Leftrightarrow -20 \leq 4 \cdot x \Leftrightarrow -5 \leq x
%$$
%Hvis vi læser løsningen ”baglæns” fås:
%$$
%x \geq -5
%$$
%Altså samme løsning som før. Heldigvis!
%\end{eks}
%
%\subsection{Dobbeltuligheder}
%En ulighed med to ulighedstegn, der vender samme vej (hvorfor skal de vende samme vej?) kaldes for en dobbeltulighed:
%$$
%x-1 < 2x + 3 < x+11
%$$
%
%Den løses ved at opdele uligheden i to uligheder:
%$$
%x - 1 < 2x + 3 \;\;\; \text{og} \;\;\; 2x + 3 < x + 11
%$$
%der løses med den angivne metode, og hvor den endelige løsning skal opfylde begge kriterier. I det angivne tilfælde skal der gælde:
%$$
%x>-4 \;\;\; \text{og} \;\;\; x<8
%$$
%Og dermed skal der alt i alt gælde at
%$$
%-4 < x < 8
%$$
%%
\end{document}

