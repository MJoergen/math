\documentclass[12pt,oneside,a4paper]{article}

\usepackage[utf8]{inputenc} % Lærer LaTeX at forstå unicode - HUSK at filen skal
% være unicode (UTF-8), standard i Linux, ikke i
% Win.

\usepackage[danish]{babel} % Så der fx står Figur og ikke Figure, Resumé og ikke
% Abstract etc. (god at have).


\usepackage{graphicx}
\usepackage{amsfonts}
\usepackage{amsthm}        % Theorems
\usepackage{amsmath}
%\usepackage{hyperref}
\usepackage{float}
\usepackage{tcolorbox}


\newcommand{\bas}{\begin{eqnarray*}}
\newcommand{\eas}{\end{eqnarray*}}
\newcommand{\be}{\begin{equation}}
\newcommand{\ee}{\end{equation}}
\newcommand{\bea}{\begin{eqnarray}}
\newcommand{\eea}{\end{eqnarray}}

\theoremstyle{plain}
\newtheorem*{thm}{Sætning}
\newtheorem*{mydef}{Definition}
\newtheorem*{eks}{Eksempel}

\DeclareMathSymbol{,}{\mathord}{letters}{"3B}

\title{Logaritmefunktioner}
\date{\vspace{-5ex}}

\begin{document}

\maketitle

\section*{Indledning}
Vi har i kapitel 6 behandlet rødder og potenser, og bl.a. brugt disse
til at løse visse typer ligninger som f.eks. $x^{1,5} = 8$, hvor den ukendte
variabel er opløftet til en vilkårlig potens.
I dette kapitel introduceres logaritmefunktioner, som bl.a. bruges
til at løse en anden type ligninger som f.eks. $5,0625^x = 1,5$, hvor den ukendte
variabel optræder som eksponenten.

Logaritmerne blev oprindeligt opfundet ca. 1594 af den skotske teolog og matematiker John
Napier (1550-1617). Hans logaritmer var dem, vi i dag kalder de naturlige
logaritmer. Titalslogaritmernes fremkomst skyldes et samarbejde mellem Napier
og den engelske professor Henry Briggs (1561-1631). Denne logaritmefunktion har
stor betydning, fordi den hænger sammen med grundtallet 10 i vores talsystem.

\section*{Logaritmefunktionen $\log$}
Vi har set, at funktionerne $f(x) = x^2$ og $g(x) = \sqrt{x}$  er hinandens omvendte
($x \geq 0$), fordi de "ophæver" hinanden, når de sammensættes (dvs. udføres
efter hinanden).  Således er $g(9) = \sqrt{9} = 3$ netop fordi $f(3) = 3^2 = 9$.

På samme måde ser vi nu på eksponentialfunktionen $f(x) = 10^x$, som er
defineret for alle reelle tal $x$.

\begin{tcolorbox}
\begin{mydef}
Den omvendte funktion til eksponentialfunktionen $f(x) = 10^x$ kaldes {\em
logaritmefunktionen med grundtal 10}, og den betegnes $g(x) = \log x$.
\end{mydef}
\end{tcolorbox}

Vi kan skrive nogle værdier for de to funktioner op:

\[
\begin{aligned}
    &f(2) = 10^2 = 100 && \text{giver at} && g(100) = \log (100) = 2 \; , \\
    &f(1) = 10^1 = 10 && \text{giver at} && g(10) = \log (10) = 1 \; , \\
    &f(0,5) = 10^{0,5} = 3,162 && \text{giver at} && g(3,162) = \log (3,162) = 0,5 \; ,
\end{aligned}
\]

Da grafen for eksponentialfunktionen ligger over $x$-aksen, er det kun positive
tal, der har logaritmer, så definitionsmængden for logaritmefunktionen er
mængden af positive tal. Værdimængden er alle tal.

Vi har således følgende:
\begin{tcolorbox}
Logaritmen til et positivt tal er den eksponent, som 10 skal opløftes til for
at give tallet.
\end{tcolorbox}

Ovenfor så vi, at $\log 100 = 2$ fordi 2 netop er den eksponent, 10 skal
opløftes til for at give 100. På samme måde er log 5 = 0,6990, fordi 10 skal
opløftes til potensen 0,6990 for at give 5, altså $10^{0,6990} = 5$.
Funktionerne $10^x$ og $\log x$ er hinandens omvendte, og der gælder således,
jævnfør kapitel 5, at
\[
    \tag{1}
    \begin{aligned}
        10^{\log x } &= x,\quad \text{for $x > 0$} \\
        \log (10^x) &= x,\quad \text{for alle $x$\,.}
    \end{aligned}
\]

Logaritmefunktionen kaldes også 10-tals-logaritmen, og tallet 10 kaldes grundtallet.

\begin{figure}[H]
    \centering
    \includegraphics[width=0.5\textwidth]{log1}
    \caption{}
    \label{fig1}
\end{figure}

På figuren oven over er tegnet graferne for funktionerne $f(x) = 10^x$ og $g(x)
= \log x$. Da de er hinandens omvendte, ligger graferne symmetrisk omkring
linjen $y = x$, dvs.  de er hinandens spejlbilleder. Grundtallet for
logaritmefunktionen findes som det tal, hvis funktionsværdi (på y-aksen) er 1,
dvs. $\log 10 = 1$. Grundtallet er altså 10.

På næste figur er tegnet et nærbillede af graferne omkring (0, 0).
\begin{figure}[H]
    \centering
    \includegraphics[width=0.5\textwidth]{log2}
    \caption{}
    \label{fig2}
\end{figure}


\section*{Den naturlige logaritmefunktionen $\ln$}
Vi har defineret 10-tals-logaritmen $\log x$ som den omvendte funktion til
eksponentialfunktionen $10^x$ med grundtallet 10. På samme måde har den
naturlige eksponentialfunktion ${\rm e}^x$ en omvendt funktion, der kaldes
{\em den naturlige logaritmefunktion}. Den betegnes $\ln x$.

\begin{tcolorbox}
Som for 10-talslogaritmen kan vi sige, at den naturlige logaritme til et
positivt tal er den eksponent, som grundtallet e skal opløftes til for at give
tallet.
\end{tcolorbox}

Fx er (benyt CAS):
\[
\ln 7 = 1,9459 \text{ , fordi } e^{1,9459} = 7 \; .
\]

Graferne for den naturlige eksponentialfunktion og den naturlige
logaritmefunktion ses på den næste figur. De er hinandens spejlbilleder i linjen $y =
x$.

\begin{figure}[H]
    \centering
    \includegraphics[width=0.5\textwidth]{log3}
    \caption{}
    \label{fig3}
\end{figure}

Grafen for den naturlige eksponentialfunktion går gennem punktet (1, e), og
grafen for den naturlige logaritmefunktion går gennem (e, 1). Tallet e er
grundtallet for den naturlige logaritme, dvs. det tal, hvis funktionsværdi er
1.

Definitions- og værdimængde for den naturlige logaritmefunktion er som for
10-tals-logaritmen

\[
\mathrm{Dm}(\ln) = ]0; \infty[ \text{ \quad og\quad } \mathrm{Vm}(\ln) = ]-\infty; \infty[ \; .
\]

Formlerne (1) har deres paralleller for den naturlige eksponential- og
logaritmefunktion:

For den naturlige logaritme $\ln x$ gælder:

\[
    \tag{2}
    \begin{aligned}
        {\rm e}^{\ln x } &= x,\quad \text{for $x > 0$} \\
        \ln ({\rm e}^x) &= x,\quad \text{for alle $x$\,.}
    \end{aligned}
\]

\subsection*{Regneregler for logaritmer}

Den helt centrale rolle som logaritmer spiller i matematikken skyldes nogle
regneregler, som vi nedenfor viser for 10-tals-logaritmen. Reglerne er
imidlertid også gyldige for den naturlige logaritmefunktion.

\begin{tcolorbox}
\begin{thm}

For funktionerne $\log$ og $\ln$ gælder for alle positive tal a og b og alle tal x
følgende regler:

\[
    \begin{aligned}
        &\log (ab) = \log a + \log b \quad  &\ln(ab) = \ln a + \ln b  \\
        &\log \frac{a}{b} = \log a - \log b &\ln \frac{a}{b} = \ln a - \ln b \\
        &\log(a^x) = x \cdot \log a         &\ln(a^x) = x \cdot \ln a \, .
    \end{aligned}
\]
\end{thm}
\end{tcolorbox}

\begin{proof}

1.

Vi kan foretage følgende omskrivning:

\[
\begin{aligned}
    &\log (ab)& &\text{Benyt (1): } a = 10^{\log a} \text{ og } b = 10^{\log b} \\
    =&\log (10^{\log a} \cdot 10^{\log b}) && \text{Benyt potensregneregel: } 10^p \cdot 10^q = 10^{p + q} \\
    =& \log(10^{\log a + \log b}) &&\text{Benyt (1) : } \log(10^x) = x \text{ , hvor } x = \log a + \log b \\
    =& \log a + \log b \; . 
\end{aligned}
\]

2.

Den anden regel viser vi ved at foretage en række ensbetydende omskrivninger af formlen:
\[
\begin{aligned}
    &\log \frac{a}{b} = \log a - \log b &&\text{ Læg } \log b \text{ til på begge sider } \\ 
    \Leftrightarrow\quad&\log \frac{a}{b} + \log b = \log a &&\text{ Benyt regel 1 ovenfor om sum af to logaritmer} \\ 
    \Leftrightarrow\quad&\log \left( \frac{a}{b} \cdot b \right) = \log a &&\text{ Reducér venstre side } \\ 
    \Leftrightarrow\quad&\log a = \log a \;
. \end{aligned}
\]
Den sidste ligning er sand, og da den er ensbetydende med den første, er denne
også sand.

Naturligvis kunne vi have ført beviset på præcis samme måde som under 1 ved at
erstatte multiplikation med division og addition med subtraktion:

\[
    \log \frac{a}{b} = \log \frac{10^{\log a}}{10^{\log b}} = \log (10^{\log a - \log b}) = \log a - \log b
\]
3.

Vi benytter igen (1) og får omskrivningerne

\[
\begin{aligned}
    &&&\log (a^x) &&\text{Benyt (1): } a = 10^{\log a} \\
    &= &&\log ({(10^{\log a})}^x) &&\text{Benyt potensregneregel: } (10^p)^q = 10^ {pq} \\
    &= &&\log (10^{x \log a}) &&\text{Benyt (1): } \log 10^p = p \\
    &= &&x \log a \; .
\end{aligned}
\]
Dermed er sætningens tre regler vist.
\end{proof}

Ved hjælp af den sidste regel i sætning 1 kan vi også finde logaritmer af
rødder. Benyttes at $a^{\frac{1}{n}} = \sqrt[n]{a}$, fås nemlig

\[
\log \sqrt[n]{a} = \log a^{\frac{1}{n}} = \frac{1}{n} \log a \; .
\]
Specielt gælder for kvadratrødder, at

\[
\log \sqrt{a} = \frac{1}{2} \log a \; .
\]

\begin{eks}

Regnereglerne giver mulighed for at reducere udtryk med logaritmer:

\[
\begin{aligned}
    &\log 3 + \log 7 = \log 21 && \log (2x) - \log (5 - x) = \log \frac{2x}{5 - x} \\
    &\log x^5 = 5 \log x && 3 \ln 2 = \ln 2^3 = \ln 8
\end{aligned}
\]
\end{eks}

\subsection*{Ligninger med logaritmer}
Ved hjælp af regnereglerne kan vi løse flere forskellige typer at ligninger.
Vi ser på nogle af dem her.

\begin{tcolorbox}
\subsubsection*{Eksempel}
Vi ser først på ligninger der indeholder eksponentialfunktioner
\[
e^x = 5\,.
\]
Vi benytter formlen (2) for den naturlige logaritmefunktion:
\[
e^x = 5 \iff \ln (e^x) = \ln 5 \iff x = \ln 5\,.
\]
\end{tcolorbox}

\begin{tcolorbox}
\subsubsection*{Eksempel}
\[
    7^x = 83\,.
\]
Benyt en regneregel for log:
\[
7^x = 83 \iff \log 7^x = \log 83 \iff x \cdot \log 7 = \log 83 \iff 
\]
\[
    x = \frac{\log 83}{\log 7} = 2,2708\,.
\]
\end{tcolorbox}

\begin{tcolorbox}
\subsubsection*{Eksempel}

Vi ser på ligninger, der indeholder logaritmefunktioner. 

\[
\log x = 2,95.
\]

Vi benytter, at $\log x$ og $10^x$ er hinandens omvendte, og får

\[
\log x = 2,95 \iff x = 10^{2,95} = 891,25 \; .
\]
\end{tcolorbox}

\begin{tcolorbox}
\subsubsection*{Eksempel}
\[
\ln x = 1,49.
\]

Tilsvarende finder vi her, at

\[
\ln x = 1,49 \iff x = e^{1,49} = 4,4371 \; .
\]
\end{tcolorbox}


\begin{tcolorbox}
\subsubsection*{Eksempel}
Den eksponentielle udvikling f udvikler sig fra en begyndelsesværdi på 375 med
en vækstrate på 13,2\% pr. år. Hvor længe varer det, inden denne størrelse er
vokset til en værdi på 1200? Vi skal finde x, så

\[
375 \cdot 1,132^x = 1200 \iff
    1,132^x = \frac{1200}{375} = 3,2 \iff
    \]
\[
    \log 1,132^x = \log 3,2 \iff
    x \cdot \log 1,132 = \log 3,2 \iff
    x = \frac{\log 3,2}{\log 1,132} = 9,3813\,.
\]

Efter 9,4 år er den pågældende størrelse altså vokset til en værdi på 1200.
\end{tcolorbox}


\begin{tcolorbox}
\subsubsection*{Eksempel}

    I nogle sammenhænge angives eksponentielle udviklinger ved hjælp af tallet {\rm e}.
En størrelse udvikler sig efter forskriften

\[
    f(t) = 428 \cdot {\rm e}^{0,08t} \; ,
\]
hvor t angiver antallet af måneder efter begyndelsesværdien. Ved hjælp af en
potensregneregel får vi

\[
    f(t) = 428 \cdot ({\rm e}^{0,08})^t = 428 \cdot 1,083^t \, .
\]
Begyndelsesværdien er altså 428, fremskrivningsfaktoren 1,083 og vækstraten
8,3\% pr. måned.
\end{tcolorbox}

\begin{tcolorbox}
\subsubsection*{Eksempel}
Hvis vi på den anden side har den eksponentielle udvikling 
\[
    g(t) = 1200 \cdot 0,94^t\,,
    \]
og ønsker at bruge skrivemåden med e, kan vi benytte, at  $a = e^{\ln a}$, så 

\[
0,94^t = (e^{\ln 0,94})^t = e^{-0,0619t} \, .
\]

Dermed kan vi altså omskrive:
\[
    g(t) = 1200 \cdot e^{-0,0619t} \, .
\]
\end{tcolorbox}


\end{document}

