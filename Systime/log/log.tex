\documentclass[12pt,oneside,a4paper]{article}

\usepackage[utf8]{inputenc} % Lærer LaTeX at forstå unicode - HUSK at filen skal
% være unicode (UTF-8), standard i Linux, ikke i
% Win.

\usepackage[danish]{babel} % Så der fx står Figur og ikke Figure, Resumé og ikke
% Abstract etc. (god at have).

\usepackage{graphicx}
\usepackage{amsfonts}
\usepackage{amsthm}        % Theorems
\usepackage{amsmath}
%\usepackage{hyperref}

\newcommand{\bas}{\begin{eqnarray*}}
\newcommand{\eas}{\end{eqnarray*}}
\newcommand{\be}{\begin{equation}}
\newcommand{\ee}{\end{equation}}
\newcommand{\bea}{\begin{eqnarray}}
\newcommand{\eea}{\end{eqnarray}}

\theoremstyle{plain}
\newtheorem*{thm}{Sætning}
\newtheorem*{mydef}{Definition}
\newtheorem*{eks}{Eksempel}

\DeclareMathSymbol{,}{\mathord}{letters}{"3B}

\title{Logaritmefunktioner}
\date{\vspace{-5ex}}

\begin{document}

\maketitle

\section*{Indledning}
Vi har i kapitel 6 behandlet rødder og potenser, og bl.a. brugt disse
til at løse ligninger som f.eks. $x^{1,5} = 8$.
I dette kapitel introduceres logaritmefunktioner, som bl.a. bruges
til at løse ligninger som f.eks. $5,0625^x = 1,5$.

Logaritmerne blev opfundet ca. 1594 af den skotske teolog og matematiker John
Napier (1550-1617). Hans logaritmer var dem, vi i dag kalder de naturlige
logaritmer. Titalslogaritmernes fremkomst skyldes et samarbejde mellem Napier
og den engelske professor Henry Briggs (1561-1631). Denne logaritmefunktion har
stor betydning, fordi den hænger sammen med grundtallet 10 i vores talsystem.

Logaritmer blev helt frem til lommeregnerens fremkomst (ca. 1975) sammen med
regnestokken benyttet til at foretage vanskelige talregninger som fx $3,4598
\cdot 45,7$ eller $\frac{3,0457}{21,98}$. Det viser sig nemlig, at "vanskelige"
regningsarter som multiplikation og division kan erstattes med "simple"
regningsarter som addition og subtraktion.

\subsection*{Dette kapitel}

Vi indfører to logaritmefunktioner, den naturlige logaritme og
titalslogaritmen, og viser de grundlæggende regneregler der gælder for dem. Ved
hjælp af logaritmerne kan vi løse lignigner, hvor den ubekendte indgår i
eksponenten.  Desuden kan vi bestemme fordoblings- og halveringskonstant for
eksponentielle udviklinger. Endelig viser vi forskellige vigtige anvendelser af
logaritmer.

\section*{Logaritmefunktionen $\log$}
Vi har set, at funktionerne $f(x) = x^2$ og $g(x) = \sqrt{x}$  er hinandens omvendte
($x \geq 0$), fordi de "ophæver" hinanden, når de sammensættes (dvs. udføres
efter hinanden). Fig. 1.2 viser en måde at illustrere at kvadratfunktionen og
kvadratrodsfunktionen er et par af omvendte funktioner. Vi har tegnet en pil
fra $x$-aksen til $y$-aksen for at illustrere kvadratfunktionen og en pil den
modsatte vej anskueliggør kvadratrodsfunktionen.

Derefter ser vi på eksponentialfunktionen $f(x) = 10^x$, som er defineret for alle
reelle tal $x$. Her kan vi på samme måde med pile fra $x$-aksen til $y$-aksen (fig.
1.3) illustrere eksponentialfunktionen og med pile de modsatte vej den omvendte
funktion. Vi fastsætter ved hjælp af dette diagram følgende definition:

Den omvendte funktion til eksponentialfunktionen $f(x) = 10^x$ kaldes
logaritmefunktionen med grundtal 10, og den betegnes $g(x) = \log x$.

Vi kan skrive nogle værdier for de to funktioner op:

\[
\begin{aligned}
    &f(2) = 10^2 = 100 && \text{giver at} && g(100) = \log (100) = 2 \; , \\
    &f(1) = 10^1 = 10 && \text{giver at} && g(10) = \log (10) = 1 \; , \\
    &f(0,5) = 10^{0,5} = 3,162 && \text{giver at} && g(3,162) = \log (3,162) = 0,5 \; ,
\end{aligned}
\]

Da grafen for eksponentialfunktionen ligger over x-aksen, er der kun positive
tal, der har logaritmer, så definitionsmængden for logaritmefunktionen er
mængden af positive tal. Værdimængden er alle tal.

Vi har, at logaritmen til et positivt tal er den eksponent, som 10 skal
opløftes til for at give tallet.

Ovenfor så vi, at $\log 100 = 2$ fordi 2 netop er den eksponent, 10 skal opløftes
til for at give 100. På samme måde er log 5 = 0,6990, fordi 10 skal opløftes
til potensen 0,6990 for at give 5, altså $10^{0,6990} = 5$. I almindelighed kan
vi skrive dette sådan:

\[
\tag{1} 10^{\log x } = x \text{ og } \log (10^x) = x
\]

Logaritmefunktionen kaldes også 10-tals-logaritmen, og tallet 10 kaldes grundtallet.

På fig. 1.4 er tegnet graferne for funktionerne $f(x) = 10^x$ og $g(x) = \log x$. Da
de er hinandens omvendte, ligger graferne symmetrisk omkring linjen y = x, dvs.
de er hinandens spejlbilleder. Grundtallet for logaritmefunktionen findes som
det tal, hvis funktionsværdi (på y-aksen) er 1 : $\log 10 = 1$.

På fig. 1.5 er tegnet et nærbillede af graferne omkring (0, 0).

\section*{Den naturlige logaritmefunktionen $\ln$}
Vi har defineret 10-tals-logaritmen log som den omvendte funktion til
eksponentialfunktionen $\exp_{10}$ med grundtallet 10. På samme måde har den
naturlige eksponentialfunktion en omvendt funktion, der kaldes den naturlige
logaritmefunktion. Den betegnes $\ln$.

Som for 10-talslogaritmen kan vi sige, at

Den naturlige logaritme til et positivt tal er den eksponent, som e skal
opløftes til for at give tallet.

Fx er (benyt cas)

\[
\ln 7 = 1,9459 \text{ , fordi } e^{1,9459} = 7 \; .
\]

Graferne for den naturlige eksponentialfunktion og den naturlige
logaritmefunktion ses på fig. 1.6. De er hinandens spejlbilleder i linjen y =
x.

Grafen for den naturlige eksponentialfunktion går gennem punktet (1, e), og
grafen for den naturlige logaritmefunktion går gennem (e, 1). Tallet e er
grundtallet for den naturlige logaritme, dvs. det tal, hvis funktionsværdi er
1.

Definitions- og værdimængde for den naturlige logaritmefunktion er som for
10-tals-logaritmen

\[
\mathrm{Dm}(\ln) = ]0; \infty[ \text{ og } \mathrm{Vm}(\ln) = ]-\infty; \infty[ \; .
\]

Grafen for den naturlige eksponentialfunktion har som nævnt i punktet (0, 1) en
tangent med hældning 1 og ligningen y = x + 1. Ved spejlingen i y = x føres
denne tangent over i en tangent til den naturlige logaritmefunktion, og denne
tangent får ligningen $y = x - 1$.

Formlerne (1) har deres paralleller for den naturlige eksponential- og
logaritmefunktion:

For den naturlige logaritme $\ln$ gælder for alle positive tal x

\[
\tag{2} e^{\ln x} = x \text{ og } \ln (e^x) = x \; .
\]

På fig. 1.7 ses graferne for 10-tals-logaritmen og den naturlige logaritme. Læg
mærke til, at grafen for 10-tals-logaritmen forløber "fladere" end grafen for
den naturlige logaritme. I øvrigt er de to logaritmefunktioner proportionale,
dvs. den enes funktionsværdier fås af den andens ved at gange med en konstant.
Der gælder, at værdierne for log er ca. 43\% af værdierne for $\ln$, idet $\log x
\approx 0,4343 \ln x$.

\subsection*{Regneregler for logaritmer}

Den helt centrale rolle som logaritmer spiller i matematikken skyldes nogle
regneregler, som vi nedenfor viser for 10-tals-logaritmen. Reglerne er
imidlertid også gyldige for den naturlige logaritmefunktion.

SÆTNING 1.1: LOGARITMEREGNEREGLERNE ID

For funktionerne $\log$ og $\ln$ gælder for alle positive tal a og b og alle tal x
følgende regler:

\[
\log (ab) = \log a + \log b \quad \ln(ab) = \ln a + \ln b
\log \frac{a}{b} = \log a - \log b \quad \ln \frac{a}{b} = \ln a - \ln b
\log(a^x) = x \cdot \log a \quad \ln(a^x) = x \cdot \ln a \; .
\]

BEVIS FOR SÆTNING 1.1 ID

1.

Vi kan foretage følgende omskrivning:

\[
\begin{aligned}
    &\log (ab)& &\text{Benyt (1): } a = 10^{\log a} \text{ og } b = 10^{\log b} \\
    =&\log (10^{\log a} \cdot 10^{\log b}) && \text{Benyt potensregneregel: } 10^p \cdot 10^q = 10^{p + q} \\
    =& \log(10^{\log a + \log b}) &&\text{Benyt (1) : } \log(10^x) = x \text{ , hvor } x = \log a + \log b \\
    =& \log a + \log b \; . 
\end{aligned}
\]

2.

Den anden regel viser vi ved at foretage en række ensbetydende omskrivninger af formlen:
\[
\begin{aligned}
    &\log \frac{a}{b} = \log a - \log b &&\text{ Læg } \log b \text{ til på begge sider } \\ 
    &\log \frac{a}{b} + \log b = \log a &&\text{ Benyt regel 1 ovenfor om sum af to logaritmer} \\ 
    &\log \left( \frac{a}{b} \cdot b \right) = \log a &&\text{ Reducér venstre side } \\ 
    &\log a = \log a \;
. \end{aligned}
\]
Den sidste ligning er sand, og da den er ensbetydende med den første, er denne
også sand.

Naturligvis kunne vi have ført beviset på præcis samme måde som under 1 ved at
erstatte multiplikation med division og addition med subtraktion:

\[
    \log \frac{a}{b} = \log \frac{10^{\log a}}{10^{\log b}} = \log (10^{\log a - \log b}) = \log a - \log b
\]
3.

Vi benytter igen (1) og får omskrivningerne

\[
\begin{aligned}
    &&&\log (a^x) &&\text{Benyt (1): } a = 10^{\log a} \\
    &= &&\log ({(10^{\log a})}^x) &&\text{Benyt potensregneregel: } (10^p)^q = 10^ {pq} \\
    &= &&\log (10^{x \log a}) &&\text{Benyt (1): } \log 10^p = p \\
    &= &&x \log a \; .
\end{aligned}
\]
Dermed er sætningens tre regler vist.

Ved hjælp af den sidste regel i sætning 1.1 kan vi også finde logaritmer af
rødder. Benyttes at $a^{\frac{1}{n}} = \sqrt[n]{a}$, fås nemlig

\[
\log \sqrt[n]{a} = \log a^{\frac{1}{n}} = \frac{1}{n} \log a \; .
\]
Specielt gælder for kvadratrødder, at

\[
\log \sqrt{a} = \frac{1}{2} \log a \; .
\]

EKSEMPEL 1.1 ID

Regnereglerne giver mulighed for at reducere udtryk med logaritmer:

\[
\begin{aligned}
    &\log 3 + \log 7 = \log 21 && \log (2x) - \log (5 - x) = \log \frac{2x}{5 - x} \\
    &\log x^5 = 5 \log x && 3 \ln 2 = \ln 2^3 = \ln 8
\end{aligned}
\]

\subsection*{Ligninger med logaritmer}
Forskellige typer af ligninger kan indeholde den ubekendte i logaritme- og
eksponentialfunktioner. Vi ser på nogle af dem her.

EKSEMPEL 1.2 ID

Vi ser først på ligninger der indeholder eksponentialfunktioner

\[
\rightarrow e^x = 5
\]

Vi benytter formlen (2) for den naturlige logaritmefunktion:

\[
e^x = 5 \iff \ln (e^x) = \ln 5 \iff x = \ln 5
\rightarrow 7^x = 83
\]

Benyt en regneregel for log:

\[
7^x = 83 \iff \log 7^x = \log 83 \iff
x \cdot \log 7 = \log 83 \iff x = \frac{\log 83}{\log 7} = 2,2708
\]

Fig. 1.8

Et udsnit af en logaritmetabel fra 1922, nemlig Siebenstellige gemeine
Logarithmen der Zahlen von 1 bis 108000. Tabellen har en 5-cifret indgang og
giver værdien for titalslogaritmen med 7 decimaler. Fx ser vi, at $\log 3,5562 =
0,5509862$.

EKSEMPEL 1.3 ID

Vi ser på ligninger, der indeholder logaritmefunktioner. 

\[
\rightarrow \log x = 2,95.
\]

Vi benytter, at $\log$ og $\exp_{10}$ er hinandens omvendte, og får

\[
\log x = 2,95 \iff x = 10^{2,95} = 891,25 \; .
\rightarrow \ln x = 1,49.
\]

Tilsvarende finder vi her, at

\[
\ln x = 1,49 \iff x = e^{1,49} = 4,4371 \; .
\]

EKSEMPEL 1.4 ID

Den eksponentielle udvikling f udvikler sig fra en begyndelsesværdi på 375 med
en vækstrate på 13,2\% pr. år. Hvor længe varer det, inden denne størrelse er
vokset til en værdi på 1200? Vi skal finde x, så

\[
375 \cdot 1,132^x = 1200 \iff x = 9,3813
\]

Efter 9,4 år er den pågældende størrelse altså vokset til en værdi på 1200.

EKSEMPEL 1.5 ID

I nogle sammenhænge angives eksponentielle udviklinger ved hjælp af tallet e.
En størrelse udvikler sig efter forskriften

\[
f(t) = 428 \cdot e^{0,08t} \; ,
\]
hvor t angiver antallet af måneder efter begyndelsesværdien. Ved hjælp af en
potensregneregel får vi

\[
f(t) = 428 \cdot (e^{0,08})^t = 428 \cdot 1,083^t \; .
\]
Begyndelsesværdien er altså 428, fremskrivningsfaktoren 1,083 og vækstraten
8,3\% pr. måned.

Hvis vi på den anden side har den eksponentielle udvikling $g(t) = 1200 \cdot 0,94^t$,
og ønsker at bruge skrivemåden med e, kan vi benytte, at  $a = e^{\ln a}$, så 

\[
0,94^t = (e^{\ln 0,94})^t = e^{-0,0619t} \; .
\]


\end{document}

