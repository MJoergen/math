\documentclass[12pt,oneside,a4paper]{article}

\usepackage[utf8]{inputenc} % Lærer LaTeX at forstå unicode - HUSK at filen skal
% være unicode (UTF-8), standard i Linux, ikke i
% Win.

\usepackage[danish]{babel} % Så der fx står Figur og ikke Figure, Resumé og ikke
% Abstract etc. (god at have).

\usepackage{graphicx}
\usepackage{amsfonts}
\usepackage{amsthm}        % Theorems
\usepackage{amsmath}
\usepackage{enumitem}
%\usepackage{hyperref}

%\renewcommand{\mid}[1]{{\rm E}\!\left[#1\right]}
\newcommand{\bas}{\begin{eqnarray*}}
\newcommand{\eas}{\end{eqnarray*}}
\newcommand{\be}{\begin{equation}}
\newcommand{\ee}{\end{equation}}
\newcommand{\bea}{\begin{eqnarray}}
\newcommand{\eea}{\end{eqnarray}}

\theoremstyle{plain}
\newtheorem*{thm}{Sætning}
\newtheorem*{mydef}{Definition}
\newtheorem*{eks}{Eksempel}

\DeclareMathSymbol{,}{\mathord}{letters}{"3B}

\title{Projektopgaver om potensfunktioner}
\date{\vspace{-5ex}}

\begin{document}

\maketitle

\section*{Opgave 1 -- Keplers 3. lov}
Planeternes bevægelse rundt om Solen har været genstand for spekulation i
årtusinder.  Da den tyske matematiker Johannes Kepler i år 1600 blev ansat hos
den danske astronom Tycho Brahe, så begyndte et bemærkelsesværdigt samarbejde,
der ledte frem til en dybere forståelse af planetbevægelserne. På baggund af
Brahe's detaljerede observationer lykkedes det Kepler i 1618 at gennemskue
systemet i planeternes bevægelse, og formulerede det som tre love.

I denne opgave skal vi undersøge Keplers 3. lov, som omhandler sammenhængen
mellem en planets omløbstid rundt om Solen, og planetens afstand til Solen.

Den følgende tabel viser sammenhørende værdier for planeternes afstand $x$ til
Solen (målt i antal millioner km) og omløbstid $y$ omkring Solen (målt i år).

\vspace{2ex}

\begin{center}
\begin{tabular}{r|r|r}
    \hline
            & $x$                 & $y$ \\
    Planet  & afstand i mio km    & omløbstid i år   \\
    \hline 
    Merkur  &  $  57,9$ &   $0,241$ \\
    Venus   &  $ 108,2$ &   $0,615$ \\
    Jorden  &  $ 149,6$ &   $1,000$ \\
    Mars    &  $ 227,9$ &   $1,881$ \\
    Jupiter &  $ 778,4$ &  $11,863$ \\
    Saturn  &  $1426,7$ &  $29,448$ \\
    Uranus  &  $2871,0$ &  $84,017$ \\
    Neptun  &  $4498,3$ & $164,791$ \\
    \hline 
\end{tabular}
\end{center}

\begin{enumerate}[label=(\alph*)]
    \item Lav en figur, hvor du viser planeternes omløbstid om Solen som
        funktion af deres afstand til Solen.
    \item Lav en lineær regression og bestem forskriften på formen $y=ax+b$ og
        diskutér modellens gyldighed.
\end{enumerate}

I stedet for en simpel lineær model skal du nu omregne tabellens data ved hjælp af 
10-tals logaritmen.
\begin{enumerate}[label=(\alph*) ,resume]
    \item Lav en ny tabel, hvor du beregner 10-tals logaritmen til både afstand
        og omløbstid, se nedenstående.
\end{enumerate}

\begin{center}
\begin{tabular}{r|r|r}
    \hline
    Planet  & $\log(x)$   & $\log(y)$ \\
    \hline 
    Merkur  &  $1,76$ &   $-0,618$ \\
    \ldots  & \ldots    & \ldots  \\
    \hline 
\end{tabular}
\end{center}

\begin{enumerate}[label=(\alph*) ,resume]
    \item Lav en figur, hvor du viser $\log(y)$ som funktion af $\log(x)$.
    \item Lav en lineær regression på denne figur og bestem forskriften på formen
        $\log(y) = a\cdot\log(x)+b$ og diskutér modellens gyldighed.
    \item Omskriv forskriften til en potensfunktion på formen $y = b\cdot x^a$. \label{pot}
\end{enumerate}

Keplers lov ses ofte formuleret på følgende måde:
$$
y^2 = k \cdot x^3
$$
\begin{enumerate}[label=(\alph*) ,resume]
    \item Vis, at svaret i spørgsmål~\ref{pot} kan omskrives til ovenstående formel.
\end{enumerate}


Udover de kendte planeter, så findes der også en række dværgplaneter, der
ligesom planeterne bevæger sig rundt om Solen. En af disse dværgplaneter er
Ceres, som befinder sig i afstanden ca. 414 millioner km fra Solen.

\begin{enumerate}[label=(\alph*) ,resume]
    \item Beregn omløbstiden i år for dværgplaneten Ceres.
\end{enumerate}

\section*{Opgave 2 -- Newtons tyngdelov}
Mange sammenhænge i fysik kan beskrives ved potensfunktioner. Newton
postulerede i 1687, at tyngdekraften mellem to masser aftager med kvadratet på
afstanden.  Denne påstand bliver verificeret i denne opgave.

Tyngdeaccelerationen på Jorden kendes som $g=9,82\, {\rm m}/{\rm s}^2$.
Ligeledes er Jordens radius en kendt værdi, nemlig $x=6730$ km.

Vi skal nu først bestemme værdien af tyngdeaccelerationen i samme af\-stand som Månens.
Denne kan beregnes som
\be
a = \frac{v^2}{r}
\label{acc}
\ee
hvor $v$ er månens fart og $r$ er månens afstand til Jordens centrum. 

\begin{enumerate}[label=(\alph*)]
    \item Vis, at ovenstående formel for tyngdeaccelerationen kan omskrives til 
        $$
        a = \left(\frac{2\pi}{T}\right)^2 \cdot r,
        $$
        hvor $T$ er Månens omløbstid om Jorden. \label{a}
\end{enumerate}

Månens omløbstid er 27,32 dage og dens afstand til Jorden er 384400 km.

\begin{enumerate}[label=(\alph*) ,resume]
    \item Omregn Månens omløbstid til sekunder og afstand til meter, og vis dermed, at tyngdeaccelerationen
        i Månens afstand har værdien $0,00272\, {\rm m}/{\rm s}^2$.
\end{enumerate}

Dette giver følgende tabel over sammenhængen mellem afstand til Jordens centrum og størrelsen af tyngdeaccelerationen:

\begin{center}
\begin{tabular}{r|r|r}
    \hline
          & $x$                         & $y$ \\
    Sted  & afstand til Jordens centrum & tyngdeacceleration   \\
          & i millioner meter           & i ${\rm m}/{\rm s}^2$  \\
    \hline 
    Jordens overflade & $6,73$  & $9,82$ \\
    ved Månen         & $384,4$ & $0,00272$ \\
    \hline 
\end{tabular}
\end{center}

Newton påstår nu, at sammenhængen mellem afstand og tyngdeacceleration kan beskrives ved en potensfunktion.

\begin{enumerate}[label=(\alph*) ,resume]
    \item Beregn værdierne af $a$ og $b$ i den potensfunktion $y=b\cdot x^a$,
        som går gennem de to punkter angivet i tabellen. \label{p1}
\end{enumerate}

Newton formulerede sin lov som
$$
a = \frac{k}{r^2},
$$
hvor $r$ er afstanden til centrum og $a$ er tyngdeaccelerationen.

\begin{enumerate}[label=(\alph*) ,resume]
    \item Vis, at svaret i~\ref{p1} kan omskrives til ovenstående formel.
\end{enumerate}

Nu kombinerer vi svarene i spørgsmål~\ref{a} og~\ref{n}, og vi får følgende ligning:
$$
\frac{b}{r^2} = \left(\frac{2\pi}{T}\right)^2 \cdot r,
$$
hvor $r$ er afstanden til Jordens centrum, og $T$ er omløbstiden
rundt om Jorden.
\begin{enumerate}[label=(\alph*) ,resume]
    \item Vis, at denne ligning kan omskrives til Keplers lov $T^2 = k\cdot r^3$. \label{newt}
\end{enumerate}

Newtons og Keplers love gælder derfor både for bevægelse rundt om Jorden og rundt om Solen.


\section*{Opgave 3 -- Den Internationale Rumstation ISS}
Den Internationale Rumstation befinder sig i en højde af ca. 386 km over Jordens overflade.
\begin{enumerate}[label=(\alph*)]
    \item Beregn afstanden fra den Internationale Rumstation til Jordens centrum.
    \item Brug ligningen i Opgave 2 spørgsmål~\ref{newt} til at beregne Rumstationens omløbstid.
\end{enumerate}


\section*{Opgave 4 -- Geostationære satellitter}
En geostationær satellit har en omløbstid som er lig med Jordens egen rotationstid.
\begin{enumerate}[label=(\alph*)]
    \item Hvad er omløbstiden for en geostationær satellit (målt i dage)?
    \item Brug ligningen i Opgave 2 spørgsmål~\ref{newt} til at beregne afstanden fra en geostationær
        satellit til Jordens centrum.
    \item Beregn herefter en geostationær satellits højde over Jordens overflade.
\end{enumerate}

\end{document}

