\documentclass[12pt,oneside,a4paper]{article}

\usepackage[utf8]{inputenc} % Lærer LaTeX at forstå unicode - HUSK at filen skal
% være unicode (UTF-8), standard i Linux, ikke i
% Win.

\usepackage[danish]{babel} % Så der fx står Figur og ikke Figure, Resumé og ikke
% Abstract etc. (god at have).

\usepackage{graphicx}
\usepackage{amsfonts}
\usepackage{amsthm}        % Theorems
\usepackage{amsmath}
\usepackage{enumitem}
%\usepackage{hyperref}

%\renewcommand{\mid}[1]{{\rm E}\!\left[#1\right]}
\newcommand{\bas}{\begin{eqnarray*}}
\newcommand{\eas}{\end{eqnarray*}}
\newcommand{\be}{\begin{equation}}
\newcommand{\ee}{\end{equation}}
\newcommand{\bea}{\begin{eqnarray}}
\newcommand{\eea}{\end{eqnarray}}

\theoremstyle{plain}
\newtheorem*{thm}{Sætning}
\newtheorem*{mydef}{Definition}
\newtheorem*{eks}{Eksempel}

\DeclareMathSymbol{,}{\mathord}{letters}{"3B}

\title{Projektopgave om potensfunktioner}
\date{\vspace{-5ex}}

\begin{document}

\maketitle

\section*{Indledning}
Planeternes bevægelse rundt om Solen har været genstand for spekulation i årtusinder.
Da den tyske matematiker Johannes Kepler i år 1600 blev ansat hos den danske
astronom Tycho Brahe, så begyndte et bemærkelsesværdigt samarbejde, der ledte frem til
en dybere forståelse af planetbevægelserne. På baggund af Brahe's detaljerede observationer
lykkedes det Kepler i 

\section*{Opgave 1}
Den følgende tabel viser sammenhørende værdier for planeternes afstand til Solen og
omløbstid omkring Solen.

\vspace{2ex}

\begin{center}
\begin{tabular}{r|r|r}
    \hline
    Planet  & afstand   & omløbstid \\
            & $a$/AU    & $T$/år   \\
    \hline 
    Merkur  &  $0,3871$ &   $0,2408$ \\
    Venus   &  $0,7233$ &   $0,6152$ \\
    Jorden  &  $1,0000$ &   $1,0000$ \\
    Mars    &  $1,5237$ &   $1,8808$ \\
    Jupiter &  $5,2034$ &  $11,8626$ \\
    Saturn  &  $9,5371$ &  $29,4475$ \\
    Uranus  & $19,1913$ &  $84,0168$ \\
    Neptun  & $30,0690$ & $164,7913$ \\
    \hline 
\end{tabular}
\end{center}

\begin{enumerate}[label=\alph*.]
    \item Lav en figur, hvor du viser planeternes omløbstid om Solen som
        funktion af deres afstand til Solen.
    \item Lav en lineær regression og diskutér modellens gyldighed.
\end{enumerate}

I stedet for en simpel lineær model skal du nu omregne tabellens data ved hjælp af 
10-tals logaritmen.
\begin{enumerate}[label=\alph*. ,resume]
    \item Lav en ny tabel, hvor du beregner 10-tals logaritmen til både afstand
        og omløbstid, se nedenstående.
\end{enumerate}

\begin{center}
\begin{tabular}{r|r|r}
    \hline
    Planet  & $\log(a)$   & $\log(T)$ \\
    \hline 
    Merkur  &  $-0,4122$ &   $-0,6183$ \\
    \ldots  & \ldots    & \ldots  \\
    \hline 
\end{tabular}
\end{center}

\begin{enumerate}[label=\alph*. ,resume]
    \item Lav en figur, hvor du viser $\log(T)$ som funktion af $\log(a)$.
    \item Lav en lineær regression på denne figur og diskutér modellens gyldighed.
    \item Opskriv forskiften på formen $\log(T) = \ldots$.
    \item Omskriv forskiften til formen $T = \ldots$.
\end{enumerate}

\end{document}

