\documentclass[12pt,oneside,a4paper]{article}

\usepackage[utf8]{inputenc} % Lærer LaTeX at forstå unicode - HUSK at filen skal
% være unicode (UTF-8), standard i Linux, ikke i
% Win.

\usepackage[danish]{babel} % Så der fx står Figur og ikke Figure, Resumé og ikke
% Abstract etc. (god at have).

\usepackage{graphicx}
\usepackage{amsfonts}
\usepackage{amsthm}        % Theorems
\usepackage{amsmath}
%\usepackage{hyperref}

%\renewcommand{\mid}[1]{{\rm E}\!\left[#1\right]}
\newcommand{\bas}{\begin{eqnarray*}}
\newcommand{\eas}{\end{eqnarray*}}
\newcommand{\be}{\begin{equation}}
\newcommand{\ee}{\end{equation}}
\newcommand{\bea}{\begin{eqnarray}}
\newcommand{\eea}{\end{eqnarray}}

\newtheorem{thm}{Sætning}[section]
\newtheorem{mydef}[thm]{Definition}
\newtheorem{eks}[thm]{Eksempel}

\DeclareMathSymbol{,}{\mathord}{letters}{"3B}

\title{Eksponentielle funktioner}

\begin{document}

\maketitle

\section{Indledning}
\section{Renteformlen}
Hvis en størrelse på 800 ({\em begyndelsesværdien}) vokser med $4\%$ pr. måned
({\em vækstraten}) i en længere periode, kan vi beregne dens værdi til
forskellige tidspunkter:
\begin{itemize}
    \item Efter 1 måned er værdien $800\cdot 1,04 = 832$.
    \item Efter 2 måneder er værdien $832\cdot 1,04 = 800 \cdot 1,04 \cdot 1,04
        = 800 \cdot 1,04^2 = 865,28$.
    \item Efter 3 måneder er værdien $865,28\cdot 1,04 = 800 \cdot 1,04^2 \cdot
        1,04 = 800 \cdot 1,04^3 = 899,89$.
\end{itemize}
Sådan kan vi åbenbart fortsætte. Efter $n$ måneder er værdien vokset til
$800\cdot 1,04^n$. Hvis ændringen havde været med procenten $r$ (skrevet som
decimaltal), havde vi efter $n$ måneder fået en værdi på $800\cdot(1+r)^n$.

Tidsrummet 'måned' kunne have været et andet (uge, år, \ldots). I almindelighed
bruger vi den mere neutrale betegnelse {\em termin} for tidsrummet mellem to
ændringer.

Hvis vi betegner værdien efter $n$ terminer med $K_n$ og begyndelsesværdien med
$K_0$, gælder åbenbart følgende formel:
\begin{eks}
    {\em Renterformlen}. En størrelse med begyndelsesværdien $K_0$ ændrer sig
    med procenter $r$ pr. termin. Efter $n$ terminer er størrelsen da ændret
    til slutværdien $K_n$, hvor
    $$
    K_n = K_0 \cdot (1+r)^n
    $$
\end{eks}
Formlen kaldes {\em renteformlen}, fordi man kan opfatte situationen, som om en
person sætter $K_0$ kr. ({\em begyndelseskapitalen}) ind på en konto i et
pengeinstitut. Beløbet står derefter til en {\em rentefod} (vækstrate) på $r$
pr. termin i $n$ terminer, uden at der hæves eller sættes yderligere beløb ind.
Så er beløbet efter $n$ terminer vokset til $K_n$ ({\em slutkapitalen}).

Selv beskedne udsving i rentefoden kan gennem tiden have store virkninger. På figuren
ses, hvordan en størrelse med begyndelsesværdien 2 (eller 200 eller 20000 \ldots) udvikler
sig gennem tiden med rentefødder på $6\%$, $9\%$ og $12\%$.

Renteformlen er et eksempel på en eksponentiel funktion. Vi ser nu på en mere generel
definition af eksponentielle funktioner.

\section{Eksponentielle funktioner}
Vi kalder funktioner af typen
$$
f(x) = ba^x
$$
for {\em eksponentielle funktioner}. Tallene $a$ og $b$ er konstanter. Følgende funktioner er 
eksempler på eksponentielle funktioner:
$$
\begin{array}{rcll}
    f(x) &=& 3 4^x & (a=4, b=3) \\
    g(x) &=& 4 0.91^z & (a=0.91, b=4) 
\end{array}
$$

\begin{eks}
    Renteformlen svarer til ovenstående definition, hvis man sætter
    \bas
    f(x) = K_n & : & \hbox{værdien efter $n$ terminer.} \\
       b = K_0 & : & \hbox{begyndelsesværdi.} \\
       a = 1+r & : & \hbox{fremskrivningsfaktor.} \\
       x = n   & : & \hbox{antallet af terminer.}
    \eas
\end{eks}

\subsection{Konstanternes betydning}
\subsubsection{Konstanten $b$}
Figuren viser grafen for de to eksponentielle funktioner $f(x)$ og $g(x)$
angivet ovenfor. Vi bemærker, at grafen i alle tre tilfælde går igennem punkter $(0, b)$.
Da
$$
f(0) = b\cdot a^0 = b
$$
må det være tilfældet for en hvilken som helst eksponentiel funktion.

\subsubsection{Konstanten $a$}
For at få et indtryk af betydningen af $a$, sætter vi $b=1$ og tegner grafen
for en række forskellige værdier af $a$, se figuren. Vi iagttager, at 
\begin{itemize}
    \item Når $a>0$ er $f(x)$ voksende, og grafen krummer opad (grafen er {\em konveks}).
    \item Når $a<0$ er $f(x)$ aftagende, og grafen nærmer sig mere og mere
        $x$-aksen når $x$ bliver større og større.
\end{itemize}
I begge tilfælde siger man også, at $x$-aksen er en vandret {\em asymptote}.
De karakteristisks træk ved eksponentialfunktioner er samlet på figuren.

\subsection{Vækstegenskaber}
Ekspontialfunktioner kaldes også {\em absolut-procent vækst}. Betegnelsen dækker over følgende egenskab:
{\em Når $x$ ændres med en fast værdi $h$, så ændres $y$ med en fast procent $r$.}
Som et eksempel på denne type vækst ser vi på den eksponentielle funktion
$$
f(x) = 2,15 \cdot 1,12^x
$$
Forøges $x_1 = 2$ med $1,3$ fås $x_2 = x_1 + 1,3 = 2 + 1,3 = 3,3$. De tilsvarende 
funktionsværdier er 
$$
y_1 = f(x_1) = f(2) = 2,15 \cdot 1,12^2 = 2,697
$$
og
$$
y_2 = f(x_2) = f(3,3) = 2,15 \cdot 1,12^3,3 = 3,125
$$
hvilket er en relativ stigning i y-værdierne på 
$$
\frac{y_2-y_1}{y_1} \cdot 100\% = \frac{3,125 - 2,697}{2,697} \cdot 100 \% = 16 \%.
$$
Resultatet af en tilsvarende beregning for hhv $x_1 = 10$ og $x_1 = 15$ er vist i tabellen nedenfor:
$$
\begin{array}{|c|c|c|c|c|}
    \hline 
    x_1 & x_2 = x_1 + 1,3 & y_1 = f(x_1) & y_2 = f(x_2) & \frac{y_2-y_1}{y_1} \cdot 100 \% \\
    \hline 
    2 & 3,3 & 2,697 & 3,125 & 16\% \\
    \hline 
    10 & 11,3 & 6,678 & 7,738 & 16\% \\
    \hline 
    15 & 16,3 & 11,768 & 13,636 & 16\% \\
    \hline
\end{array}
$$
Det interessant er her, at vi i alle tre tilfælde med en stigning i $x$-værdierne på $1,3$ får en ændring
i $y$-værdierne på $16\%$.

Resultatet er tilsyneladende uafhængigt af, hvilken $x$-værdi, der tages udgangspunkt i. Den følgende beregning viser,
at det er tilfældet:
$$
f(x + 1,3) = 2,15 \cdot 1,12^{x+1,3} = 2,15 \cdot 1,12^x \cdot 1,12^1,3 = 2,15 \cdot 1,12^x \cdot 1,16 = f(x) \cdot 1,16.
$$
Beregningen viser, at hvis der lægges $1,3$ til $x$, så skal $f(x)$ ganges med $1,16$. Det er det samme som at sige,
at når $x$ stiger med 1,13, så stiger $f(x)$ med $16\%$.

\begin{thm}
    For en eksponentiel funktion $f(x) = ba^x$ gælder:
    Hvis $x$ ændres med en fast værdi $h$, så ændres $f(x)$ med en fast procent $r_y$. Skrives $r_y$ som decimaltal gælder formlen
    $$
    1+r_y = a^h
    $$
\end{thm}
\begin{proof}
    For en hvilken som helst positiv konstant $h$ har vi
    $$
    f(x+h) = ba^{x+h} = ba^x a^h = f(x) a^h
    $$
    hvilket viser, at når $x$ stiger med $h$, så fremskrives $y=f(x)$ med faktoren $a^h$. Lader vi $r_y$ som decimaltal være bestemt
    ved
    $$
    a^h = 1+r_y
    $$
    så viser ligningen $f(x+h) = f(x) a^h$, at når $x$ stiger med værdien $h$, så stiger $y=f(X)$ med $r_y$ procent.
\end{proof}

\begin{eks}
    Vi ser på $f(x) = 21,5 \cdot 1,2^x$ og ønsker at svare på følgende spørgsmål:
    \begin{enumerate}
        \item Hvor meget vokser $f(x)$ med, når $x$ vokser med $3,1$?
        \item Hvor meget skal $x$ aftage med, for at $f(x)$ aftager med $58\%$?
    \end{enumerate}

    Løsning:
    \begin{enumerate}
        \item Vi benytter sætningen oven for med $h=3,1$:
            $$
            1+r_y = 1,2^3,1 = 1,76 \Leftrightarrow r_y = 0,76
            $$
            Funktionsværdien $f(x)$ vokser altså med $76\%$.
        \item Vi benytter samme sætning, men nu med $r_y = -58\% = -0,58$.
            $$
            1-0,58 = 1,2^h \Leftrightarrow 0,42 = 1,2^h \Leftrightarrow \ln(0,42) = \ln(1,2^h) \Leftrightarrow
            $$
            $$
            \ln(0,42) = h\cdot \ln(1,2) \Leftrightarrow h = -0,21.
            $$
            Dvs $x$ skal aftage med $0,21$.
    \end{enumerate}
\end{eks}



\end{document}

