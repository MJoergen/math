\documentclass[12pt,oneside,a4paper]{article}

\usepackage[utf8]{inputenc} % Lærer LaTeX at forstå unicode - HUSK at filen skal
% være unicode (UTF-8), standard i Linux, ikke i
% Win.

\usepackage[danish]{babel} % Så der fx står Figur og ikke Figure, Resumé og ikke
% Abstract etc. (god at have).

\usepackage{graphicx}
\usepackage{amsfonts}
\usepackage{amsthm}        % Theorems
\usepackage{amsmath}
%\usepackage{hyperref}

%\renewcommand{\mid}[1]{{\rm E}\!\left[#1\right]}
\newcommand{\bas}{\begin{eqnarray*}}
\newcommand{\eas}{\end{eqnarray*}}
\newcommand{\be}{\begin{equation}}
\newcommand{\ee}{\end{equation}}
\newcommand{\bea}{\begin{eqnarray}}
\newcommand{\eea}{\end{eqnarray}}

\newtheorem{thm}{Sætning}[section]
\newtheorem{mydef}[thm]{Definition}
\newtheorem{eks}[thm]{Eksempel}

\DeclareMathSymbol{,}{\mathord}{letters}{"3B}

\title{Eksponentielle funktioner}

\begin{document}

\maketitle

\section{Indledning}
\section{Renteformlen}
\section{Eksponentielle funktioner}
Vi kalder funktioer af typen
$$
f(x) = ba^x
$$
for {\em eksponentielle funktioner}. Tallene $a$ og $b$ er konstanter. Følgende funktioner er 
eksempler på eksponentielle funktioner:
$$
\begin{array}{rcll}
    f(x) &=& 3 4^x & (a=4, b=3) \\
    g(x) &=& 4 0.91^z & (a=0.91, b=4) 
\end{array}
$$

\begin{eks}
    Man støder ofte på eksponentielle funktioner
    \begin{itemize}
        \item Renteformlen kan skrives som
            $$
            f(x) = K_0 (1+r)^x
            $$
            hvor $a=1+r$ og $b=K_0$.
        \item 
    \end{itemize}
\end{eks}

\subsection{Konstanternes betydning}
\subsubsection{Konstanten $b$}
Figuren viser grafen for de to eksponentielle funktioner $f(x)$ og $g(x)$
angivet ovenfor. Vi bemærker, at grafen i alle tre tilfælde går igennem punkter $(0, b)$.
Da
$$
f(0) = b\cdot a^0 = b
$$
må det være tilfældet for en hvilken som helst eksponentiel funktion.

\subsubsection{Konstanten $a$}
For at få et indtryk af betydningen af $a$, sætter vi $b=1$ og tegner grafen
for en række forskellige værdier af $a$, se figuren. Vi iagttager, at 
\begin{itemize}
    \item Når $a>0$ er $f(x)$ voksende, og grafen krummer opad (grafen er {\em konveks}).
    \item Når $a<0$ er $f(x)$ aftagende, og grafen nærmer sig mere og mere
        $x$-aksen når $x$ bliver større og større.
\end{itemize}
I begge tilfælde siger man også, at $x$-aksen er en vandret {\em asymptote}.
De karakteristisks træk ved eksponentialfunktioner er samlet på figuren.

\subsection{Vækstegenskaber}
Ekspontialfunktioner kaldes også {\em absolut-procent vækst}. Betegnelsen dækker over følgende egenskab:
{\em Når $x$ ændres med en fast værdi $h$, så ændres $y$ med en fast procent $r$.}
Som et eksempel på denne type vækst ser vi på den eksponentielle funktion
$$
f(x) = 2,15 \cdot 1,12^x
$$
For $x=2$ finder vi $f(2) = 2,15 \cdot 1,12^2 = $


\end{document}

