\documentclass[12pt,oneside,a4paper]{article}

\usepackage[utf8]{inputenc} % Lærer LaTeX at forstå unicode - HUSK at filen skal
% være unicode (UTF-8), standard i Linux, ikke i
% Win.

\usepackage[danish]{babel} % Så der fx står Figur og ikke Figure, Resumé og ikke
% Abstract etc. (god at have).

%\usepackage{graphicx}
\usepackage{amsfonts}
\usepackage{amsthm}        % Theorems
\usepackage{amsmath}
%\usepackage{hyperref}

%\renewcommand{\mid}[1]{{\rm E}\!\left[#1\right]}
\newcommand{\bas}{\begin{eqnarray*}}
\newcommand{\eas}{\end{eqnarray*}}
\newcommand{\be}{\begin{equation}}
\newcommand{\ee}{\end{equation}}
\newcommand{\bea}{\begin{eqnarray}}
\newcommand{\eea}{\end{eqnarray}}

\newtheorem{thm}{Sætning}[section]
\newtheorem{mydef}[thm]{Definition}
\newtheorem{eks}[thm]{Eksempel}

\DeclareMathSymbol{,}{\mathord}{letters}{"3B}

\title{Linearitet}
\date{September 2015}

\begin{document}

\maketitle

\section{Indledning}
I dette afsnit skal vi arbejde med lineære variabelsammenhænge.
Vi begynder med at kigge på nogle eksempler.

\subsection{Eksempel}
Et teleselskab har et simpelt takstsystem:
\begin{itemize}
    \item Abonnement 20 kr. pr. måned, samt 0,50 kr. pr. talt minut.
\end{itemize}
Hvis en abonnent en måned taler 25 minutter, skal han betale:
$$
20 + 0,50\cdot25 = 32,50 \, {\rm kr},
$$
og tales der $x$ minutter en bestemt måned, er prisen $y$ kroner bestemt ved:
$$
y = 20 + 0,50\cdot x
$$
Man siger, at der er en {\em lineær sammenhæng} mellem samtaletiden og prisen.

Hvis vi tegner sammenhængen i et koordinatsystem med antallet af minutter på
$x$-aksen og prisen på $y$-aksen, så vil grafen danne en ret linje med hældning
$0,50$ og som skærer $y$-aksen i tallet $20$, se figur 3.5.

Hældningen $0,50$ angiver prisstigningen for hvert ekstra minut, der tales.
Således er prisen for 20 minutters samtale 30 kr., og for 21 minutters
samtale 30,50 kr.

\subsection{Eksempel}
En prøvekørsel af en bil begynder ved 5-km stenen uden for en by. Bilen kører
en længere strækning med 90 km/t. Efter 8 minutters kørsel befinder bilen sig
ved 17-km-stenen (hvorfor?), og efter $x$ minutter er den ved $y$-km-stenen,
hvor
$$
y = 5 + 1,5\cdot x
$$
Her er der tale om en lineær sammenhæng mellem tiden i minutter og tallet på
kilometerstenen.

\section{Lineære sammenhænge}
\begin{mydef}
    En variabelsammenhæng er en sammenhæng mellem en uafhængig variabel, ofte
    kaldet $x$, og en afhængig variabel, ofte kaldet $y$.
\end{mydef}

\begin{mydef}
    En lineær sammenhæng er en variabelsammenhæng givet ved følgende ligning:
    $$
    y = a\cdot x + b
    $$
    hvor $a$ og $b$ er tal.
\end{mydef}

\begin{thm}
    Grafen for en lineær sammenhæng er en ret linje.  Linjens placering i et
    koordinatsystem er bestemt af tallen $a$ og $b$.  Således er $a$ linjens
    hældning, dvs $y$ vokser med $a$ hver gang $x$ vokser med 1.  Linjen skærer
    $y$-aksen i punktet $(0; b)$.  Tallet $a$ kaldes derfor linjens "hældning"
    og tallet $b$ kaldes linjens "skæring med $y$-aksen".
\end{thm}

Eksempler på lineære sammenhænge er:

\begin{tabular}{ll}
    $\bullet\quad y=3x-5$  & Her er $a=3$ og $b=-5$. \\
    $\bullet\quad y=-2x+1$ & Her er $a=-2$ og $b=1$. \\
    $\bullet\quad y=4-x$   & Her er $a=-1$ og $b=4$. \\
    $\bullet\quad y=3x$    & Her er $a=3$ og $b=0$.
\end{tabular}

[Billede af forskellige rette linjer]

Vi kan opstille en tabel ("sildeben") over koordinater $(x; y)$, der passer i
ligningen $y=3x-5$:
$$
\begin{tabular}{c|c|c|c|c|c|c}
    x &  -2 & -1 &  0 &  1 & 2 & 3 \\
    \hline
    y & -11 & -8 & -5 & -2 & 1 & 4
\end{tabular}
$$
Linjen går altså gennem punkterne $(0; -5)$, $(1; -2)$ osv, se figur 3.2.

\section{Ligefrem proportionalitet}
Man bruger den talemåde, at to variabler $x$ og $y$ er {\em ligefrem
proportionale} (eller blot: proportionale), hvis de "stiger og falder i samme
takt". Vi skal se på, hvad denne lidt løse udtalelse dækker over.

Hvis en bil kører med konstant hastighed, f.eks. 90 km/t, kan vi opstille en
tabel over sammenhængen mellem tiden $x$ og den tilbagelagte afstand $y$:
$$
\begin{tabular}{c|c|c|c|c|c|c}
    x ({\rm min}) & 10 & 20 & 30 & 40 & 60 & 120 \\
    \hline
    y ({\rm km})  & 15 & 30 & 45 & 60 & 90 & 180  
\end{tabular}
$$
Vi ser, at $y$ netop er $1,5$ gange så stor som $x$, så sammenhængen mellem $x$
og $y$ kan udtrykkes som
$$
y = 1,5\cdot x
$$
Talemåden "$y$ ændrer sig i samme takt som $x$" betyder netop, at $x$ skal
ganges med et fast tal (her $1,5$) for at give $y$. Man kan også udtrykke det
sådan:
\begin{itemize}
    \item Når $x$ bliver dobbelt så stor, bliver $y$ også dobbelt så stor.
    \item Når $x$ bliver tre gange så stor, bliver $y$ også tre gange så stor.
    \item osv.
\end{itemize}
Hvis man med konstant hastighed kører dobbelt så lang tid, tilbagelægger man
også dobbelt så lang afstand, se figur 3.6.

I dette tilfælde er tidsrummet $x$ og afstanden $y$ proportionale, og tallet
$1,5$ kaldes proportionalitetsfaktoren, fordi det netop er en faktor foran $x$.

Man kan også sige, at forholdet mellem $y$ og $x$ er konstant, fordi vi får:
$$
y = 1,5\cdot x \, \Leftrightarrow \, \frac{y}{x} = 1,5
$$
Forholdet mellem afstand og tid kaldes hastighed, og tallet angiver her netop
hastigheden: bilen kører med en hastighed på $1,5$ km/min, svarende til $90$
km/t.

Sammenhængen illustreres som en ret linje gennem $(0;0)$ med hældningen $1,5$.

\begin{mydef}
    To variabler $x$ og $y$ kaldes proportionale, hvis der findes et tal $k$, så
    $$
    y = k\cdot x \, {\rm eller} \, \frac{y}{x} = k
    $$
    Tallet $k$ kaldes proportionalitetsfaktoren. Sammenhængen mellem $x$ og $y$
    illustreres i koordinatsystemet af en ret linje gennem $(0; 0)$ med 
    hældning $k$.
\end{mydef}

\section{Linje gennem to punkter}
Vi vil bestemme hældningen for en linje, der går gennem
to punkter med kendte koordinater. Vi viser følgende sætning:
\begin{thm}
    Hvis $A(x_1; y_1)$ og $B(x_2; y_2)$ er to punkter på en ret linje, der ikke
    er lodret, er hældningen $a$ bestemt ved
    $$
    a = \frac{y_2-y_1}{x_2-x_1}
    $$
\end{thm}
\begin{proof}
    Linjens ligning er 
    $$
    y = a\cdot x + b
    $$
    og netop de punkter, som ligger på linjen, har koordinater, der passer i
    ligningen.  Da $A$ og $B$ ligger på linjen, passer deres koordinater altså
    i ligningen, dvs.
    $$
    y_1 = a\cdot x_1 + b \, {\rm og} \, y_2 = a\cdot x_2 + b 
    $$
    Vi trækker den første ligning fra den sidste og får
    \bas
    y_2 - y_1 &=& a\cdot x_2 + b - (a\cdot x_1 + b) \\
              &=& a\cdot x_2 - a\cdot x_1 \\
              &=& a\cdot \left(x_2-x_1\right) 
    \eas
    altså
    $$
    a = \frac{y_2-y_1}{x_2-x_1}
    $$
    og det var netop, hvad vi ville vise.
\end{proof}

Læg mærke til, at vi i den sidste ligning har divideret med tallet $x_2-x_1$ på
begge sider af lighedstegnet. Dette tal er nemlig ikke $0$, fordi $x_1$ og
$x_2$ er forskellige tal -- vi har jo netop forudsat, at linjen ikke er lodret.

I formlen for hældningen $a$ angiver tælleren $y_2-y_1$ den lodrette afstand
mellem punkterne $A$ og $B$, mens nævneren $x_2-x_1$ angiver den vandrette
afstand. Vi kan derfor lidt populært sige, at
$$
a = \pm \frac{\rm lodret afstand}{\rm vandret afstand}
$$
Her står $\pm$ for at minde om, at der skal anbringes et minus foran brøken,
hvis hældningen er negativ.
\begin{eks}
    Vi ser på linjen $m$ gennem $A(-1; 3)$ og $B(4; 5)$.
    Linjens hældning er
    $$
    a = \frac{5-3}{4-(-1)} = \frac{2}{5}
    $$
\end{eks}

Når man har beregnet hældningen $a$ så vil man også gerne bestemme skæringen
$b$.
\begin{thm}
    Hvis $A(x_1; y_1)$ er et punkt på en ret linje, og linjen har hældningen
    $a$, så er skæringen $b$ bestemt ved
    $$
    b = y_1 - a\cdot x_1
    $$
\end{thm}
\begin{proof}
    Linjens ligning er 
    $$
    y = a\cdot x + b
    $$
    og netop de punkter, som ligger på linjen, har koordinater, der passer i
    ligningen.  Da $A$ ligger på linjen, passer dens koordinater altså i
    ligningen, dvs.
    $$
    y_1 = a\cdot x_1 + b 
    $$
    Heri isoleres $b$:
    $$
    b = y_1 - a\cdot x_1
    $$
\end{proof}

Bemærk, at linjens ligning kan også skrives som
\bas
y &=& a\cdot x + y_1 - a\cdot x_1 \\
  &=& a\cdot (x-x_1) + y_1 
\eas


\section{Løse førstegradsligninger}

\section{Skæring mellem linjer}
Hvis to linjers ligninger er kendt, vil vi finde koordinaterne til linjernes
skæringspunkt (hvis de ikke er parallelle).

Linjerne $m$ og $n$ har f.eks. ligningner
$$
m: y=\frac{3}{4}x+2\,{\rm og}\,n: y=-\frac{1}{2}x+4\frac{1}{2}
$$
I skæringspunktet mellem linjerne skal $y$-værdien for de to linjer være den
samme, så der må gælde, at
$$
\frac{3}{4}x+2=-\frac{1}{2}x+4\frac{1}{2}
$$
og denne ligning løses lettest ved at gange med brøkernes fællesnævner $4$ på
begge sider:
$$
3x+8 = -2x+18\,\Leftrightarrow\,5x=10\,\Leftrightarrow\,x=2.
$$
Denne værdi af $x$ indsættes i en af ligningerne, ligegyldig hvilken:
$$
y=-\frac{1}{2}\cdot 2+4\frac{1}{2} = -1+4\frac12 = 3\frac12.
$$
Skæringspunktet har altså koordinaterne $(2; 3\frac12)$, hvilket ser ud til at
stemme med figuren.

\section{Lineær regression og vækst}
I mange tilfælde undersøger vi data, som kun tilnærmelsesvist passer med en ret
linje. Vi ser på et eksempel.

\subsection{Måling af sands densitet}
[Billede af måleglas med sand ovenpå en vægt]

Vi stiller et måleglas ovenpå en vægt og hælder gradvist sand i glasset. Vi
aflæser sandets rumfang i ${\rm cm}^3$ og kalder det for $x$. Vi aflæser også
vægtens visning i gram og kalder det for $y$. Det kunne f.eks. give følgende
tabel:
$$
\begin{tabular}{c|c|c|c|c|c|c}
    x &   0 &  50 & 100 & 150 & 200 & 250 \\
    \hline
    y & 300 & 400 & 500 & 600 & 700 & 800
\end{tabular}
$$
Hvis vi afsætter disse punkter i et koordinatsystem, så får vi følgende figur:

[Billede af graf af ovenstående punkter]

Vi ser således, at når mængden af sand øges, så stiger vægten. 
På figuren er også indtegnet en ret linje, og det er tydeligt, at punkterne
ligger meget tæt på denne linje.  Der er altså en sammenhæng mellem vægtens
visning og mængden af sandet. Da grafen passer med en ret linje, kalder vi
denne sammenhæng for en {\em lineær sammenhæng}.

Hældningen $a$ fortolkes som det vægten stiger med (i gram), når der tilføres 
$1\, {\rm cm}^3$ sand. Dette kaldes for sandets {\em densitet}. Vi ser således, 
at sand har en densitet på ca. $1,8\, {\rm g}/{\rm cm}^3$.

Med et CAS-værktøj er det muligt at beregne den "bedste" rette linje ud fra
nogle givne punkter. Dette kaldes {\em lineær regression}, og den beregnede
linje kaldes {\em regressionslinjen}. Nogle steder kaldes det også for
tendenslinjen.

\subsection{Korrelation}
Det er muligt at angive et tal for, hvor godt linjen passer med de givne
punkter.  De fleste CAS-værktøjer vil udover linjens ligning og give en {\em
korrelationskoefficient}.  Denne vil altid være mellem $0$ og $1$. Værdien er
præcis $1$ netop når punkterne ligger præcist på en ret linje.  I praksis vil
man anse den lineære sammenhæng for god, når korrelationskoefficienten er
$0,99$ eller derover.


\subsection{Et lys der brænder}
[Billede af et brændende stearinlys på en vægt]

Vi stiller et stearinlys ovenpå en vægt og antænder lyset. Efter lidt tid måler
vi tiden i minutter og kalder det for $x$. Vi aflæser også vægtens visning i
gram og kalder det for $y$. Det kunne f.eks. give følgende tabel:
$$
\begin{tabular}{c|c|c|c|c|c|c}
    x &  0 &  5 & 10 & 15 & 20 & 25 \\
    \hline
    y & 30 & 29 & 28 & 27 & 26 & 25
\end{tabular}
$$
Hvis vi afsætter disse punkter i et koordinatsystem, så får vi følgende figur:

[Billede af graf af ovenstående punkter]

Vi ser således, at når tiden går, så bliver vægten mindre.
På figuren er også indtegnet en ret linje, og det er tydeligt, at punkterne
ligger meget tæt på denne linje.  Der er altså også her en lineær sammenhæng
mellem vægtens visning og den tid lyset har brændt.

\section{Intervaller og uligheder}

\end{document}


