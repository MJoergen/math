\documentclass[12pt,oneside,a4paper]{article}

\usepackage[utf8]{inputenc} % Lærer LaTeX at forstå unicode - HUSK at filen skal
% være unicode (UTF-8), standard i Linux, ikke i
% Win.

\usepackage[danish]{babel} % Så der fx står Figur og ikke Figure, Resumé og ikke
% Abstract etc. (god at have).

%\usepackage{graphicx}
\usepackage{amsfonts}
\usepackage{amsthm}        % Theorems
\usepackage{amsmath}
%\usepackage{hyperref}

%\renewcommand{\mid}[1]{{\rm E}\!\left[#1\right]}
\newcommand{\bas}{\begin{eqnarray*}}
\newcommand{\eas}{\end{eqnarray*}}
\newcommand{\be}{\begin{equation}}
\newcommand{\ee}{\end{equation}}
\newcommand{\bea}{\begin{eqnarray}}
\newcommand{\eea}{\end{eqnarray}}

\newtheorem{thm}{Sætning}[section]
\newtheorem{mydef}[thm]{Definition}
\newtheorem{eks}[thm]{Eksempel}

\title{Lineære sammenhænge}
\date{September 2015}

\begin{document}

\maketitle

\section{Eksempler på sammenhænge}
I dette kapitel skal vi arbejde med variabelsammenhænge.

\begin{mydef}
En variabelsammenhæng er en sammenhæng mellem en uafhængig variabel, ofte kaldet $x$, og en afhængig variabel, ofte kaldet $y$.
\end{mydef}

Vi begynder dette kapitel med at se på nogle eksempler.
\section{Lineære sammenhænge}
\begin{mydef}
    En lineær sammenhæng er en sammenhæng givet ved følgende formel:
    $$
    y = a\cdot x + b
    $$
    hvor $a$ og $b$ er tal.
    
    Tallet $a$ kaldes "hældning" og tallet $b$ kaldes "skæring med $y$-aksen".
\end{mydef}
Grafen for en lineær sammenhæng vil altid være en ret linje. Der gælder også
omvendt, at enhver ret linje kan beskrives ved en lineær sammenhæng, dog ikke hvis linjen er lodret.

[Billede af forskellige rette linjer]

Eksempler på lineære sammenhænge er:

\begin{tabular}{ll}
    $\bullet\quad y=2x+1$  & Her er $a=2$ og $b=1$. \\
    $\bullet\quad y=-3x-4$ & Her er $a=-3$ og $b=-4$. \\
    $\bullet\quad y=1-x$   & Her er $a=-1$ og $b=1$. \\
    $\bullet\quad y=3x$    & Her er $a=3$ og $b=0$.
\end{tabular}


\subsection{$y=ax+b$}
\subsection{$y=ax$}
\subsection{Betydning af $a$ og $b$}
\subsection{Grafens udseende}
\section{Løse førstegradsligninger}
\section{To ligninger med to ubekendte}
\section{Linjens ligning (hældning og punkt)}
\section{Bestemmelse af $a$ og $b$}
\section{Lineær regression og vækst}
\subsection{Måling af sands densitet}
[Billede af måleglas med sand ovenpå en vægt]

Vi stiller et måleglas ovenpå en vægt og hælder gradvist sand i glasset. Vi
aflæser sandets rumfang i ${\rm cm}^3$ og kalder det for $x$. Vi aflæser også
vægtens visning i gram og kalder det for $y$. Det kunne f.eks. give følgende
tabel:
$$
\begin{tabular}{c|c|c|c|c|c|c}
    x &   0 &  50 & 100 & 150 & 200 & 250 \\
    \hline
    y & 300 & 400 & 500 & 600 & 700 & 800
\end{tabular}
$$
Hvis vi afsætter disse punkter i et koordinatsystem, så får vi følgende figur:

[Billede af graf af ovenstående punkter]

Vi ser således, at når mængden af sand øges, så stiger vægten. 
På figuren er også indtegnet en ret linje, og det er tydeligt, at punkterne
ligger meget tæt på denne linje.  Der er altså en sammenhæng mellem vægtens
visning og mængden af sandet. Da grafen passer med en ret linje, kalder vi
denne sammenhæng for en {\em lineær sammenhæng}.

\subsection{Et lys der brænder}
[Billede af et brændende stearinlys på en vægt]

Vi stiller et stearinlys ovenpå en vægt og antænder lyset. Efter lidt tid måler
vi tiden i minutter og kalder det for $x$. Vi aflæser også vægtens visning i
gram og kalder det for $y$. Det kunne f.eks. give følgende tabel:
$$
\begin{tabular}{c|c|c|c|c|c|c}
    x &  0 &  5 & 10 & 15 & 20 & 25 \\
    \hline
    y & 30 & 29 & 28 & 27 & 26 & 25
\end{tabular}
$$
Hvis vi afsætter disse punkter i et koordinatsystem, så får vi følgende figur:

[Billede af graf af ovenstående punkter]

Vi ser således, at når tiden går, så bliver vægten mindre.
På figuren er også indtegnet en ret linje, og det er tydeligt, at punkterne
ligger meget tæt på denne linje.  Der er altså også her en lineær sammenhæng
mellem vægtens visning og den tid lyset har brændt.

\subsection{Korrelation}
\section{Intervaller og uligheder}

\end{document}


