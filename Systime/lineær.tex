\documentclass[12pt,oneside,a4paper]{article}

\usepackage[utf8]{inputenc} % Lærer LaTeX at forstå unicode - HUSK at filen skal
% være unicode (UTF-8), standard i Linux, ikke i
% Win.

\usepackage[danish]{babel} % Så der fx står Figur og ikke Figure, Resumé og ikke
% Abstract etc. (god at have).

%\usepackage{graphicx}
\usepackage{amsfonts}
\usepackage{amsthm}        % Theorems
\usepackage{amsmath}
%\usepackage{hyperref}

%\renewcommand{\mid}[1]{{\rm E}\!\left[#1\right]}
\newcommand{\bas}{\begin{eqnarray*}}
\newcommand{\eas}{\end{eqnarray*}}
\newcommand{\be}{\begin{equation}}
\newcommand{\ee}{\end{equation}}
\newcommand{\bea}{\begin{eqnarray}}
\newcommand{\eea}{\end{eqnarray}}

\newtheorem{thm}{Sætning}[section]
\newtheorem{mydef}[thm]{Definition}
\newtheorem{eks}[thm]{Eksempel}

\DeclareMathSymbol{,}{\mathord}{letters}{"3B}

\title{Linearitet}
\date{September 2015}

\begin{document}

\maketitle

\section{Indledning}
I dette kapitel skal vi arbejde med lineære variabelsammenhænge.
Vi begynder med at kigge på nogle eksempler.

\subsection{Eksempel}
Et teleselskab har et simpelt takstsystem:
\begin{itemize}
    \item Abonnement 20 kr. pr. måned, samt 0,50 kr. pr. talt minut.
\end{itemize}
Hvis en abonnent en måned taler 25 minutter, skal han betale:
$$
20 + 0,50\cdot25 = 32,50 \, {\rm kr},
$$
og tales der $x$ minutter en bestemt måned, er prisen $y$ kroner bestemt ved:
$$
y = 20 + 0,50\cdot x
$$
Man siger, at der er en {\em lineær sammenhæng} mellem samtaletiden og prisen.

Hvis vi tegner sammenhængen i et koordinatsystem med antallet af minutter på
$x$-aksen og prisen på $y$-aksen, så vil grafen danne en ret linje med hældning
$0,50$ og som skærer $y$-aksen i tallet $20$, se figur 3.5.

Hældningen $0,50$ angiver prisstigningen for hvert ekstra minut, der tales.
Således er prisen for 20 minutters samtale 30 kr., og for 21 minutters
samtale 30,50 kr.

\subsection{Eksempel}
En prøvekørsel af en bil begynder ved 5-km stenen uden for en by. Bilen kører
en længere strækning med 90 km/t. Efter 8 minutters kørsel befinder bilen sig
ved 17-km-stenen (hvorfor?), og efter $x$ minutter er den ved $y$-km-stenen,
hvor
$$
y = 5 + 1,5\cdot x
$$
Her er der tale om en lineær sammenhæng mellem tiden i minutter og tallet på
kilometerstenen.

\section{Lineære sammenhænge}
\begin{mydef}
    En variabelsammenhæng er en sammenhæng mellem en uafhængig variabel, ofte
    kaldet $x$, og en afhængig variabel, ofte kaldet $y$.
\end{mydef}

\begin{mydef}
    En lineær sammenhæng er en variabelsammenhæng givet ved følgende ligning:
    $$
    y = a\cdot x + b
    $$
    hvor $a$ og $b$ er tal.
    
    Tallet $a$ kaldes "hældning" og tallet $b$ kaldes "skæring med $y$-aksen".
\end{mydef}

\begin{thm}
    Grafen for en lineær sammenhæng er en ret linje.
    Linjens placering i et koordinatsystem er bestemt af tallen $a$ og $b$.
    Således er $a$ linjens hældning, dvs $y$ vokser med $a$ hver gang $x$ vokser med 1.
    Linjen skærer $y$-aksen i punktet $(0, b)$.
\end{thm}

Eksempler på lineære sammenhænge er:

\begin{tabular}{ll}
    $\bullet\quad y=2x+1$  & Her er $a=2$ og $b=1$. \\
    $\bullet\quad y=-3x-4$ & Her er $a=-3$ og $b=-4$. \\
    $\bullet\quad y=1-x$   & Her er $a=-1$ og $b=1$. \\
    $\bullet\quad y=3x$    & Her er $a=3$ og $b=0$.
\end{tabular}


[Billede af forskellige rette linjer]

\subsection{$y=ax+b$}
\subsection{$y=ax$}
\subsection{Betydning af $a$ og $b$}
\subsection{Grafens udseende}
\section{Løse førstegradsligninger}
\section{To ligninger med to ubekendte}
\section{Linjens ligning (hældning og punkt)}
\section{Bestemmelse af $a$ og $b$}
\section{Lineær regression og vækst}
\subsection{Måling af sands densitet}
[Billede af måleglas med sand ovenpå en vægt]

Vi stiller et måleglas ovenpå en vægt og hælder gradvist sand i glasset. Vi
aflæser sandets rumfang i ${\rm cm}^3$ og kalder det for $x$. Vi aflæser også
vægtens visning i gram og kalder det for $y$. Det kunne f.eks. give følgende
tabel:
$$
\begin{tabular}{c|c|c|c|c|c|c}
    x &   0 &  50 & 100 & 150 & 200 & 250 \\
    \hline
    y & 300 & 400 & 500 & 600 & 700 & 800
\end{tabular}
$$
Hvis vi afsætter disse punkter i et koordinatsystem, så får vi følgende figur:

[Billede af graf af ovenstående punkter]

Vi ser således, at når mængden af sand øges, så stiger vægten. 
På figuren er også indtegnet en ret linje, og det er tydeligt, at punkterne
ligger meget tæt på denne linje.  Der er altså en sammenhæng mellem vægtens
visning og mængden af sandet. Da grafen passer med en ret linje, kalder vi
denne sammenhæng for en {\em lineær sammenhæng}.

\subsection{Et lys der brænder}
[Billede af et brændende stearinlys på en vægt]

Vi stiller et stearinlys ovenpå en vægt og antænder lyset. Efter lidt tid måler
vi tiden i minutter og kalder det for $x$. Vi aflæser også vægtens visning i
gram og kalder det for $y$. Det kunne f.eks. give følgende tabel:
$$
\begin{tabular}{c|c|c|c|c|c|c}
    x &  0 &  5 & 10 & 15 & 20 & 25 \\
    \hline
    y & 30 & 29 & 28 & 27 & 26 & 25
\end{tabular}
$$
Hvis vi afsætter disse punkter i et koordinatsystem, så får vi følgende figur:

[Billede af graf af ovenstående punkter]

Vi ser således, at når tiden går, så bliver vægten mindre.
På figuren er også indtegnet en ret linje, og det er tydeligt, at punkterne
ligger meget tæt på denne linje.  Der er altså også her en lineær sammenhæng
mellem vægtens visning og den tid lyset har brændt.

\subsection{Korrelation}
\section{Intervaller og uligheder}

\end{document}


