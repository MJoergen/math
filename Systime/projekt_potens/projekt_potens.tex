\documentclass[12pt,oneside,a4paper]{article}

\usepackage[utf8]{inputenc} % Lærer LaTeX at forstå unicode - HUSK at filen skal
% være unicode (UTF-8), standard i Linux, ikke i
% Win.

\usepackage[danish]{babel} % Så der fx står Figur og ikke Figure, Resumé og ikke
% Abstract etc. (god at have).

\usepackage{graphicx}
\usepackage{amsfonts}
\usepackage{amsthm}        % Theorems
\usepackage{amsmath}
\usepackage{enumitem}
%\usepackage{hyperref}

%\renewcommand{\mid}[1]{{\rm E}\!\left[#1\right]}
\newcommand{\bas}{\begin{eqnarray*}}
\newcommand{\eas}{\end{eqnarray*}}
\newcommand{\be}{\begin{equation}}
\newcommand{\ee}{\end{equation}}
\newcommand{\bea}{\begin{eqnarray}}
\newcommand{\eea}{\end{eqnarray}}

\theoremstyle{plain}
\newtheorem*{thm}{Sætning}
\newtheorem*{mydef}{Definition}
\newtheorem*{eks}{Eksempel}

\DeclareMathSymbol{,}{\mathord}{letters}{"3B}

\title{Projektopgaver om potensfunktioner}
\date{\vspace{-5ex}}

\begin{document}

\maketitle

\section*{Opgave 1 -- Keplers 3. lov for planeter}
Planeternes bevægelse rundt om Solen har været genstand for spekulation i
årtusinder.  Da den tyske matematiker Johannes Kepler i år 1600 blev ansat hos
den danske astronom Tycho Brahe, begyndte et bemærkelsesværdigt samarbejde,
der ledte frem til en dybere forståelse af planetbevægelserne. På baggund af
Brahes detaljerede observationer lykkedes det Kepler i 1618 at gennemskue
systemet i planeternes bevægelse, og formulerede det som tre love.

I denne opgave skal vi undersøge Keplers 3. lov, som omhandler sammenhængen
mellem en planets omløbstid om solen, og planetens afstand til solen.

Den følgende tabel viser sammenhørende værdier for planeternes afstand $x$ til
solen (målt i antal millioner km) og omløbstid $y$ omkring solen (målt i år).

\vspace{2ex}

\begin{center}
\begin{tabular}{|r|r|r|}
    \hline
            & $x$                 & $y$ \\
    Planet  & afstand i mio km    & omløbstid i år   \\
    \hline 
    Merkur  &  $  57,9$ &   $0,241$ \\
    Venus   &  $ 108,2$ &   $0,615$ \\
    Jorden  &  $ 149,6$ &   $1,000$ \\
    Mars    &  $ 227,9$ &   $1,881$ \\
    Jupiter &  $ 778,4$ &  $11,863$ \\
    Saturn  &  $1426,7$ &  $29,448$ \\
    Uranus  &  $2871,0$ &  $84,017$ \\
    Neptun  &  $4498,3$ & $164,791$ \\
    \hline 
\end{tabular}
\end{center}

\begin{enumerate}[label=(\alph*)]
    \item Tegn en figur, hvor du viser planeternes omløbstid om Solen som
        funktion af deres afstand til solen.
    \item Foretag en lineær regression og bestem forskriften på formen
        $y=ax+b$. Diskutér modellens gyldighed.
\end{enumerate}

I stedet for en simpel lineær model skal du nu omregne tabellens data ved hjælp af 
10-tals logaritmen.
\begin{enumerate}[label=(\alph*) ,resume]
    \item Udfyld en tabel, hvor du beregner 10-tals logaritmen til både afstand
        og omløbstid, se nedenstående.
\end{enumerate}

\begin{center}
\begin{tabular}{|r|r|r|}
    \hline
    Planet  & $\log(x)$   & $\log(y)$ \\
    \hline 
    Merkur  &  $1,76$ &   $-0,618$ \\
    \ldots  & \ldots    & \ldots  \\
    \hline 
\end{tabular}
\end{center}

\begin{enumerate}[label=(\alph*) ,resume]
    \item Tegn en figur, hvor du viser $\log(y)$ som funktion af $\log(x)$.
    \item Foretag en lineær regression på denne figur og bestem forskriften på formen
        $\log(y) = a\cdot\log(x)+b$ og diskutér modellens gyldighed. \label{log}
\end{enumerate}

Keplers lov ses ofte formuleret på følgende måde:
$$
y = k \cdot x^{3/2}
$$
\begin{enumerate}[label=(\alph*) ,resume]
    \item Vis, at svaret i spørgsmål~\ref{log} kan omskrives til ovenstående formel.
\end{enumerate}


Udover de kendte planeter, findes der også en række dværgplaneter, der
ligesom planeterne bevæger sig rundt om Solen. En af disse dværgplaneter er
Ceres, som befinder sig i afstanden ca. 414 millioner km fra Solen.

\begin{enumerate}[label=(\alph*) ,resume]
    \item Beregn omløbstiden i år for dværgplaneten Ceres.
\end{enumerate}

\section*{Opgave 2 -- Newtons tyngdelov}
Mange sammenhænge i fysik kan beskrives ved potensfunktioner. Newton
postulerede i 1687, at tiltrækningen mellem to masser aftager med kvadratet på
afstanden til centrum.  Denne påstand bliver verificeret i denne opgave.

Tiltrækningen ved Jordens overflade kendes som tyngdeaccelerationen $g=9,82\,
{\rm m}/{\rm s}^2$.  Ligeledes er Jordens radius en kendt værdi, nemlig
$x=6730$ km.

Vi skal nu først bestemme en værdi for Jordens tiltrækning på månen.
Denne kan beregnes som
\be
a = \frac{v^2}{r}\,,
\label{acc}
\ee
hvor $v$ er månens fart målt i meter pr. sekund og $r$ er månens afstand til
Jordens centrum målt i meter.  Tiltrækningen måles i ${\rm m}/{\rm s}^2$.

\begin{enumerate}[label=(\alph*)]
    \item Vis, at ovenstående formel for tiltrækningen kan omskrives til 
        $$
        a = \left(\frac{2\pi}{T}\right)^2 \cdot r\,,
        $$
        hvor $T$ er Månens omløbstid om Jorden målt i sekunder, idet månens
        hastighed $v$ antages at være lig med omkredsen af månens bane
        divideret med omløbstiden. \label{a}
\end{enumerate}

Månens omløbstid er 27,32 dage og dens afstand til Jordens centrum er 384400 km.

\begin{enumerate}[label=(\alph*) ,resume]
    \item Omregn Månens omløbstid til sekunder og månens afstand til Jordens
        centrum til meter. Vis derefter, at Jordens tiltrækning på månen har
        værdien $0,00272\, {\rm m}/{\rm s}^2$.
\end{enumerate}

Dette giver følgende tabel over sammenhængen mellem afstand til Jordens centrum
og størrelsen af tiltrækningen:

\begin{center}
\begin{tabular}{|r|r|r|}
    \hline
          & $x$                         & $y$ \\
    Sted  & afstand til Jordens centrum & tiltrækning   \\
          & i km                        & i ${\rm m}/{\rm s}^2$  \\
    \hline 
    Jordens overflade & $6730$  & $9,82$ \\
    \hline
    ved Månen         & $384400$ & $0,00272$ \\
    \hline 
\end{tabular}
\end{center}

Newton kommer nu med den hypotese, at sammenhængen mellem afstand og
tiltrækning kan beskrives ved en potensfunktion.

\begin{enumerate}[label=(\alph*) ,resume]
    \item Beregn værdierne af $a$ og $b$ i den potensfunktion $y=b\cdot x^a$,
        som går gennem de to punkter angivet i tabellen. \label{p1}
\end{enumerate}

Newton formulerede sin lov som
$$
y = \frac{b}{x^2}\,.
$$

\begin{enumerate}[label=(\alph*) ,resume]
    \item Vis, at svaret i~\ref{p1} kan omskrives til ovenstående formel.
\end{enumerate}


\section*{Opgave 3 -- Keplers 3. lov benyttet for bevægelse om Jorden}
Keplers 3. lov gælder også for bevægelse om Jorden. Den siger, at
$$
T^2 = k\cdot r^3,
$$
hvor $T$ er omløbstiden om Jorden i timer, $r$ er afstanden til Jordens centrum
i tusinde km, og $k$ er en konstant. Denne formel gælder både for satellitter
og for Jordens måne.

Månen har en omløbstid på $27,32$ dage og en afstand til Jordens centrum på
$384,4$ tusinde km.
\begin{enumerate}[label=(\alph*)]
    \item Omregn månens omløbstid på $27,32$ dage til timer.
    \item Beregn værdien af $k$.
\end{enumerate}



Den Internationale Rumstation befinder sig i en højde af ca. $386$ km over
Jordens overflade. Jordens radius er ca. $6730$ km.
\begin{enumerate}[label=(\alph*), resume]
    \item Beregn afstanden fra den Internationale Rumstation til Jordens
        centrum målt i tusinde km.
    \item Brug ovenstående ligning til at beregne rumstationens omløbstid.
    \item Hvor mange gange kommer rumstationen rundt om Jorden på et døgn?
\end{enumerate}


En geostationær satellit er en satellit, som hele tiden befinder sig over det
samme punkt på Ækvator.  Satellittens omløbstid er derfor lig med
Jordens egen rotationstid.
\begin{enumerate}[label=(\alph*), resume]
    \item Gør rede for, at omløbstiden for en geostationær satellit er 24 timer.
    \item Brug ovenstående ligning til at beregne afstanden fra en geostationær
        satellit til Jordens centrum.
    \item Hvor højt over Jordens overflade befinder en geostationær satellit sig?
\end{enumerate}

\end{document}

