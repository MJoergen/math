\documentclass[12pt,oneside,a4paper]{article}

\usepackage[utf8]{inputenc} % Lærer LaTeX at forstå unicode - HUSK at filen skal
% være unicode (UTF-8), standard i Linux, ikke i
% Win.

\usepackage[danish]{babel} % Så der fx står Figur og ikke Figure, Resumé og ikke
% Abstract etc. (god at have).

\usepackage{graphicx}
\usepackage{amsfonts}
\usepackage{amsthm}        % Theorems
\usepackage{amsmath}
\usepackage{float}         % Så kan man bedre styre, hvor figurerne havner henne
                           % vha [H].
%\usepackage{hyperref}
\usepackage{tcolorbox}

%\renewcommand{\mid}[1]{{\rm E}\!\left[#1\right]}
\newcommand{\bas}{\begin{eqnarray*}}
\newcommand{\eas}{\end{eqnarray*}}
\newcommand{\be}{\begin{equation}}
\newcommand{\ee}{\end{equation}}
\newcommand{\bea}{\begin{eqnarray}}
\newcommand{\eea}{\end{eqnarray}}

\newtheorem{thm}{Sætning}[section]
\newtheorem{mydef}[thm]{Definition}
\newtheorem{eks}[thm]{Eksempel}
\newtheorem{bevis}[thm]{Bevis}

\DeclareMathSymbol{,}{\mathord}{letters}{"3B}

\title{Finansiel regning}
\date{\vspace{-5ex}}

\begin{document}

\maketitle

%%%%%%%%%%%%%%%%%%%%%%%%%%%%%%%%%%%%%%%%%%%%%%%%%%%%%%%%%%%%%%%%%

\section{Renteformlen}
\begin{figure}[ht]
    \centering
    \includegraphics[width=10cm]{penge}
\end{figure}

Vi begynder med et eksempel. Vi har en opsparing på 1000 kroner, som vi
indsætter på en opsparingskonto i banken, der giver 5 \% i rente om året. Dvs.
efter 1 år bliver der indsat 5 \% af 1000 kroner på kontoen. Det
udregner vi til $ \frac{5}{100} \cdot 1000 = 50 $ kroner. Så efter 1 år står
der nu 1050 kroner på kontoen.  Efter det andet år bliver der nu indsat 5 \% af
1050 kroner, dvs $ \frac{5}{100} \cdot 1050 = 52,50$ kroner.
Det giver et samlet beløb efter 2 år på 1102,50 kroner. Læg mærke til, at
rentebeløbet det andet år (52,50 kroner) er større end rentebeløbet det første
år (50 kroner).

Hvis vi gerne vil beregne, hvad saldoen bliver efter 8 år, så kan vi fortsætte
ovenstående beregninger. Det er praktisk at opstille beregningerne i et skema
som vist her:
\[
\begin{array}{|r|c|l|}
    \hline
    \mbox{efter år} & \mbox{saldo} & \mbox{rentebeløb} \\
    \hline
0 & 1000 & 0,05\cdot 1000 = 50 \\
    \hline
1 & 1050 & 0,05\cdot 1050 = 52,50 \\
    \hline
2 & 1102,50 & 0,05\cdot 1102,50 = 55,13 \\
    \hline
3 & 1157,63 & 0,05\cdot 1157,63 = 57,88 \\
    \hline
4 & 1215,51 & 0,05\cdot 1215,51 = 60,78 \\
    \hline
5 & 1276,29 & 0,05\cdot 1276,29 = 63,81 \\
    \hline
6 & 1340,10 & 0,05\cdot 1340,10 = 67,01 \\
    \hline
7 & 1407,11 & 0,05\cdot 1407,11 = 70,36 \\
    \hline
8 & 1477,47 & \\
\hline
\end{array}
\]

Udviklingen i saldoen er vist i nedenstående figur, hvor saldo og rente er vist med hver sin farve.
\begin{figure}[H]
    \centering
    \includegraphics{fin-1}
\end{figure}

Vi har således beregnet, at efter 8 år, så vil der stå 1477,47 kroner på saldoen.  Dette
er dog en temmelig besværlig metode, og der er en meget nemmere måde at nå frem
til resultatet 1477,47 kroner, hvilket vi skal se i det følgende.

For det første bemærker vi, at for hvert år der går, bliver saldoen ganget med
1,05. Dette kan vi indse ved følgende udregning: Efter det første år bliver
saldoen udregnet som:
\[
1050 = 1000 + 50 = 1\cdot 1000 + 0,05\cdot 1000 = 1,05 \cdot 1000.
\]

Saldoen efter 2 år kan så udregnes som
\[
1,05\cdot 1050 = 1,05 \cdot (1,05 \cdot 1000) = 1,05^2 \cdot 1000.
\]
Saldoen efter 3 år kan på samme måde udregnes som
\[
1,05\cdot1,05\cdot1,05\cdot 1000 = 1,05^3 \cdot 1000.
\]
Efter 8 år er saldoen
derfor på $1,05^8\cdot 1000$. Hvis vi udregner dette på lommeregner, så giver
det netop ca. 1477 kroner.

Denne metode kan udtrykkes ved følgende formel:
\begin{tcolorbox}
\begin{thm}
    {\em Renteformlen} kan udregne saldoen på en opsparing, som begynder med
    værdien $K_0$ og som efter hver {\em termin} bliver forøget med en fast
    rente $r$. Saldoen efter $n$ terminer er givet ved følgende formel:
    $$
    K_n = K_0 \cdot (1+r)^n
    $$
    hvor $K_0$ er {\em begyndelseskapitalen}, $r$ er {\em vækstraten}, $n$ er antallet af
    terminer, og $K_n$ er {\em slutkapitalen} efter de $n$ terminer.
\end{thm}
\end{tcolorbox}
En termin er typisk 1 år, men kan også være f.eks. en måned. Terminen er bestemt af,
hvor tit der tilskrives et nyt rentebeløb. Vækstraten $r$ er renten udtrykt som
decimaltal. Dvs. hvis renten er 5\%, så er vækstraten $r=\frac{5}{100}=0,05$.

I det følgende gennemgår vi nogle eksempler på anvendelse af renteformlen.
\begin{tcolorbox}
\begin{eks}
    En arv på 20000 kroner indsættes på en opsparingskonto, som
    giver 4\% i rente om året. Hvad vil der stå på kontoen efter 7 år?
\end{eks}
\begin{proof}[Svar]
    Først udregnes vækstraten $r=\frac{4}{100} = 0,04$. Så indsættes i renteformlen:
    $$
    K_7 = 20000 \cdot (1 + 0,04)^7 = 26318,64.
    $$
    Altså efter syv år står der over 26000 kroner på saldoen.
\end{proof}
\end{tcolorbox}

Renteformlen kan også bruges, når man låner penge, hvis man ikke betaler af
(afdrager) på gælden undervejs.
\begin{tcolorbox}
\begin{eks}
    Hvis en kassekredit har et underskud på 17000 kroner, og der
    lægges 1,8 \% i rente hver måned, hvor meget er gælden så på efter 2 år?
\end{eks}
\begin{proof}[Svar]
    Når renten bliver tilskrevet hver måned, skal de 2 år omregnes til
    måneder, dvs. 24 måneder. Dette er antallet af terminer. Vækstraten er $r =
    \frac{1,8}{100} = 0,018$, og renteformlen giver derfor:
    $$
    K_{24} = 17000 \cdot (1 + 0,018)^{24} = 26085,29
    $$
    Den samlede gæld er derfor vokset til over 26000 kroner på de 2 år, dvs.
    der betales over 9000 kr. i rente.
\end{proof}
\end{tcolorbox}

\subsection{Effektiv rente}
Hvis renten ikke tilskrives årligt, så rejser sig spørgsmålet, om der er
forskel på at få $12\%$ i rente om året, og at få $1\%$ i rente om måneden?  Vi
ser på et eksempel med en startkapital på $1000$ kr, og spørger nu, hvilket
beløb der står på kontoen efter 1 år.

En årlig rente på $12\%$ svarer til $n=1$ og $r=0,12$. Det giver en slutkapital
på:
\[
    1000 \cdot 1,12^1 = 1120 {\,\rm kr.}
\]
I situationen med en månedlig rente på $1\%$ har vi derimod $n=12$ og $r=0,01$.
Så giver renteformlen, at slutkapitalen er
\[
    1000\cdot 1,01^{12} = 1126,83 {\,\rm kr.}
\]
Så den månedlige rentetilskrivning giver en samlet stigning på $126,83$ kr, dvs.
$12,68\%$ efter 1 år. Dette kaldes også den {\em effektive rente}.
\begin{tcolorbox}
\begin{thm}
    Vi udregner den årlige effektive rente $r_{\rm eff}$ (som decimaltal) på
    følgende måde:
    \[
        r_{\rm eff} = (1+r)^n-1\,,
    \]
hvor $n$ er antallet af terminer på et år.
\end{thm}
\end{tcolorbox}
\begin{tcolorbox}
\begin{eks}
    En kassekredit tilskrives en rente på $1,2\%$ om måneden. Den årlige
    effektive rente er da
    \[
        r_{\rm eff} = (1+0,012)^{12}-1 = 1,012^{12}-1 = 0,1539\,.
    \]
    Dette svarer altså til $15,39\%$.
\end{eks}
\end{tcolorbox}

\subsection{Gennemsnitlig rente}
Hvis renten ændrer sig fra år til år, så kan man definere den gennemsnitlige
rente. Denne har den egenskab, at hvis renten var konstant lig med den
gennemsnitlige rente, så ville saldoen være uændret.  Vi ser først på et
eksempel. Hvis en startkapital på $1000$ kr. tilskrives $3\%$ det første år,
dernæst $9\%$ det andet år, og til sidst $6\%$ det tredje år, så vil saldoen
efter de tre år være givet ved
\[
    1000\cdot 1,03 \cdot 1,09 \cdot 1,06 = 1190,06\,.
\]
Den årlige gennemsnitlige rente findes da ved at løse ligningen
\[
    1000\cdot (1+r)^3 = 1190,06 \Leftrightarrow 
\]
\[
    (1+r)^3=1,19006 \Leftrightarrow
\]
\[
    1+r = 1,0597 \Leftrightarrow
\]
\[
    r = 5,97\%\,.
\]
Ovenstående beregning kan formuleres i følgende sætning:
\begin{tcolorbox}
\begin{thm}
    Den gennemsnitlige rente knyttet til renterne $r_1$, $r_2$, \ldots, $r_n$
    er givet ved:
    \[
        r_g = \sqrt[n]{(1+r_1)\cdot(1+r_2)\cdots(1+r_n)}-1 \,.
    \]
\end{thm}
\end{tcolorbox}
\begin{tcolorbox}
\begin{eks}
    En bankkonto med variabel rente tilskrives det første år $5\%$, det andet
    år $10\%$, og det tredje år $15\%$ i rente. Den gennemsnitlige årlige rente
    i løbet af de tre år er da givet ved:
    \[
        r_g = \sqrt[3]{1,05\cdot1,10\cdot1,15}-1 = \sqrt[3]{1,32825}-1 = 0,0992\,.
    \]
Det vil sige, at på de tre år er kontoen blevet tilskrevet en gennemsnitlig
    årlig rente på $9,92\%$.
\end{eks}
\end{tcolorbox}


\section{Annuitetsopsparing}
En \emph{annuitetsopsparing} er kendetegnet ved, at man hver termin (f.eks. hver
måned) foretager en fast indbetaling (også kaldet \emph{ydelse}). Efter hver
termin tilskrives rente, som vi har set på tidligere, og ligesom før er
rentefoden også her konstant. Et eksempel på en annuitetsopsparing er en
børneopsparing, hvor forældrene indbetaler et fast beløb hver måned, indtil
barnet bliver 18 år. Til en børneopsparing hører der typisk en høj rentefod,
fordi opsparingen ikke kan hæves før barnet fylder 18 år.

\begin{tcolorbox}
\subsection{Eksempel}
I slutningen af hvert år sætter vi 1000kr. ind på en konto, og samtidig
tilskrives en rente på 5 $\%$ hvert år. Renten beregnes ud fra
opsparingen inden indbetalingen.  Det giver følgende udvikling:
\\

\begin{tabular}{|l|l|l|}
    \hline
    \textbf{Antal} & \textbf{Rente} & \textbf{Opsparing} \\
    \textbf{måneder} &  & \\
    \hline
    1 & 0 & 0 + 0 + 100 = 100 \\
    \hline
    2 & $100\cdot 0,05 = 5$ & 100 + 5 + 100 = 205 \\
    \hline
    3 & $205\cdot 0,05 = 10$ & 205 + 10 + 100 = 315 \\
    \hline
    4 & $315\cdot 0,05 = 16$ & 315 + 16 + 100 = 431 \\
    \hline
    5 & $431\cdot 0,05 = 22$ & 431 + 22 + 100 = 553 \\
    \hline
    6 & $553\cdot 0,05 = 28$ & 553 + 27 + 100 = 680 \\
    \hline
    7 & $680\cdot 0,05 = 34$ & 680 + 34 + 100 = 814 \\
    \hline
    8 & $814\cdot 0,05 = 41$ & 814 + 41 + 100 = 955 \\
    \hline
\end{tabular}
\\

Altså ser vi, at efter 8 år er der 955 kr. på saldoen. De 800 kr. er den
samlede indbetaling, og de resterende 155 kr. er den samlede rente.
Denne udvikling er vist i den følgende figur, hvor saldo, rente og indbetaling er vist med hver sin farve.
\begin{figure}[H]
    \centering
    \includegraphics{fin-2}
\end{figure}
\end{tcolorbox}

Vi ser nu på en alternativ måde at foretage denne beregning. Det vil lede os
til en generel formel. Vi betragter hver indbetaling for sig, og bruger
renteformlen til at beregne den samlede rente til denne ene indbetaling.

Den første indbetaling foregik efter 1 år, og ved udgangen af de 8 år
er der ialt blevet tilskrevet renter 7 gange.  Det giver en samlet værdi på:
\[
    100\cdot 1,05^7 = 141 {\,\,\rm kr.}
\]
Denne udregning gentager vi for alle de øvrige indbetalinger, og lægger alle
værdierne sammen.  Dette er vist i den følgende tabel:
\\

\begin{tabular}{|l|l|l|}
    \hline
    \textbf{Antal måneder} & \textbf{Indbetaling} & \textbf{Værdi efter 6 måneder} \\
    \hline
    1 & 100 & $100 \cdot 1,05^7 = 141$ \\
    \hline
    1 & 100 & $100 \cdot 1,05^6 = 134$ \\
    \hline
    1 & 100 & $100 \cdot 1,05^5 = 128$ \\
    \hline
    2 & 100 & $100 \cdot 1,05^4 = 122$ \\
    \hline
    3 & 100 & $100 \cdot 1,05^3 = 116$ \\
    \hline
    4 & 100 & $100 \cdot 1,05^2 = 110$ \\
    \hline
    5 & 100 & $100 \cdot 1,05 = 105$ \\
    \hline
    6 & 100 & $100 = 100$ \\
    \hline
    \textbf{Samlet} & \textbf{800} & \textbf{955} \\
    \hline
\end{tabular}
\\

Udregningen i den ovenstående tabel kan udtrykkes på følgende måde:
\[
A_n = y\cdot(1+r)^{n-1} + \ldots + y\cdot(1+r) + y \,,
\]
hvor $A_n$ er saldoen efter i alt $n$ terminer,  $y$ er den faste (månedlige)
ydelse, $r$ er rentefoden, og $n$ er det samlede antal terminer.

I praksis benytter man følgende sætning til at udregne saldoen:
\begin{tcolorbox}
\begin{thm}
Saldoen $A_n$ efter $n$ terminer, hvor der til hver termin indbetales en fast ydelse $y$,
samt tilskrives rente med rentefoden $r$, er givet ved følgende formel:
\[
A_n = y\cdot\frac{(1+r)^n-1}{r}\,.
\]
\end{thm}
\end{tcolorbox}

Med denne formel kan vi beregne ovenstående saldo efter 6 terminer:
\[
A_8 = 100\cdot\frac{(1+0,05)^8-1}{0,05} = 955 \,.
\]

\begin{tcolorbox}
\begin{proof}
Beviset for ovenstående sætning tager udgangspunkt i formlen fra før:
\[
A_n = y\cdot(1+r)^{n-1} + \ldots + y\cdot(1+r) + y \,.
\]
Vi søger nu at reducere dette udtryk. Først indfører vi fremskrivningsfaktoren $a=1+r$:
\[
A_n = y\cdot a^{n-1} + \ldots + y\cdot a + y \,.
\]
Så sætter vi $y$ uden for en parentes:
\[
A_n = y\cdot\big(a^{n-1} + \ldots + a + 1\big) \,.
\]
Nu bruger vi et smart trick: Vi ganger og dividerer med $a-1$:
\[
A_n = y\cdot\frac{\big(a^{n-1} + \ldots + a + 1\big)\cdot(a-1)}{a-1} \,.
\]
I tælleren ganger vi parenteserne sammen:
\[
A_n = y\cdot\frac{a^{n} + \ldots + a^2 + a - a^{n-1} - \ldots - a - 1}{a-1} \,.
\]
Til sidst reduceres tælleren:
\[
A_n = y\cdot\frac{a^{n} - 1}{a-1} \,.
\]
Vi indsætter nu definitionen på fremskrivningsfaktoren $a=1+r$:
\[
A_n = y\cdot\frac{(1+r)^{n} - 1}{r} \,.
\]
Og vi har vist det ønskede.
\end{proof}
\end{tcolorbox}

\begin{tcolorbox}
\subsection{Eksempel}
Vi ser på endnu et eksempel. Der indsættes 500 kr. hver måned på en højrente konto, som giver
0,4 $\%$ om måneden. Hvad bliver saldoen efter 4 år?

Vi udregner $n=4\cdot12 = 48$ og dernæst:
\[
A_{48} = 500\cdot\frac{1,004^{48}-1}{0,004} = 26400,82 \,.
\]
Af det samlede beløb udgør indbetalingerne $500\cdot 48 = 24000$ kr., og rentetilskrivningerne
udgør resten, dvs. 2400,82 kr. Så en rente på 0,4 $\%$ om måneden udgør ca. 10 $\%$ efter 4 år.

\end{tcolorbox}

\section{Annuitetslån}
Et \emph{annuitetslån} er (ligesom annuitetsopsparing) kendetegnet ved, at man
hver termin (f.eks. hver måned) foretager en fast indbetaling (også kaldet
\emph{ydelse}). Efter hver termin tilskrives rente, som vi har set på
tidligere, og ligesom før er rentefoden også her konstant. Dog er der her tale
om en gæld, således at den tilskrevne rente øger gælden, mens indbetalingerne
mindsker gælden.  De mest almindelige lån er af denne type, f.eks. forbrugslån,
billån, samt kreditforeningslån til huskøb.

Vi ser på et eksempel først:
\begin{tcolorbox}
\subsection{Eksempel}
En person låner 14000 kr til en rentefod af $1\%$ om måneden, og betaler en fast
ydelse på 400 kr om måneden.

I slutningen af hver måned øges gælden med renten, men nedskrives derefter med
ydelsen. Renten beregnes ud fra gælden inden indbetalingen.  Det giver
følgende udvikling:
\\

\begin{tabular}{|l|l|l|}
    \hline
    \textbf{Antal} & \textbf{Rente} & \textbf{Gæld} \\
    \textbf{måneder} &  & \\
    \hline
    1 & $14000\cdot 0,01 = 140$ & 14000 + 140 - 400 = 13740 \\
    \hline
    2 & $13740\cdot 0,01 = 137$ & 13740 + 137 - 400 = 13477 \\
    \hline
    3 & $13477\cdot 0,01 = 135$ & 13477 + 135 - 400 = 13212 \\
    \hline
    4 & $13212\cdot 0,01 = 132$ & 13212 + 132 - 400 = 12944 \\
    \hline
    5 & $12944\cdot 0,01 = 129$ & 12944 + 129 - 400 = 12673 \\
    \hline
    6 & $12673\cdot 0,01 = 127$ & 12673 + 127 - 400 = 12400 \\
    \hline
\end{tabular}
\\

Vi ser her, at den månedlige rente falder efterhånde som gælden falder. Det
beløb, som gælden falder med hver måned, kaldes \emph{afdrag}. Der gælder
altså følgende sammenhæng:
\[
\mbox{rente + afdrag = ydelse}
\]
I begyndelsen af et annuitetslån vil renten være høj og afdragene vil være små.
Gælden vil derfor falde meget langsomt i begyndelsen af lånets løbetid.
Gælden vil først være faldet væsentligt, når der er betalt en hel del
afdrag på lånet.  Efterhånden som gælden mindskes, så vil renten blive
mindre, og afdragene blive større. Denne variation af rente og afdrag 
hen over lånets løbetid er vist i følgende figur.

BILLEDE

\end{tcolorbox}

Generelt har vi følgende sammenhæng:
\begin{tcolorbox}
\begin{thm}
    Den månedlige ydelse $y$ kan beregens ud fra lånets størrelse 
  (dvs gældens værdi i begyndelsen) $G$, rentefoden $r$ og antallet $n$ terminer
\[
    y = G\cdot\frac{r}{1-(1+r)^{-n}}\,.
\]
\end{thm}
\end{tcolorbox}

\begin{tcolorbox}
\subsection{Eksempel}
Vi vil låne 5000 kr til en rente på $1 \%$ om måneden, og vil betale lånet tilbage over 
4 år (dvs. 48 terminer).
Vi skal dermed betale en månedlig ydelse givet ved
\[
    y = 5000\cdot\frac{0,01}{1-(1+0,01)^{-48}} = 132 \,\,\mbox {kr} \,. 
\]
I løbet af hele lånets løbetid har vi i alt betalt $132\cdot 48 = 6336$ kr. Af
disse udgør renterne i alt $6336-5000 = 1336$ kr.
\end{tcolorbox}

Andre gange er man interesseret i at vide, hvor lang tid et lån skal løbe over, givet
en maksimal ydelse. Dette løser vi ved at isolere $n$ i formlen, hvilket giver:
\[
    n = -\frac{\log\big(1-\frac{G\cdot r}{y}\big)}{\log(1+r)} \,.
\]

\begin{tcolorbox}
\subsection{Eksempel}
Vi ønsker at låne penge til en bil. Det samlede lån bliver på 120000 kr, og
rentefoden er $0,8\%$ om måneden. Vi kan maksimalt 1800 kr om måneden, og
vil gerne vide, hvor lang tid lånet varer.
Vi indsætter i formlen:
\[
n = -\frac{\log\big(1-\frac{120000\cdot0,008}{1000}\big)}{\log(1+0,008)} = 96 \,,
\]
hvilket svarer til 8 år.
I den periode er der i alt indbetalt $96\cdot 1800 = 172800$ kr, og det betyder,
at renterne i alt udgør 172800-120000 = 52800 kr.
\end{tcolorbox}



\end{document}

