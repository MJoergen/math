\documentclass[12pt,oneside,a4paper]{article}

\usepackage[utf8]{inputenc} % Lærer LaTeX at forstå unicode - HUSK at filen skal
% være unicode (UTF-8), standard i Linux, ikke i
% Win.

\usepackage[danish]{babel} % Så der fx står Figur og ikke Figure, Resumé og ikke
% Abstract etc. (god at have).

\usepackage{graphicx}
\usepackage{amsfonts}
\usepackage{amsthm}        % Theorems
\usepackage{amsmath}
\usepackage{float}         % Så kan man bedre styre, hvor figurerne havner henne
                           % vha [H].
%\usepackage{hyperref}
\usepackage{tcolorbox}

%\renewcommand{\mid}[1]{{\rm E}\!\left[#1\right]}
\newcommand{\bas}{\begin{eqnarray*}}
\newcommand{\eas}{\end{eqnarray*}}
\newcommand{\be}{\begin{equation}}
\newcommand{\ee}{\end{equation}}
\newcommand{\bea}{\begin{eqnarray}}
\newcommand{\eea}{\end{eqnarray}}

\newtheorem{thm}{Sætning}[section]
\newtheorem{mydef}[thm]{Definition}
\newtheorem{eks}[thm]{Eksempel}
\newtheorem{bevis}[thm]{Bevis}

\DeclareMathSymbol{,}{\mathord}{letters}{"3B}

\title{Finansiel regning}
\date{\vspace{-5ex}}

\begin{document}

\maketitle

%%%%%%%%%%%%%%%%%%%%%%%%%%%%%%%%%%%%%%%%%%%%%%%%%%%%%%%%%%%%%%%%%

\section{Renteformlen}
\begin{figure}[ht]
    \centering
    \includegraphics[width=10cm]{penge}
\end{figure}

Vi begynder med et eksempel. Vi indsætter en opsparing på 2000 kroner
på en opsparingskonto i banken, der giver en \emph{rentefod} på 5 \%
om året. Efter 1 år bliver der altså tilskrevet 5 \% af 2000 kroner på kontoen. Det
udregner vi til $ \frac{5}{100} \cdot 2000 = 100$ kroner. Efter 1 år står
der nu 2100 kroner på kontoen.  Efter det andet år bliver der tilskrevet 5 \% af
2100 kroner, dvs $ \frac{5}{100} \cdot 2100 = 105$ kroner.  Det giver et samlet
beløb efter 2 år på 2205 kroner. Læg mærke til, at renten det andet år
(105 kroner) er større end renten det første år (100 kroner).

Hvis vi vil beregne, hvad saldoen bliver efter 10 år, så kan vi fortsætte
ovenstående beregninger. Det er praktisk at opstille beregningerne i et skema
som vist her:
\[
\begin{array}{|c|c|c|}
    \hline
    \mbox{\textbf{Antal år}} & \mbox{\textbf{Saldo (i hele kr.)}} &
        \mbox{\textbf{Rente (i hele kr.)}} \\
    \hline
0 & 2000 & 0,05\cdot 2000 = 100 \\
    \hline
1 & 2000 + 100 = 2100 & 0,05\cdot 2100 = 105 \\
    \hline
2 & 2100 + 105 = 2205 & 0,05\cdot 2205 = 110 \\
    \hline
3 & 2205 + 110 = 2315 & 0,05\cdot 2315 = 116 \\
    \hline
4 & 2315 + 116 = 2431 & 0,05\cdot 2431 = 122 \\
    \hline
5 & 2431 + 122 = 2553 & 0,05\cdot 2553 = 128 \\
    \hline
6 & 2553 + 128 = 2681 & 0,05\cdot 2681 = 134 \\
    \hline
7 & 2681 + 134 = 2815 & 0,05\cdot 2815 = 141 \\
    \hline
8 & 2815 + 141 = 2956 & 0,05\cdot 2956 = 148 \\
    \hline
9 & 2956 + 148 = 3104 & 0,05\cdot 3104 = 155 \\
    \hline
10 & 3104 + 155 = 3259 & \\
\hline
\end{array}
\]

Udviklingen i saldoen er vist på nedenstående figur, hvor saldo og rente er vist
med hver sin farve.
\begin{figure}[H]
    \centering
    \includegraphics{fin-1}
\end{figure}

Vi har således beregnet, at der efter 10 år vil stå 3259 kroner på saldoen.  Dette
er dog en temmelig besværlig metode, og der er en meget nemmere måde at nå frem
til resultatet, hvilket vi skal se i det følgende.

For det første bemærker vi, at for hvert år der går, bliver saldoen ganget med
1,05. Dette kan vi indse ved følgende udregning: Efter det første år bliver
saldoen udregnet som:
\[
2100 = 2000 + 100 = 1\cdot 2000 + 0,05\cdot 2000 = 1,05 \cdot 2000.
\]

Saldoen efter 2 år kan så udregnes som
\[
1,05\cdot 2100 = 1,05 \cdot (1,05 \cdot 2000) = 1,05^2 \cdot 2000.
\]
Saldoen efter 3 år kan på samme måde udregnes som
\[
1,05\cdot1,05\cdot1,05\cdot 2000 = 1,05^3 \cdot 2000.
\]
Efter 10 år er saldoen derfor på $1,05^{10}\cdot 2000$. Hvis vi udregner dette
giver det $3257,79$ kroner.
Forskellen skyldes, at vi i tabellen har afrundet til helt antal kr.

Denne metode kan udtrykkes ved følgende formel:
\begin{tcolorbox}
\begin{thm}
{\em Renteformlen} kan udregne saldoen på en opsparing, som begynder med
værdien $K_0$ og som efter hver {\em termin} bliver forøget med en rente
givet ved en fast rentefod $r$. Saldoen efter $n$ terminer er givet ved
følgende formel:
$$
K_n = K_0 \cdot (1+r)^n \;,
$$
hvor $K_0$ er {\em begyndelseskapitalen}, $r$ er rentefoden, $n$ er antallet af
terminer, og $K_n$ er {\em slutkapitalen} (saldoen) efter de $n$ terminer.
\end{thm}
\end{tcolorbox}
En termin er typisk 1 år, men kan også være f.eks. en måned. Terminen er
intervallet mellem to rentetilskrivninger.  Rentefoden $r$ skal skrives som
decimaltal.  Hvis rentefoden er 5\%, så er $r=\frac{5}{100}=0,05$.

I det følgende gennemgår vi nogle eksempler på anvendelse af renteformlen.
\begin{tcolorbox}
\begin{eks}
En arv på $20000$ kroner indsættes på en opsparingskonto, som tilskrives en
årlig rentefod på $4$\%. Hvad vil der stå på kontoen efter $7$ år?
\end{eks}
\begin{proof}
Først skrives rentefoden som decimaltal: $r=\frac{4}{100} = 0,04$. Så
indsættes i renteformlen:
\[ 
K_7 = 20000 \cdot (1 + 0,04)^7 = 26319\,.
\]
Altså efter syv år står der over 26000 kroner på saldoen.
Kontoens værdi er steget med $26319 - 20000 = 6319$ kr, som er renterne.
\end{proof}
\end{tcolorbox}

Renteformlen kan også bruges, når man låner penge, hvis man ikke betaler af
(afdrager) på gælden undervejs.
\begin{tcolorbox}
\begin{eks}
Hvis en kassekredit har et underskud på 17000 kroner, og rentefoden er 1,8 \%
pr. måned, hvor meget er gælden så efter 2 år?
\end{eks}
\begin{proof}
Når renten bliver tilskrevet hver måned, skal de 2 år omregnes til
måneder, dvs. 24 måneder. Dette er antallet af terminer. Rentefoden er $r =
\frac{1,8}{100} = 0,018$, og renteformlen giver derfor:
\[
K_{24} = 17000 \cdot (1 + 0,018)^{24} = 26085 \,.
\]
Den samlede gæld er derfor vokset til 26085 kr. på de 2 år, dvs.  der betales
over 9000 kr. i rente!
\end{proof}
\end{tcolorbox}

\subsection{Effektiv rente}
Hvis renten ikke tilskrives årligt, kan man stille spørgsmålet, om der er
forskel på at få $12\%$ i rente om året, og at få $1\%$ i rente om måneden.  Vi
ser på et eksempel med en begyndelseskapital på $1000$ kr, og spørger nu, hvilket
beløb der står på kontoen efter 1 år.

En årlig rente på $12\%$ svarer til $n=1$ og $r=0,12$. Det giver en slutkapital
på:
\[
    1000 \cdot 1,12^1 = 1120 {\,\rm kr.}
\]
I situationen med en månedlig rente på $1\%$ har vi derimod $n=12$ og $r=0,01$.
Så giver renteformlen, at slutkapitalen er
\[
    1000\cdot 1,01^{12} = 1126,83 {\,\rm kr.}
\]
Den månedlige rentetilskrivning giver en samlet stigning på $126,83$ kr, dvs.
$12,68\%$ efter 1 år. Dette kaldes også den {\em effektive rente}.

\begin{tcolorbox}
\begin{thm}
    Vi udregner den årlige effektive rente $r_{\rm e}$ (som decimaltal) på
    følgende måde:
    \[
        r_{\rm e} = (1+r)^n-1\,,
    \]
hvor $n$ er antallet af terminer på et år, og $r$ er rentefoden pr. termin.
\end{thm}
\end{tcolorbox}
\begin{tcolorbox}
\begin{eks}
    En kassekredit tilskrives en rente på $1,2\%$ om måneden. Den årlige
    effektive rente er da
    \[
        r_{\rm e} = (1+0,012)^{12}-1 = 1,012^{12}-1 = 0,1539\,.
    \]
    Dette svarer altså til $15,39\%$.
\end{eks}
\end{tcolorbox}

\subsection{Gennemsnitlig rentefod}
Hvis rentefoden ændrer sig fra år til år, kan man definere den gennemsnitlige
rentefod:

\begin{tcolorbox}
Den gennemsnitlige rentefod er den \emph{konstante} rentefod, der giver samme
slutkapital som den variable rentefod.
\end{tcolorbox}

Vi ser på et eksempel: Hvis en startkapital på $1000$ kr. tilskrives $3\%$ det
første år, dernæst $9\%$ det andet år, og til sidst $6\%$ det tredje år, så vil
saldoen efter de tre år være givet ved
\[
    1000\cdot 1,03 \cdot 1,09 \cdot 1,06 = 1190,06\,.
\]
Den årlige gennemsnitlige rente findes da ved at løse ligningen
\[
    1000\cdot (1+r)^3 = 1190,06 \Leftrightarrow 
\]
\[
    (1+r)^3=1,19006 \Leftrightarrow
\]
\[
    1+r = 1,0597 \Leftrightarrow
\]
\[
    r = 5,97\%\,.
\]
Læg mærke til, at den gennemsnitlige rente er mindre end gennemsnittet af
rentefoden hvert år.  Ovenstående beregning kan formuleres i følgende sætning:
\begin{tcolorbox}
\begin{thm}
    Den gennemsnitlige rentefod knyttet til rentefødderne $r_1$, $r_2$, \ldots, $r_n$
    er givet ved:
    \[
        r_g = \sqrt[n]{(1+r_1)\cdot(1+r_2)\cdots(1+r_n)}-1 \,.
    \]
\end{thm}
\end{tcolorbox}
\begin{tcolorbox}
\begin{eks}
    En bankkonto med variabel rentefod tilskrives det første år $5\%$, det andet
    år $10\%$, og det tredje år $15\%$ i rente. Den gennemsnitlige årlige rentefod
    i løbet af de tre år er da givet ved:
    \[
        r_g = \sqrt[3]{1,05\cdot1,10\cdot1,15}-1 = \sqrt[3]{1,32825}-1 = 0,0992 = 9,92\%\,.
    \]
På de tre år har kontoen heft en gennemsnitlig årlig rentefod på $9,92\%$.

    Bemærk at rækkefølgen af rentefødderne ikke spiller nogen rolle.
\end{eks}
\end{tcolorbox}


\section{Annuitetsopsparing}
En \emph{annuitetsopsparing} er kendetegnet ved, at man hver termin (f.eks. hver
måned) foretager en fast indbetaling (også kaldet \emph{ydelse}). Efter hver
termin tilskrives rente, som vi har set på tidligere, og ligesom før er
rentefoden også her konstant. Et eksempel på en annuitetsopsparing er en
børneopsparing, hvor forældrene indbetaler et fast beløb hver måned, indtil
barnet bliver 18 år. Til en børneopsparing hører der typisk en relativt høj
rentefod, fordi opsparingen ikke kan hæves før barnet fylder 18 år.

Vi begynder med at se på et eksempel:
\begin{tcolorbox}
\subsection*{Eksempel}
I slutningen af hvert år sætter vi 100kr. ind på en konto, og samtidig
tilskrives en rente på 5 $\%$ hvert år. Renten beregnes ud fra
opsparingen inden indbetalingen.  Det giver følgende udvikling:
\\

\begin{tabular}{|l|l|l|}
    \hline
    \textbf{Antal} & \textbf{Rente} & \textbf{Opsparing} \\
    \textbf{år} &  & \\
    \hline
    1 & 0 & 0 + 0 + 100 = 100 \\
    \hline
    2 & $100\cdot 0,05 = 5$ & 100 + 5 + 100 = 205 \\
    \hline
    3 & $205\cdot 0,05 = 10$ & 205 + 10 + 100 = 315 \\
    \hline
    4 & $315\cdot 0,05 = 16$ & 315 + 16 + 100 = 431 \\
    \hline
    5 & $431\cdot 0,05 = 22$ & 431 + 22 + 100 = 553 \\
    \hline
    6 & $553\cdot 0,05 = 28$ & 553 + 28 + 100 = 681 \\
    \hline
    7 & $681\cdot 0,05 = 34$ & 681 + 34 + 100 = 815 \\
    \hline
    8 & $815\cdot 0,05 = 41$ & 815 + 41 + 100 = 956 \\
    \hline
\end{tabular}
\\

Altså ser vi, at efter 8 år er der 956 kr. på saldoen. De 800 kr. er den
samlede indbetaling, og de resterende 156 kr. er den samlede rente.
Denne udvikling er vist i den følgende figur, hvor saldo, rente og indbetaling er vist med hver sin farve.
\begin{figure}[H]
    \centering
    \includegraphics{fin-2}
\end{figure}
\end{tcolorbox}

Vi ser nu på en alternativ måde at foretage denne beregning. Dette vil lede os
til en generel formel for opsparingens værdi til et vilkårligt tidspunkt. Vi
betragter hver indbetaling for sig, og bruger renteformlen til at beregne den
samlede rente til hver indbetaling.

Den første indbetaling foregik efter 1 år, og ved udgangen af de 8 år
er der ialt blevet tilskrevet renter 7 gange.  Det giver en samlet værdi på:
\[
    100\cdot 1,05^7 = 141 {\,\,\rm kr.}
\]
Denne udregning gentager vi for alle de øvrige indbetalinger, og lægger alle
værdierne sammen.  Dette er vist i den følgende tabel:
\\

\begin{tabular}{|l|l|l|}
    \hline
    \textbf{Antal år} & \textbf{Indbetaling} & \textbf{Værdi efter 8 år} \\
    \hline
    1 & 100 & $100 \cdot 1,05^7 = 141$ \\
    \hline
    2 & 100 & $100 \cdot 1,05^6 = 134$ \\
    \hline
    3 & 100 & $100 \cdot 1,05^5 = 128$ \\
    \hline
    4 & 100 & $100 \cdot 1,05^4 = 122$ \\
    \hline
    5 & 100 & $100 \cdot 1,05^3 = 116$ \\
    \hline
    6 & 100 & $100 \cdot 1,05^2 = 110$ \\
    \hline
    7 & 100 & $100 \cdot 1,05 = 105$ \\
    \hline
    8 & 100 & $100 = 100$ \\
    \hline
    \textbf{Samlet} & \textbf{800} & \textbf{956} \\
    \hline
\end{tabular}
\\

Udregningen i ovenstående tabel kan udtrykkes på følgende måde:
\[
A_8 = 100\cdot 1,05^7 + 100 \cdot 1,05^6 + \ldots + 100 \cdot 1,05 + 100 \;.
\]
Generelt kan dette skrives som:
\[
A_n = y\cdot(1+r)^{n-1} + \ldots + y\cdot(1+r) + y \,,
\]
hvor $A_n$ er saldoen efter i alt $n$ terminer,  $y$ er den faste
ydelse, $r$ er rentefoden, og $n$ er det samlede antal terminer.

I praksis benytter man følgende formel til at udregne saldoen:
\begin{tcolorbox}
\begin{thm}
Saldoen $A_n$ efter $n$ terminer, hvor der til hver termin indbetales en fast ydelse $y$,
samt tilskrives rente med rentefoden $r$, er givet ved følgende formel:
\[
A_n = y\cdot\frac{(1+r)^n-1}{r}\,.
\]
\end{thm}
\end{tcolorbox}

Med denne formel kan vi beregne ovenstående saldo efter 8 terminer:
\[
A_8 = 100\cdot\frac{(1+0,05)^8-1}{0,05} = 954,91 \,.
\]
Forskellen skyldes, at vi i tabellen har foretaget afrundinger undervejs.


\begin{tcolorbox}
\begin{proof}
Beviset for ovenstående sætning tager udgangspunkt i formlen fra før:
\[
A_n = y\cdot(1+r)^{n-1} + \ldots + y\cdot(1+r) + y \,.
\]
Vi søger nu at reducere dette udtryk.  Af nemhedsgrunde sætter vi $1+r = a$:
\[
A_n = y\cdot a^{n-1} + \ldots + y\cdot a + y \,.
\]
Så sætter vi $y$ uden for en parentes:
\[
A_n = y\cdot\big(a^{n-1} + \ldots + a + 1\big) \,.
\]
Nu bruger vi et smart trick: Vi ganger og dividerer med $a-1$:
\[
A_n = y\cdot\frac{\big(a^{n-1} + \ldots + a + 1\big)\cdot(a-1)}{a-1} \,.
\]
I tælleren ganger vi parenteserne sammen:
\[
A_n = y\cdot\frac{a^{n} + \ldots + a^2 + a - a^{n-1} - \ldots - a - 1}{a-1} \,.
\]
Til sidst reduceres tælleren:
\[
A_n = y\cdot\frac{a^{n} - 1}{a-1} \,.
\]
Vi indsætter nu definitionen på fremskrivningsfaktoren $a=1+r$:
\[
A_n = y\cdot\frac{(1+r)^{n} - 1}{r} \,.
\]
Dermed er det ønskede blevet vist.
\end{proof}
\end{tcolorbox}

\begin{tcolorbox}
\subsection*{Eksempel}
Der indsættes 500 kr. hver måned på en højrentekonto, som giver
0,4 $\%$ om måneden. Hvad bliver saldoen efter 4 år?

Vi udregner antallet af terminer $n=4\cdot12 = 48$ og dernæst:
\[
A_{48} = 500\cdot\frac{1,004^{48}-1}{0,004} = 26401 \,.
\]
Af det samlede beløb udgør indbetalingerne $500\cdot 48 = 24000$ kr., og
rentetilskrivningerne udgør resten, dvs. 2401 kr. En rente på 0,4 $\%$
om måneden udgør ca. 10 $\%$ efter 4 år.

\end{tcolorbox}

\section{Annuitetslån}
Et \emph{annuitetslån} er (ligesom annuitetsopsparing) kendetegnet ved, at man
hver termin (f.eks. hver måned) foretager en fast indbetaling (også kaldet
\emph{ydelse}). Efter hver termin tilskrives rente, som vi har set på
tidligere, og ligesom før er rentefoden også her konstant. Dog er der her tale
om en gæld, således at den tilskrevne rente øger gælden, mens indbetalingerne
mindsker gælden.  De mest almindelige lån er af denne type, f.eks. forbrugslån,
billån, samt kreditforeningslån til huskøb.

Vi ser på et eksempel først:
\begin{tcolorbox}
\subsection*{Eksempel}
En person låner 1400 kr til en rentefod af $10\%$ om året, og betaler en fast
ydelse på 300 kr om året.

Ved udgangen af hvert år øges gælden med renten, men nedskrives derefter med
ydelsen. Renten beregnes ud fra gælden inden indbetalingen.  Det giver
følgende udvikling:
\\

\begin{tabular}{|l|l|l|}
    \hline
    \textbf{Antal år} & \textbf{Rente} & \textbf{Gæld} \\
    \hline
    1 & $1400\cdot 0,1 = 140$ & 1400 + 140 - 300 = 1240 \\
    \hline
    2 & $1240\cdot 0,1 = 124$ & 1240 + 124 - 300 = 1064 \\
    \hline
    3 & $1064\cdot 0,1 = 106$ & 1064 + 106 - 300 =  870 \\
    \hline
    4 & $870\cdot 0,1 = 87$ & 870 + 87 - 300 = 657 \\
    \hline
    5 & $657\cdot 0,1 = 66$ & 657 + 66 - 300 = 423 \\
    \hline
    6 & $423\cdot 0,1 = 42$ & 423 + 42 - 300 = 165 \\
    \hline
    7 & $165\cdot 0,1 = 17$ & 165 + 17 - 182 = 0 \\
    \hline
\end{tabular}
\\

Denne udvikling er illustreret i nedenstående figur, som viser gæld, afdrag og rente:
\begin{figure}[H]
    \centering
    \includegraphics{fin-3}
\end{figure}

Læg mærke til, at ydelsen det sidste år kun er på 182 kr.

Det samlede beløb der er betalt for lånet er summen af alle ydelserne, dvs:
\[
6\cdot 300 + 182 = 1982 \,\,\mbox{kr}\,.
\]
Den samlede rente betalt må derfor være $1982 - 1400 = 582$ kr.

\end{tcolorbox}

Vi ser her, at den årlige rente falder, efterhånden som gælden falder. Det
beløb, som gælden falder med hvert år, kaldes \emph{afdrag}. Der gælder
altså følgende sammenhæng:
\[
\mbox{rente + afdrag = ydelse}\,.
\]
I begyndelsen af et annuitetslån vil renten være høj og afdragene være små.
Gælden vil derfor falde meget langsomt i begyndelsen af lånets løbetid.  Gælden
vil først være faldet væsentligt, når der er betalt en hel del afdrag på lånet.
Efterhånden som gælden mindskes, vil renten blive mindre og afdragene
større. Dette er illustreret på følgende figur, som viser afdrag og rente.
\begin{figure}[H]
    \centering
    \includegraphics{fin-4}
\end{figure}

Det er hensigtsmæssigt at have formler til at beregne ydelse og/eller løbetiden
ud fra oplysninger om lånets størrelse og rentefoden.  Generelt har vi følgende
sammenhæng:

\begin{tcolorbox}
\begin{thm}
Den faste ydelse $y$ kan beregnes ud fra lånets størrelse (dvs gældens
værdi i begyndelsen) $G$, rentefoden $r$ og antallet $n$ terminer
\[
    y = G\cdot\frac{r}{1-(1+r)^{-n}}\,.
\]
\end{thm}
\end{tcolorbox}

I eksemplet fra før har vi, at lånets løbetid er 7 år, rentefoden er 10 $\%$, og
lånets størrelse er 1400 kr. Det giver følgende ydelse:
\[
y = 1400\cdot\frac{0,1}{1-(1+0,1)^{-7}} = 288 \,\,\mbox{kr}\,.
\]
Ydelsen på de 300 kr. i eksemplet er større end 288 kr., og derfor bliver den
(eller de) sidste ydelse mindre.

\begin{proof}
Vi lader $G_n$ være gældens størrelse efter $n$ terminer. Vi undersøger nu gældens størrelse
efter de tre første terminer.
I begyndelsen er gældens størrelse
\[
G_0 = G \,.
\]
Efter den første termin er gælden vokset med de tilskrevne renter $(r\cdot G_0)$, men faldet
med den indbetalte ydelse $(y)$.  Dvs efter den føste termin er gældens størrelse
\[
G_1 = G_0 + r\cdot G_0 - y = G_0\cdot (1+r) - y \,.
\]
Af nemhedsgrunde sætter vi $1+r = a$:
\[
G_1 = G_0\cdot a - y \,.
\]
Dvs efter hver termin bliver restgælden multipliceret med $a$ og derefter reduceret med $y$.
Efter to terminer er restgælden nu:
\[
G_2 = G_1 \cdot a - y \,.
\]
Heri indsættes resultatet for $G_1$:
\[
G_2 = \Big( G_0 \cdot a - y \Big) \cdot a - y = G_0 \cdot a^2 - y \cdot a - y \,.
\]
Efter tre terminer er restgælden:
\[
G_3 = G_2 \cdot a - y \,.
\]
Heri indsættes resultatet for $G_2$:
\[
G_3 = \Big( G_0 \cdot a^2 - y\cdot a - y \Big) \cdot a - y 
= G_0 \cdot a^3 - y \cdot a^2 - y \cdot a - y \,.
\]
Dette kan også skrives som:
\[
G_3 = G_0 \cdot a^3 - y \cdot \Big( a^2 + a + 1 \Big) \,.
\]
Dette system fortsætter åbenbart, og efter $n$ terminer bliver restgælden:
\[
G_n = G_0 \cdot a^n - y \cdot \Big( a^{n-1} + \ldots + \cdot a + 1 \Big) \,.
\]
Vi har tidligere vist, at
\[
a^{n-1} + \ldots + \cdot a + 1  = \frac{a^n - 1}{a-1} \,.
\]
Dermed kan restgælden efter $n$ terminer skrives som:
\[
G_n = G_0 \cdot a^n - y \cdot \frac{a^n - 1}{a-1} \,.
\]
Da restgælden efter $n$ terminer er $0$, så har vi:
\[
0 = G_0 \cdot a^n - y \cdot \frac{a^n - 1}{a-1} \,.
\]
Heri isoleres $y$:
\[
y = G_0 \cdot \frac{a^n \cdot (a-1)}{a^n - 1} \,.
\]
Til sidst forlænger vi brøken med $a^{-n}$:
\[
y = G_0 \cdot \frac{a-1}{1 - a^{-n}} \,.
\]
Vi indsætter nu $a = 1+r$:
\[
    y = G_0 \cdot \frac{r}{1 - (1+r)^{-n}} \,.
\]

\end{proof}

Vi ser nu på endnu et eksempel:
\begin{tcolorbox}
\subsection*{Eksempel}
Vi vil låne 5000 kr til en rentefod på $1 \%$ om måneden, og vil betale lånet
tilbage over 4 år (dvs. 48 terminer).  Vi skal dermed betale en månedlig
ydelse givet ved
\[
    y = 5000\cdot\frac{0,01}{1-(1+0,01)^{-48}} = 132 \,\,\mbox {kr} \,. 
\]
I løbet af hele lånets løbetid vil vi i alt have betalt $132\cdot 48 = 6336$ kr. Af
disse udgør renterne i alt $6336-5000 = 1336$ kr.
\end{tcolorbox}

Undertiden er man interesseret i løbetiden, hvis ydelsen er givet.
Dette løser vi ved at isolere $n$ i formlen, hvilket giver:
\begin{tcolorbox}
\begin{thm}
\[
    n = -\frac{\log\big(1-\frac{G\cdot r}{y}\big)}{\log(1+r)} \,.
\]
\end{thm}
\begin{proof}
Vi begynder med formlen i sætning 3.1:
\[
    y = G\cdot\frac{r}{1-(1+r)^{-n}}\,.
\]
Vi ganger først med nævneren og dividerer med $y$:
\[
1-(1+r)^{-n} = \frac{G\cdot r}{y}\,.
\]
Ved at flytte leddene fås
\[
1 - \frac{G\cdot r}{y} =(1+r)^{-n}\,.
\]
Nu tages logaritmen på begge sider:
\[
\log\Big(1 - \frac{G\cdot r}{y}\Big) =\log\Big((1+r)^{-n}\Big) = -n\log(1+r)\,.
\]
Til sidst divideres med $-\log(1+r)$ på begge sider:
\[
n = -\frac{\log\big(1-\frac{G\cdot r}{y}\big)}{\log(1+r)} \,.
\]
Vi har således vist det ønskede.
\end{proof}
\end{tcolorbox}

\begin{tcolorbox}
\subsection*{Eksempel}
Vi ønsker at låne penge til en bil. Det samlede lån bliver på 120000 kr, og
rentefoden er $0,8\%$ om måneden. Vi kan maksimalt betale 1800 kr om
måneden, og vil finde løbetiden.  Vi indsætter i formlen:
\[
n = -\frac{\log\big(1-\frac{120000\cdot0,008}{1000}\big)}{\log(1+0,008)} = 96 \,,
\]
hvilket svarer til 8 år.
I den periode er der i alt indbetalt $96\cdot 1800 = 172800$ kr, og det betyder,
at renterne i alt udgør $172800-120000 = 52800$ kr.
\end{tcolorbox}

\end{document}

