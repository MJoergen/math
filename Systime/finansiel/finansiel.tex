\documentclass[12pt,oneside,a4paper]{article}

\usepackage[utf8]{inputenc} % Lærer LaTeX at forstå unicode - HUSK at filen skal
% være unicode (UTF-8), standard i Linux, ikke i
% Win.

\usepackage[danish]{babel} % Så der fx står Figur og ikke Figure, Resumé og ikke
% Abstract etc. (god at have).

\usepackage{graphicx}
\usepackage{amsfonts}
\usepackage{amsthm}        % Theorems
\usepackage{amsmath}
\usepackage{float}         % Så kan man bedre styre, hvor figurerne havner henne
                           % vha [H].
%\usepackage{hyperref}
\usepackage{tcolorbox}

%\renewcommand{\mid}[1]{{\rm E}\!\left[#1\right]}
\newcommand{\bas}{\begin{eqnarray*}}
\newcommand{\eas}{\end{eqnarray*}}
\newcommand{\be}{\begin{equation}}
\newcommand{\ee}{\end{equation}}
\newcommand{\bea}{\begin{eqnarray}}
\newcommand{\eea}{\end{eqnarray}}

\newtheorem{thm}{Sætning}[section]
\newtheorem{mydef}[thm]{Definition}
\newtheorem{eks}[thm]{Eksempel}
\newtheorem{bevis}[thm]{Bevis}

\DeclareMathSymbol{,}{\mathord}{letters}{"3B}

\title{Finansiel regning}
\date{\vspace{-5ex}}

\begin{document}

\maketitle

%%%%%%%%%%%%%%%%%%%%%%%%%%%%%%%%%%%%%%%%%%%%%%%%%%%%%%%%%%%%%%%%%

\section{Annuiteter}
Vi skal i dette afsnit se på to \emph{annuitetsopsparing} og
\emph{annuitetslån}.  En annuitetsopsparing er kendetegnet ved, at man hver
termin (f.eks. hver måned) indbetaler en fast ydelse. Efter hver termin
tilskrives rente, som vi har set på tidligere, og ligesom før er rentefoden også her
konstant. Et eksempel på en annuitetsopsparing er en børneopsparing, hvor forældrene
indsætter et fast beløb hver måned, indtil barnet bliver 18 år. Til en børneopsparing
hører der typisk en høj rentefod, fordi opsparingen ikke kan hæves før barnet fylder 18 år.

\begin{tcolorbox}
\subsection{Eksempel}
Hver måned sætter vi 1000 kr. ind på en konto, og samtidig tilskrives en rente
på 0,2 $\%$ hver måned. Renten beregnes ud fra opsparingen inden indbetalingen.
Det giver følgende udvikling:
\\

\begin{tabular}{|l|l|l|l|}
    \hline
    \textbf{Antal} & \textbf{Primo} & \textbf{Rente} & \textbf{Ultimo} \\
    \textbf{måneder} & \textbf{saldo} &  & \textbf{saldo} \\
    \hline
    1 & 0 & 0 & 0 + 0 + 1000 = 1000 \\
    \hline
    2 & 1000 & $1000\cdot 0,002 = 2$ & 1000 + 2 + 1000 = 2002 \\
    \hline
    3 & 2002 & $2002\cdot 0,002 = 4$ & 2002 + 4 + 1000 = 3006 \\
    \hline
    4 & 3006 & $3006\cdot 0,002 = 6$ & 3006 + 6 + 1000 = 4012 \\
    \hline
    5 & 4012 & $4012\cdot 0,002 = 8$ & 4012 + 8 + 1000 = 5020 \\
    \hline
    6 & 5020 & $5020\cdot 0,002 = 10$ & 5020 + 10 + 1000 = 6030 \\
    \hline
\end{tabular}
\\

Altså ser vi, at efter 6 måneder er der 6030 kr. på saldoen. De 6000 kr. er den samlede
indbetaling, og de 30 kr. er den samlede rente.

Vi ser nu på en alternativ måde at foretage denne beregning. Det vil lede os til en generel
formel. Vi betragter hver indbetaling for sig, og bruger renteformlen til at beregne
den samlede rente til denne ene indbetaling.
\\

\begin{tabular}{|l|l|l|}
    \hline
    \textbf{Antal måneder} & \textbf{Indbetaling} & \textbf{Værdi efter 6 måneder} \\
    \hline
    1 & 1000 & $1000 \cdot 1,002^5 = 1010$ \\
    \hline
    2 & 1000 & $1000 \cdot 1,002^4 = 1008$ \\
    \hline
    3 & 1000 & $1000 \cdot 1,002^3 = 1006$ \\
    \hline
    4 & 1000 & $1000 \cdot 1,002^2 = 1004$ \\
    \hline
    5 & 1000 & $1000 \cdot 1,002 = 1002$ \\
    \hline
    6 & 1000 & $1000 = 1000$ \\
    \hline
    \textbf{Samlet} & \textbf{6000} & \textbf{6030} \\
    \hline
\end{tabular}
\\

\end{tcolorbox}

I praksis benytter man følgende sætning til at udregne saldoen:
\begin{tcolorbox}
\begin{thm}
Saldoen $A_n$ efter $n$ terminer, hvor der til hver termin indbetales en fast ydelse $y$,
samt tilskrives en rente $r$, er givet ved følgende formel:
\[
A_n = y\cdot\frac{(1+r)^n-1}{r}\,.
\]
\end{thm}
\end{tcolorbox}

Med denne formel kan vi beregne ovenstående saldo efter 5 terminer:
\[
A_5 = 1000\cdot\frac{(1+0,002)^6-1}{0,002} = 6030 \,.
\]

\begin{tcolorbox}
\begin{proof}
Beviset for denne sætning er temmelig langt, og vi deler det derfor op i to dele. For 
det første skal vi indse, at 
\[
A_n = y\cdot(1+r)^{n-1} + \ldots + y\cdot(1+r) + y \,.
\]
Vi søger nu at reducere dette udtryk. Først indfører vi fremskrivningsfaktoren $a=1+r$:
\[
A_n = y\cdot a^{n-1} + \ldots + y\cdot a + y \,.
\]
Så sætter vi $y$ uden for en parentes:
\[
A_n = y\cdot\big(a^{n-1} + \ldots + a + 1\big) \,.
\]
Nu bruger vi et smart trick: Vi ganger og dividerer med $a-1$:
\[
A_n = y\cdot\frac{\big(a^{n-1} + \ldots + a + 1\big)\cdot(a-1)}{a-1} \,.
\]
I tælleren ganger vi parenteserne sammen:
\[
A_n = y\cdot\frac{a^{n} + \ldots + a^2 + a - a^{n-1} - \ldots - a - 1}{a-1} \,.
\]
Til sidst reduceres tælleren:
\[
A_n = y\cdot\frac{a^{n} - 1}{a-1} \,.
\]
Vi indsætter nu definitionen på fremskrivningsfaktoren $a=1+r$:
\[
A_n = y\cdot\frac{(1+r)^{n} - 1}{r} \,.
\]
Og vi har vist det ønskede.
\end{proof}
\end{tcolorbox}

\begin{tcolorbox}
\subsection{Eksempel}
Vi ser på endnu et eksempel. Der indsættes 500 kr. hver måned på en højrente konto, som giver
0,4 $\%$ om måneden. Hvad bliver saldoen efter 4 år?

Vi udregner:
\[
A_{48} = 500\cdot\frac{1,004^{48}-1}{0,004} = 26400,82 \,.
\]
Af det samlede beløb udgør indbetalingerne $500\cdot 48 = 24000$ kr., og rentetilskrivningerne
udgør resten, 2400,82 kr. Så en rente på 0,4 $\%$ om måneden udgør ca. 10 $\%$ efter 4 år.

\end{tcolorbox}

\end{document}

