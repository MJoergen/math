\documentclass[12pt,oneside,a4paper]{article}

\usepackage[utf8]{inputenc} % Lærer LaTeX at forstå unicode - HUSK at filen skal
% være unicode (UTF-8), standard i Linux, ikke i
% Win.

\usepackage[danish]{babel} % Så der fx står Figur og ikke Figure, Resumé og ikke
% Abstract etc. (god at have).

\usepackage{graphicx}
\usepackage{amsfonts}
\usepackage{amsthm}        % Theorems
\usepackage{amsmath}
\usepackage{float}         % Så kan man bedre styre, hvor figurerne havner henne
                           % vha [H].
%\usepackage{hyperref}
\usepackage{tcolorbox}

%\renewcommand{\mid}[1]{{\rm E}\!\left[#1\right]}
\newcommand{\bas}{\begin{eqnarray*}}
\newcommand{\eas}{\end{eqnarray*}}
\newcommand{\be}{\begin{equation}}
\newcommand{\ee}{\end{equation}}
\newcommand{\bea}{\begin{eqnarray}}
\newcommand{\eea}{\end{eqnarray}}

\newtheorem{thm}{Sætning}[section]
\newtheorem{mydef}[thm]{Definition}
\newtheorem{eks}[thm]{Eksempel}
\newtheorem{bevis}[thm]{Bevis}

\DeclareMathSymbol{,}{\mathord}{letters}{"3B}

\title{Finansiel regning}
\date{\vspace{-5ex}}

\begin{document}

\maketitle

%%%%%%%%%%%%%%%%%%%%%%%%%%%%%%%%%%%%%%%%%%%%%%%%%%%%%%%%%%%%%%%%%

\section{Annuitetsopsparing}
En \emph{annuitetsopsparing} er kendetegnet ved, at man hver termin (f.eks. hver
måned) foretager en fast indbetaling (også kaldet \emph{ydelse}). Efter hver
termin tilskrives rente, som vi har set på tidligere, og ligesom før er
rentefoden også her konstant. Et eksempel på en annuitetsopsparing er en
børneopsparing, hvor forældrene indbetaler et fast beløb hver måned, indtil
barnet bliver 18 år. Til en børneopsparing hører der typisk en høj rentefod,
fordi opsparingen ikke kan hæves før barnet fylder 18 år.

\begin{tcolorbox}
\subsection{Eksempel}
I slutningen af hver måned sætter vi 1000 kr. ind på en konto, og samtidig
tilskrives en rente på 0,2 $\%$ hver måned. Renten beregnes ud fra
opsparingen inden indbetalingen.  Det giver følgende udvikling:
\\

\begin{tabular}{|l|l|l|}
    \hline
    \textbf{Antal} & \textbf{Rente} & \textbf{Opsparing} \\
    \textbf{måneder} &  & \\
    \hline
    1 & 0 & 0 + 0 + 1000 = 1000 \\
    \hline
    2 & $1000\cdot 0,002 = 2$ & 1000 + 2 + 1000 = 2002 \\
    \hline
    3 & $2002\cdot 0,002 = 4$ & 2002 + 4 + 1000 = 3006 \\
    \hline
    4 & $3006\cdot 0,002 = 6$ & 3006 + 6 + 1000 = 4012 \\
    \hline
    5 & $4012\cdot 0,002 = 8$ & 4012 + 8 + 1000 = 5020 \\
    \hline
    6 & $5020\cdot 0,002 = 10$ & 5020 + 10 + 1000 = 6030 \\
    \hline
\end{tabular}
\\

Altså ser vi, at efter 6 måneder er der 6030 kr. på saldoen. De 6000 kr. er den
samlede indbetaling, og de 30 kr. er den samlede rente.
\end{tcolorbox}

Vi ser nu på en alternativ måde at foretage denne beregning. Det vil lede os
til en generel formel. Vi betragter hver indbetaling for sig, og bruger
renteformlen til at beregne den samlede rente til denne ene indbetaling.

Den første indbetaling foregik efter 1 måned, og ved udgangen af de 6 måneder
er der ialt blevet tilskrevet renter 5 gange.  Det giver en samlet værdi på:
\[
    1000\cdot 1,002^5 = 1010 {\,\,\rm kr.}
\]
Denne udregning gentager vi for alle de øvrige indbetalinger, og lægger alle
værdierne sammen.  Dette er vist i den følgende tabel:
\\

\begin{tabular}{|l|l|l|}
    \hline
    \textbf{Antal måneder} & \textbf{Indbetaling} & \textbf{Værdi efter 6 måneder} \\
    \hline
    1 & 1000 & $1000 \cdot 1,002^5 = 1010$ \\
    \hline
    2 & 1000 & $1000 \cdot 1,002^4 = 1008$ \\
    \hline
    3 & 1000 & $1000 \cdot 1,002^3 = 1006$ \\
    \hline
    4 & 1000 & $1000 \cdot 1,002^2 = 1004$ \\
    \hline
    5 & 1000 & $1000 \cdot 1,002 = 1002$ \\
    \hline
    6 & 1000 & $1000 = 1000$ \\
    \hline
    \textbf{Samlet} & \textbf{6000} & \textbf{6030} \\
    \hline
\end{tabular}
\\

Udregningen i den ovenstående tabel kan udtrykkes på følgende måde:
\[
A_n = y\cdot(1+r)^{n-1} + \ldots + y\cdot(1+r) + y \,,
\]
hvor $A_n$ er saldoen efter i alt $n$ terminer,  $y$ er den faste (månedlige)
ydelse, $r$ er rentefoden, og $n$ er det samlede antal terminer.

I praksis benytter man følgende sætning til at udregne saldoen:
\begin{tcolorbox}
\begin{thm}
Saldoen $A_n$ efter $n$ terminer, hvor der til hver termin indbetales en fast ydelse $y$,
samt tilskrives rente med rentefoden $r$, er givet ved følgende formel:
\[
A_n = y\cdot\frac{(1+r)^n-1}{r}\,.
\]
\end{thm}
\end{tcolorbox}

Med denne formel kan vi beregne ovenstående saldo efter 6 terminer:
\[
A_6 = 1000\cdot\frac{(1+0,002)^6-1}{0,002} = 6030 \,.
\]

\begin{tcolorbox}
\begin{proof}
Beviset for ovenstående sætning tager udgangspunkt i formlen fra før:
\[
A_n = y\cdot(1+r)^{n-1} + \ldots + y\cdot(1+r) + y \,.
\]
Vi søger nu at reducere dette udtryk. Først indfører vi fremskrivningsfaktoren $a=1+r$:
\[
A_n = y\cdot a^{n-1} + \ldots + y\cdot a + y \,.
\]
Så sætter vi $y$ uden for en parentes:
\[
A_n = y\cdot\big(a^{n-1} + \ldots + a + 1\big) \,.
\]
Nu bruger vi et smart trick: Vi ganger og dividerer med $a-1$:
\[
A_n = y\cdot\frac{\big(a^{n-1} + \ldots + a + 1\big)\cdot(a-1)}{a-1} \,.
\]
I tælleren ganger vi parenteserne sammen:
\[
A_n = y\cdot\frac{a^{n} + \ldots + a^2 + a - a^{n-1} - \ldots - a - 1}{a-1} \,.
\]
Til sidst reduceres tælleren:
\[
A_n = y\cdot\frac{a^{n} - 1}{a-1} \,.
\]
Vi indsætter nu definitionen på fremskrivningsfaktoren $a=1+r$:
\[
A_n = y\cdot\frac{(1+r)^{n} - 1}{r} \,.
\]
Og vi har vist det ønskede.
\end{proof}
\end{tcolorbox}

\begin{tcolorbox}
\subsection{Eksempel}
Vi ser på endnu et eksempel. Der indsættes 500 kr. hver måned på en højrente konto, som giver
0,4 $\%$ om måneden. Hvad bliver saldoen efter 4 år?

Vi udregner $n=4\cdot12 = 48$ og dernæst:
\[
A_{48} = 500\cdot\frac{1,004^{48}-1}{0,004} = 26400,82 \,.
\]
Af det samlede beløb udgør indbetalingerne $500\cdot 48 = 24000$ kr., og rentetilskrivningerne
udgør resten, dvs. 2400,82 kr. Så en rente på 0,4 $\%$ om måneden udgør ca. 10 $\%$ efter 4 år.

\end{tcolorbox}

\section{Annuitetslån}
Et \emph{annuitetslån} er (ligesom annuitetsopsparing) kendetegnet ved, at man
hver termin (f.eks. hver måned) foretager en fast indbetaling (også kaldet
\emph{ydelse}). Efter hver termin tilskrives rente, som vi har set på
tidligere, og ligesom før er rentefoden også her konstant. Dog er der her tale
om en gæld, således at den tilskrevne rente øger gælden, mens indbetalingerne
mindsker gælden.  De mest almindelige lån er af denne type, f.eks. forbrugslån,
billån, samt kreditforeningslån til huskøb.

Vi ser på et eksempel først:
\begin{tcolorbox}
\subsection{Eksempel}
En person låner 14000 kr til en rentefod af $1\%$ om måneden, og betaler en fast
ydelse på 400 kr om måneden.

I slutningen af hver måned øges gælden med renten, men nedskrives derefter med
ydelsen. Renten beregnes ud fra gælden inden indbetalingen.  Det giver
følgende udvikling:
\\

\begin{tabular}{|l|l|l|}
    \hline
    \textbf{Antal} & \textbf{Rente} & \textbf{Gæld} \\
    \textbf{måneder} &  & \\
    \hline
    1 & $14000\cdot 0,01 = 140$ & 14000 + 140 - 400 = 13740 \\
    \hline
    2 & $13740\cdot 0,01 = 137$ & 13740 + 137 - 400 = 13477 \\
    \hline
    3 & $13477\cdot 0,01 = 135$ & 13477 + 135 - 400 = 13212 \\
    \hline
    4 & $13212\cdot 0,01 = 132$ & 13212 + 132 - 400 = 12944 \\
    \hline
    5 & $12944\cdot 0,01 = 129$ & 12944 + 129 - 400 = 12673 \\
    \hline
    6 & $12673\cdot 0,01 = 127$ & 12673 + 127 - 400 = 12400 \\
    \hline
\end{tabular}
\\

Vi ser her, at den månedlige rente falder efterhånde som gælden falder. Det
beløb, som gælden falder med hver måned, kaldes \emph{afdrag}. Der gælder
altså følgende sammenhæng:
\[
\mbox{rente + afdrag = ydelse}
\]
I begyndelsen af et annuitetslån vil renten være høj og afdragene vil være små.
Gælden vil derfor falde meget langsomt i begyndelsen af lånets løbetid.
Gælden vil først være faldet væsentligt, når der er betalt en hel del
afdrag på lånet.  Efterhånden som gælden mindskes, så vil renten blive
mindre, og afdragene blive større. Denne variation af rente og afdrag 
hen over lånets løbetid er vist i følgende figur.

BILLEDE

\end{tcolorbox}

Generelt har vi følgende sammenhæng:
\begin{tcolorbox}
\begin{thm}
    Den månedlige ydelse $y$ kan beregens ud fra lånets størrelse 
  (dvs gældens værdi i begyndelsen) $G$, rentefoden $r$ og antallet $n$ terminer
\[
    y = G\cdot\frac{r}{1-(1+r)^{-n}}\,.
\]
\end{thm}
\end{tcolorbox}

\begin{tcolorbox}
\subsection{Eksempel}
Vi vil låne 5000 kr til en rente på $1 \%$ om måneden, og vil betale lånet tilbage over 
4 år (dvs. 48 terminer).
Vi skal dermed betale en månedlig ydelse givet ved
\[
    y = 5000\cdot\frac{0,01}{1-(1+0,01)^{-48}} = 132 \,\,\mbox {kr} \,. 
\]
I løbet af hele lånets løbetid har vi i alt betalt $132\cdot 48 = 6336$ kr. Af
disse udgør renterne i alt $6336-5000 = 1336$ kr.
\end{tcolorbox}

Andre gange er man interesseret i at vide, hvor lang tid et lån skal løbe over, givet
en maksimal ydelse. Dette løser vi ved at isolere $n$ i formlen, hvilket giver:
\[
    n = -\frac{\log\big(1-\frac{G\cdot r}{y}\big)}{\log(1+r)} \,.
\]

\begin{tcolorbox}
\subsection{Eksempel}
Vi ønsker at låne penge til en bil. Det samlede lån bliver på 120000 kr, og
rentefoden er $0,8\%$ om måneden. Vi kan maksimalt 1800 kr om måneden, og
vil gerne vide, hvor lang tid lånet varer.
Vi indsætter i formlen:
\[
n = -\frac{\log\big(1-\frac{120000\cdot0,008}{1000}\big)}{\log(1+0,008)} = 96 \,,
\]
hvilket svarer til 8 år.
I den periode er der i alt indbetalt $96\cdot 1800 = 172800$ kr, og det betyder,
at renterne i alt udgør 172800-120000 = 52800 kr.
\end{tcolorbox}



\end{document}

