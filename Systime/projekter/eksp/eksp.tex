\documentclass[12pt,oneside,a4paper]{article}

\usepackage[utf8]{inputenc} % Lærer LaTeX at forstå unicode - HUSK at filen skal
% være unicode (UTF-8), standard i Linux, ikke i
% Win.

\usepackage[danish]{babel} % Så der fx står Figur og ikke Figure, Resumé og ikke
% Abstract etc. (god at have).

\usepackage{graphicx}
\usepackage{amsfonts}
\usepackage{amsthm}        % Theorems
\usepackage{amsmath}
\usepackage{enumitem}
%\usepackage{hyperref}

%\renewcommand{\mid}[1]{{\rm E}\!\left[#1\right]}
\newcommand{\bas}{\begin{eqnarray*}}
\newcommand{\eas}{\end{eqnarray*}}
\newcommand{\be}{\begin{equation}}
\newcommand{\ee}{\end{equation}}
\newcommand{\bea}{\begin{eqnarray}}
\newcommand{\eea}{\end{eqnarray}}

\theoremstyle{plain}
\newtheorem*{thm}{Sætning}
\newtheorem*{mydef}{Definition}
\newtheorem*{eks}{Eksempel}

\DeclareMathSymbol{,}{\mathord}{letters}{"3B}

\title{Projektopgaver om eksponentialfunktioner}
\date{\vspace{-5ex}}

\begin{document}

\maketitle

\section*{Opgave 1 -- Fordoblingskonstant}
Vi har givet en eksponentialfunktion
\[
    f(x) = b\cdot a^x\,,
\]
hvor $a=1+r$ er fremskrivningsfaktoren og $r$ er vækstraten.
Vækstraten er knyttet til den procentvise stigning $p$ ved
\[
    r = \frac{p}{100}\,,
\]
hvor $p$ angiver stigningen i procent.

Hvis $a>1$ er funktionen voksende, og vi definerer fordoblingskonstanten
$T_2$ ud fra egenskaben
\[
    f(x+T_2) = 2\cdot f(x)\,.
\]

\begin{enumerate}[label=(\alph*)]
    \item Vis, ved at indsætte forskriften i ovenstående ligning, at 
fordoblingskonstanten opfylder følgende ligning:
\[
    a^{T_2} = 2\,.
\]
\end{enumerate}

Der gælder følgende eksakte formel for fordoblingskonstanten:
\[
    T_2 = \frac{\log(2)}{\log(1+r)}\,.
\]
\begin{enumerate}[label=(\alph*), resume]
    \item Bevis ovenstående formel.
\end{enumerate}

Der gælder endvidere følgende tilnærmede formel, som til tider kan være nyttig:
\[
    \bar T_2 = \frac{70}{p} \,.
\]
Her er $\bar T_2$ en tilnærmelse til fordoblingskonstanten $T_2$.

\begin{enumerate}[label=(\alph*), resume]
    \item Udfyld nedenstående tabel, og kommentér på resultatet.
        \begin{center}
            \begin{tabular}{|r|c|l|}
\hline
                $p$ & $T_2$ & $\bar T_2$ \\
                \hline
                1 & & \\
                \hline
                2 & & \\
                \hline
                5 & & \\
                \hline
                10 & & \\
                \hline
                20 & & \\
                \hline
            \end{tabular}
        \end{center}

    \item Tegn graferne for de to funktioner $T_2(p)$ og $\bar T_2(p)$ i samme
        koordinatsystem for $p$ mellem $1$ og $20$.

Det kan vises, at når $x$ er lille, så gælder der, at
        \[
            \log(x) \approx 0,4343 \cdot (x-1) \,.
        \]

    \item Brug ovenstående resultat til at udlede formlen for $\bar T_2$.

\end{enumerate}


\section*{Opgave 2 -- Kvadratrødder uden lommeregner}
Vi har alle lært, at $\sqrt{9} = 3$, fordi $3^2 = 9$. I denne opgave skal vi
arbejde med en metode til at beregne en tilnærmelse til kvadratroden af et
vilkårligt tal, f.eks.
$\sqrt{21}$.

Vi starter med parablen $f(x) = x^2$. For at beregne kvadratroden $\sqrt{21}$,
så er det netop det samme som at løse ligningen $f(x) = 21$, dvs at finde
$x$-koordinaten for skæringspunktet mellem grafen for $f(x)$ og linjen $y=21$,
se figur 1.

\begin{figure}[ht]
    \centering
    \includegraphics[width=0.5\textwidth]{kva1}
    \caption{}
    \label{fig1}
\end{figure}

I kapitel XXX lærte vi, at parablen har en tangent i ethvert punkt $(x_0,\,
x_0^2)$, og at denne tangent har ligningen
\[
    y=2x_0x - x_0^2 \,.
\]

Vi tager nu udgangspunkt i punktet $(4,\,16)$ på parablen. Vi kunne have valgt
et hvilket som helst andet punkt, men det er smart at vælge et punkt, som
ligger tæt på det ønskede punkt med $y$-koordinaten $23$.

\begin{enumerate}[label=(\alph*)]
    \item Vis, at parablen i punktet $(4,\,16)$ har en tangent med ligningen
        $y=8x-16$.
\end{enumerate}

Idéen er nu, at vi kan nemt beregne, hvor denne tangent skærer linjen $y=21$. 
Da tangenten er tæt på parablen, så vil dette skæringspunkt være tæt på 
skæringspunktet med parablen, se figur 2.

\begin{figure}[ht]
    \centering
    \includegraphics[width=0.5\textwidth]{kva2}
    \caption{}
    \label{fig2}
\end{figure}

\begin{enumerate}[label=(\alph*), resume]
    \item Vis, at tangenten skærer linjen $y=21$ når $x=4,625$.
\end{enumerate}

Det vil sige, at vores kvadratrod er tæt på $4,625$. Men vi kan
komme tættere på resultatet ved at gentage processen. Dvs nu tager vi
udgangspunkt i punktet $(4,625,\,4,625^2)$ på parablen. Igen beregner vi en ligning for tangenten i dette punkt.

\begin{enumerate}[label=(\alph*), resume]
    \item Vis, at parablen i punktet $(4,625,\,4,625^2)$ har en tangent med
        ligningen $y=9,25x-21,3906$.
\end{enumerate}
Igen beregner vi nemt, hvor denne nye tangent skærer linjen $y=21$.

\begin{enumerate}[label=(\alph*), resume]
    \item Vis, at denne nye tangent skærer linjen $y=21$ når $x=4,5828$.
\end{enumerate}

Vi er nu kommet frem til, at vores kvadratrod er tæt på $4,5828$.
Det er nu klart, at denne proces kan gentages vilkårligt antal gange.
Det viser sig, at hver gang kommer man tættere på det rigtige resultat, og
at denne proces forløber forbavsende hurtigt.

\begin{enumerate}[label=(\alph*), resume]
    \item Gentag ovenstående proces to gange til, og udfyld de tommme felter i
        nedenstående skema.
        \begin{center}
            \begin{tabular}{|r|c|l|}
\hline
                $x$ & tangent & ny $x$ \\
                \hline
                4 & 8x-16 & 4,625 \\
                \hline
                4,625 & 9,25x-21,3906 & 4,5828 \\
                \hline
                4,5828 & & \\
                \hline
                 & & \\
                \hline
            \end{tabular}
        \end{center}
\end{enumerate}

Nu søger vi at generalisere ovenstående proces til en vilkårlig 
kvadratrod. Vi søger altså nu at bestemme værdien af $\sqrt{A}$, hvor
$A$ er et givet tal. Vi lader igen $x_0$ være vores startværdi.

\begin{enumerate}[label=(\alph*), resume]
    \item Vis, at parablens tangent i punktet $(x_0,\,x_0^2)$ 
        skærer linjen $y=A$ når
        \[
            x = \frac{A+x_0^2}{2x_0}\,.
        \]
    \item Vis, at denne formel kan omskrives til 
        \[
            x = x_0 + \frac{A-x_0^2}{2x_0}\,.
        \]
    På denne form er brøkens tæller netop, hvor stor afvigelsen er i $y$-værdien,
        og hele brøken er justeringen i $x$-værdien.
    \item Vis, at formlen også kan omskrives til
        \[
            x = \frac12 \cdot \Big(\frac{A}{x_0} + x_0\Big) \,.
        \]
    Denne sidste form er velegnet til beregninger, fordi den indeholder færrest
        mulige regne-operationer.

    \item Brug denne sidste formel til at beregne kvadratroden $\sqrt{43}$ med
        mindst fem decimaler. Du skal gentage processen indtil dit resultat
        ikke længere ændrer sig.
\end{enumerate}


\section*{Opgave 3 -- Eksponentialfunktioner uden lommeregner}
Vi har i afsnit XXX beskrevet, hvordan man opløfter et tal til en potens, som
kan skrives som en brøk. Således er f.eks.
\[
    16^{3/4} = (\sqrt[4]{16})^3 = 2^3 = 8 \,.
\]
Denne metode er dog upraktisk til at beregne f.eks.
\[
    10^{1,72} \,.
\]

I stedet for bruger vi en anden metode, som giver en tilnærmelse til den
rigtige værdi.  Betragt først heltallige potenser af $0,5$, dvs.
$0,5$, $0,25$, $0,125$, osv. Disse udregnes nemt ved at multiplicere
det foregående tal med $0,5$.
\begin{enumerate}[label=(\alph*)]
    \item Udfyld de tomme felter i nedenstående tabel med potenser af $0,5$:
        \begin{center}
            \begin{tabular}{|r|l|}
\hline
                $n$ & $0,5^n$ \\
                \hline
                $0$ & $1$ \\
                \hline
                $1$ & $0,5$ \\
                \hline
                $2$ & $0,25$ \\
                \hline
                $3$ & $0,125$ \\
                \hline
                $4$ & $0,0625$ \\
                \hline
                $5$ & $0,03125$ \\
                \hline
                $6$ &  \\
                \hline
                $7$ &  \\
                \hline
            \end{tabular}
        \end{center}
\end{enumerate}

For at bruge denne metode skal man nu skrive potensen (i vort eksempel $1,72$)
som en tilnærmet sum af potenser af $0,5$. I vort eksempel ser vi f.eks., at
\[
    1 + 0,5 + 0,125 + 0,0625 + 0,03125 = 1,71875 \,.
\]
Dette er meget tæt på eksponenten $1,72$. Med denne metode vil vi derfor kunne
beregne $10^{1,71875}$ og dette vil så være tæt på det ønskede $10^{1,72}$.
Således har vi
\[
    10^{1,71875} = 10^{1 + 0,5 + 0,125 + 0,0625 + 0,03125}
\]
\[
    = 10^1 \cdot 10^{0,5} \cdot 10^{0,126} \cdot 10^{0,0625} \cdot 10^{0,03125} \,.
\]

For at komme videre, laver vi nu en tabel over gentagne kvadratrødder af tallet
$10$. Denne tabel skal kun laves én gang, og kan eventuelt udarbejdes ved at
bruge metoden angivet i projektet oven over.  Således er:
\[
    \sqrt{10} = 10^{0,5} = 3,162278 \,,
\]
og
\[
    \sqrt{3,162278} = 10^{0,25} = 1,778279 \,.
\]
Således kan man halvere potensen ved at fortsætte med at tage kvadratroden af
det foregående tal.

\begin{enumerate}[label=(\alph*), resume]
    \item Udfyld de tomme felter i nedenstående tabel ved at tage kvadratroden
        af det foregående tal:
        \begin{center}
            \begin{tabular}{|l|l|}
\hline
                $x$ & $10^x$ \\
                \hline
                $1$ & $10,0$ \\
                \hline
                $0,5$ & $3,162278$ \\
                \hline
                $0,5^2 = 0,25$ & $1,778279$ \\
                \hline
                $0,5^3 = 0,125$ &  \\
                \hline
                $0,5^4$ & \\
                \hline
                $0,5^5$ & \\
                \hline
                $0,5^6$ & \\
                \hline
                $0,5^7$ & \\
                \hline
            \end{tabular}
        \end{center}
\end{enumerate}

Med denne tabel kan man nemt beregne potenser af andre tal end de givne. F.eks. er 
\[
    10^{1,75} = 10^{1+0,5+0,25} = 10^1 \cdot 10^{0,5} \cdot 10^{0,25}
    = 10 \cdot 3,162278 \cdot 1,778279 = 56,234126 \,.
\]

\begin{enumerate}[label=(\alph*), resume]
    \item Brug nu samme metode til at udregnge potensen $10^{1,71875}$ som en
        tilnærmelse til $10^{1,72}$.
\end{enumerate}

\begin{enumerate}[label=(\alph*), resume]
    \item Brug ovenstående metode til at vise, at tallet $1,72$ kan tilnærmes som
        \[
            0,5^0 + 0,5^1 + 0,5^3 + 0,5^4 + 0,5^5 + 0,5^{10} = 1,719727 \,.
        \]
\item Brug ovenstående metode til at beregne $10^{1,719727}$ som en bedre
        tilnærmelse til $10^{1,72}$.
\end{enumerate}

\section*{Opgave 4 -- Parablens toppunkt}
Det er tidligere blevet vist, at grafen hørende til funktionen $f(x) = ax^2 + bx + c$
er en parabel, og at denne parabel har et toppunkt givet ved koordinaterne:
\[
    T = \Big(\frac{-b}{2a}, \frac{-d}{4a}\Big) \,,
\]
hvor diskriminanten $d=b^2-4ac$. Her skal vi se på et alternativt bevis for dette
resultat.

Antag, at $a>0$. Vi undersøger nu skæringspunkterne mellem grafen for $f(x)$ og
vandrette linjer givet ved $y=k$, hvor $k$ er en endnu ukendt konstant.
Specielt er vi interesseret i, hvordan antallet af skæringspunkter afhænger af
værdien af $k$.

Som eksempel viser figur 3 grafen for funktionen $f(x) = x^2-6x+7$ samt tre linjer 
af formen $y=k$. For $k=5$ er der to skæringspunkter, for $k=-2$ er der ét skæringspunkt,
og for $k=-5$ er der ingen skæringspunkter.

\begin{figure}[ht]
    \centering
    \includegraphics[width=0.5\textwidth]{kva3}
    \caption{}
    \label{fig3}
\end{figure}

\begin{enumerate}[label=(\alph*)]
    \item Vis, at ligningen $f(x) = k$ er en andengradsligning.
\end{enumerate}

\begin{enumerate}[label=(\alph*), resume]
    \item Vis, at antallet af skæringspunkter afhænger af fortegnet af 
        \[
            b^2-4a(c-k) \,.
        \]
    \item Vis, at når $k$ er stor, så er der to skæringspunkter, og at
        når $k$ er lille, så er der ikke nogen skæringspunkter.
    \item Vis, at der er netop ét skæringspunkt, når
        \[
            k = \frac{-d}{4a} \,.
        \]
    \item Forklar, hvorfor dette skæringspunkt netop må være parablens toppunkt.
\end{enumerate}

\section*{Opgave 5 -- Parablens tangent}
I denne opgave skal vi arbejde med tangenter. Tangenter behandles mere generelt
i kapitlet om differentialregning, men her undersøges de specielt for parabler.
Vi bruger derfor i denne opgave følgende definition.
\begin{mydef}
    Tangenten til en parabel i et givet punkt er en ret linje gennem punktet, som
    ikke skærer parablen andre steder.
\end{mydef}
Geometrisk er en tangent en ret linje som er "parallel" med grafen i et bestemt punkt.

Vi begynder med parablen givet ved funktionen $f(x)=x^2$. Vi vil udregne en ligning 
for tangenten til parablen i punktet $(3,\, 9)$. Ligningen for en ret linje gennem
et givet punkt er 
\[
    y-y_0 = a\cdot (x-x_0) \,,
\]
hvor $a$ er linjens hældning og punktet har koordinaterne $(x_0,\, y_0)$.
I figur 4 er tegnet parablen samt tangenten til denne i punktet $(3,\, 9)$.

\begin{figure}[ht]
    \centering
    \includegraphics[width=0.5\textwidth]{kva4}
    \caption{}
    \label{fig4}
\end{figure}

\begin{enumerate}[label=(\alph*)]
    \item Vis, at tangentens ligning kan skrives som 
        \[
            y-9 = a\cdot(x-3)\,,
        \]
    hvor $a$ er en endnu ukendt konstant.
\end{enumerate}
For at beregne værdien af hældnignen $a$, så benytter vi, at tangenten og parablen
kun må have ét skæringspunkt.

\begin{enumerate}[label=(\alph*), resume]
    \item Vis, at skæringspunkterne mellem tangenten og parablen er givet
        ved løsningerne til følgende ligning
        \[
            a\cdot(x-3) + 9 = x^2 \,.
        \]
    \item Vis, at dette er en andengradsligning, og at diskriminanten er givet ved
        \[
            d = a^2 - 12a + 36 \,.
        \]
    \item Vis, at diskriminanten kan omskrives til
        \[
            d = (a-6)^2 \,.
        \]
    \item Forklar, hvorfor tangentens hældning må være $a=6$.
    \item Opskriv tangentens ligning på formen $y=ax+b$.
\end{enumerate}

Nu vælger vi et vilkårligt punkt på parablen, dvs punktet $(x_0,\, x_0^2)$, hvor
$x_0$ er vilkårlig.
\begin{enumerate}[label=(\alph*), resume]
    \item Vis, at skæringspunkterne mellem tangenten og parablen er givet
        ved løsningerne til følgende ligning
        \[
            a\cdot(x-x_0) + x_0^2 = x^2 \,.
        \]
    \item Vis, at dette er en andengradsligning, og at diskriminanten er givet ved
        \[
            d = a^2 - 4x_0a + 4x_0^2 \,.
        \]
    \item Vis, at diskriminanten kan omskrives til
        \[
            d = (a-2x_0)^2 \,.
        \]
    \item Forklar, hvorfor tangentens hældning må være $a=2x_0$.
    \item Opskriv tangentens ligning på formen $y=ax+b$.
\end{enumerate}

\end{document}

