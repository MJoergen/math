\documentclass[12pt,oneside,a4paper]{article}

\usepackage[utf8]{inputenc} % Lærer LaTeX at forstå unicode - HUSK at filen skal
% være unicode (UTF-8), standard i Linux, ikke i
% Win.

\usepackage[danish]{babel} % Så der fx står Figur og ikke Figure, Resumé og ikke
% Abstract etc. (god at have).

\usepackage{graphicx}
\usepackage{amsfonts}
\usepackage{amsthm}        % Theorems
\usepackage{amsmath}
\usepackage{enumitem}
%\usepackage{hyperref}

%\renewcommand{\mid}[1]{{\rm E}\!\left[#1\right]}
\newcommand{\bas}{\begin{eqnarray*}}
\newcommand{\eas}{\end{eqnarray*}}
\newcommand{\be}{\begin{equation}}
\newcommand{\ee}{\end{equation}}
\newcommand{\bea}{\begin{eqnarray}}
\newcommand{\eea}{\end{eqnarray}}

\theoremstyle{plain}
\newtheorem*{thm}{Sætning}
\newtheorem*{mydef}{Definition}
\newtheorem*{eks}{Eksempel}

\DeclareMathSymbol{,}{\mathord}{letters}{"3B}

\title{Projekter}
\date{\vspace{-5ex}}

\begin{document}

\maketitle

\section*{Projekt 1 -- Beregning af logaritmer uden CAS}
I dette projekt vil vi finde frem til en metode til at beregne
titals-logaritmer uden brug af CAS.

Først ser vi på $\log x$ til tal tæt på $x=1$.

\begin{enumerate}[label=(\alph*)]
    \item Vis ved hjælp af differentialregning, at funktionen $f(x) = \log x$ i
        punktet $(1, 0)$ har en tangent med ligningen $y = 0,4343 \cdot (x-1)$.
\end{enumerate}
Da tangenten er en god approksimation til funktionen tæt på røringspunktet,
så har vi altså, at 
\[
    \tag{1}
    \log x \approx 0,4343\cdot(x-1),\quad \mbox{for $x\approx 1$}\,.
\]
For eksempel har vi med denne formel, at $\log 1,01 \approx 0,4343\cdot 0,01 = 0,004343$.
CAS giver værdien $\log 1,01 = 0,004321$. Den procentvise afvigelse er derfor:
\[
    \frac{0,004343-0,004321}{0,004321}\cdot 100 = 0,5 \% \,.
\]

\begin{enumerate}[label=(\alph*), resume]
    \item Benyt formel (1) til at beregne en tilnærmelse til $\log
        1,06$, og sammenlign med den korrekte værdi udregnet på CAS.
    \item Hvor stor er den procentvise afvigelse?
\end{enumerate}

Vi kan finde en mere nøjagtig formel end (1) på følgende måde: Først erstat $x$ med
$\frac1x$ i formel (1). Det giver
\[
    \tag{2}
    \log\frac1x \approx 0,4343\cdot\Big(\frac1x-1\Big)\,.
\]

\begin{enumerate}[label=(\alph*), resume]
    \item Vis, at $\log\frac1x = -\log x$.
    \item Vis dernæst, ved at trække (2) fra (1), at 
        \[
            \tag{3}
            \log x \approx \frac{0,4343}{2} \cdot \Big(x-\frac1x\Big),\quad
            \mbox{for $x\approx 1$}\,.
            \]
    \item Benyt denne gang formel (3) til at beregne en tilnærmelse til $\log
        1,06$, og sammenlign med den korrekte værdi udregnet på CAS.
    \item Hvor stor er den procentvise afvigelse denne gang?
\end{enumerate}

Skal man beregne logaritmer til tal, som ikke er tæt på 1, så er der andre metoder.
Vi ser først på situationen med tal tæt på $10$:

\begin{enumerate}[label=(\alph*), resume]
    \item Vis, at $\log x = \log\frac{x}{10} + 1$.
    \item Vis tilsvarende, at 
        \[
            \tag{4}
            \log x = \log\frac{x}{10^n} + n,\quad\mbox{ for alle $n$}\,.
        \]

\end{enumerate}
Således kan vi kombinere (1) og (4) til at finde $\log 1003 = \log 1,003 + 3
\approx 0,4343\cdot 0,003 + 3 = 3,0013$.
\begin{enumerate}[label=(\alph*), resume]
    \item Benyt ovenstående metode til at udregne $\log 102$.
    \item Hvor stor er den procentvise afvigelse ?
\end{enumerate}

Vi ser dernæst på situationen med tal tæt på $\sqrt{10} = 3,1623$, Her kan man
bruge en anden af regnereglerne for logaritmer:
\begin{enumerate}[label=(\alph*), resume]
    \item Vis, at $\log x = \frac12 \cdot \log x^2$.
    \item Vis tilsvarende, at
        \[
            \tag{5}
            \log x = \frac1n \cdot \log x^n,\quad \mbox{for alle $n$}\,.
        \]
\end{enumerate}
Således kan vi finde $\log 3,2 = \frac12 \log 3,2^2 = \frac12 \log 10,24
 = \frac12 (1 + \log 1,024) \approx \frac12 (1 + 0,4343\cdot 0,024)
 = 0,5052$.
\begin{enumerate}[label=(\alph*), resume]
    \item Benyt ovenstående metode til at udregne $\log 3,3$.
    \item Hvor stor er den procentvise afvigelse ?
\end{enumerate}

Skal man beregne logaritmer af andre tal, så opløfter man tallet i en passende
potens, således at resultatet er tæt på en potens af ti. Som eksempel ser vi på
$\log 7,2$. Ved at opløfte i forskellige potenser kan vi danne følgende tabel:

\begin{tabular}{|c|c|}
    \hline
    $n$ & $7,2^n$ \\
    \hline
    1 & 7,2 \\
    \hline
    2 & 51,84 \\
    \hline
    3 & 373,248 \\
    \hline
    4 & 2687,39 \\
    \hline
    5 & 19349,2 \\
    \hline
    6 & 139314 \\
    \hline
    7 & 1003061 \\
    \hline
\end{tabular}

Ved potensen syv fås altså et resultat, som kun er lidt større end en million.
Det kan vi bruge i følgende udregning:
\[
    \log 7,2 = \frac17 \log 1003061 = \frac17(6+\log1,003061)
    \approx \frac17(6+0,4343\cdot0,003061) = 0,8573 \,.
\]

\begin{enumerate}[label=(\alph*), resume]
    \item Benyt ovenstående metode til at udregne $\log 1,47$.
    \item Hvor stor er den procentvise afvigelse ?
    \item Benyt ovenstående metode til at udregne $\log 6,3$.
    \item Hvor stor er den procentvise afvigelse ?
\end{enumerate}

\end{document}

