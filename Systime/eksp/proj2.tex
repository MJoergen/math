Projekt: Kulstof-14 datering.

Dette projekt omhandler en snedig metode, som arkæologer bruger til at bestemme
alderen af arkæologiske fund.

I alle levende organismer er der en naturlig forekomst af kulstof-14, som er en
radioaktiv carbon-isotop. Andelen af det radioaktive kulstof-14 i dag er
nogenlunde konstant ca. 10^-12 af alle kulstof-atomerne.

Når en biologisk organisme dør, så henfalder kulstuf-14 atomerne med en
halveringstid T_{1/2} på 5730 år.

Andelen af kulstof-14 atomer i en organisme følger derfor en aftagende
eksponentialfunktion.

(a) Vis, at formlen for halveringskonstanten kan omskrives til:
log(a) = log(1/2) / T_{1/2}.

(b) Vis, at denne ligning kan omskrives til:
a = (1/2) ^ {1/T_{1/2}}.

(c) Beregn en talværdi for fremskrivningsfaktoren a.

(d) Opskrift en forskrift, der angiver andelen af kulstof-14 atomer som
funktion af antal år efter organismens død.

(e) Beregn andelen af kulstof-14 atomer i en organisme, som er død for 2000 år
siden.

Ved anvendelser af kulstof-14 metoden måler man med massespektroskopi andelen
af kulstof-14 atomer.

(f) I en organisme finder man andelen af kulstof-14 atomer til at være
10^{-13}. Beregn antallet af år siden organismens død.

