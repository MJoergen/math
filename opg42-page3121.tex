\documentclass[12pt,oneside,a4paper]{article}

\usepackage[utf8]{inputenc} % Lærer LaTeX at forstå unicode - HUSK at filen skal
% være unicode (UTF-8), standard i Linux, ikke i
% Win.

\usepackage[danish]{babel} % Så der fx står Figur og ikke Figure, Resumé og ikke
% Abstract etc. (god at have).

%\renewcommand{\mid}[1]{{\rm E}\!\left[#1\right]}
\newcommand{\bas}{\begin{eqnarray*}}
\newcommand{\eas}{\end{eqnarray*}}

\begin{document}

\section{Opgave 42}
Løs (uden hjælpemidler) ligningen
$$
x^7+3x^6+2x^5+4x^4+4x^3+2x^2+3x+1=0
$$

\section{Løsning}
Vi ser ved indsættelse, at $x=-1$ er en løsning.
Polynomiers division giver da følgende ligning:
$$
x^6+2x^5+4x^3+2x+1=0
$$

På grund af symmetrien i koefficienterne, så ser vi straks, at hvis $x$ er en løsning, da er også $\frac 1x$ en løsning.
Derfor dividerer vi med $x^3$ så det giver:
$$
x^3+\frac {1}{x^3} + 2 \left(x^2+\frac{1}{x^2}\right) + 4 = 0
$$
Vi laver nu substitutionen
$$
t=x+\frac 1x
$$
og udregner
$$
t^2 = x^2+\frac{1}{x^2} + 2
$$
og
$$
t^3 = x^3+\frac {1}{x^3} + 3\left(x+\frac 1x\right)
$$
Dermed kan ligningen reduceres til
$$
t^3-3t+2t^2=0
$$
Da $t\neq 0$ så kan vi dividere med $t$:
$$
t^2+2t-3=0
$$
Løsningerne er $t=1$ og $t=-3$.

Vi undersøger de to mulige værdier af $t$ hver for sig. 
Substitutionen for $t$ omskrives til:
$$
x^2-tx+1=0
$$
Hvis $t=1$ så giver det
$$
x^2-x+1=0
$$
som ikke har nogen reelle løsninger.
Dernæst hvis $t=-3$ så har vi
$$
x^2+3x+1=0
$$
Her er løsningerne
$$
x=\frac{-3\pm\sqrt 5}{2}
$$
Vi har således fundet i alt tre reelle løsninger.

\end{document}

