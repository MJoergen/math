\documentclass[12pt,oneside,a4paper]{article}

\usepackage[utf8]{inputenc} % Lærer LaTeX at forstå unicode - HUSK at filen skal
% være unicode (UTF-8), standard i Linux, ikke i
% Win.

\usepackage[danish]{babel} % Så der fx står Figur og ikke Figure, Resumé og ikke
% Abstract etc. (god at have).

%\renewcommand{\mid}[1]{{\rm E}\!\left[#1\right]}
\newcommand{\bas}{\begin{eqnarray*}}
\newcommand{\eas}{\end{eqnarray*}}

\begin{document}

\section{Geometriopgave 100}

\section{Løsning}
Uden tab af generalitet kan vi sætte $a=1$.

Vi indfører et koordinatsystem med origo i punktet $A$ og med $x$-aksen i retning mod punktet $C$. Så har punket $P$ koordinaterne $(\frac 34, \frac 34)$ og den lille cirkel har radius $r=\frac 14$.

Tangenten $DQ$ har ligningen $-hx+y=0$, hvor $h>1$ er (den endnu ukendte) hældning af linjen $DQ$. Afstanden fra punktet $P$ til denne linje skal derfor være ilg med radius. Med formlen for afstand fra punkt til linje giver det følgende ligning:
$$
\frac{\left|-hP_x+P_y\right|}{\sqrt{h^2+1}} = r
$$

Vi indsætter nu de kendte tal og det giver:
$$
\frac 34 \frac{h-1}{\sqrt{h^2+1}} = \frac 14
$$
Dette omskrives til følgende andengradsligning:
$$
4h^2-9h+4=0
$$

Punktet $Q$ vil have koordinaterne $(\frac 1h, 1)$, og den søgte afstand $|DQ|$ bliver da:
$$
|DQ| = \sqrt{\frac{1}{h^2} + 1} = \frac{\sqrt{h^2+1}}{h}
$$
Ud fra andengradsligningen får vi
$$
h^2+1 = \frac 94 h
$$
Dermed kan den søgte afstand skrives som:
$$
|DQ| = \frac{3}{2} \frac{1}{\sqrt h}
$$

Ved at dividere andengradsligningen med $h$ får vi
$$
4h-9+\frac 4h=0
$$
Dermed har vi, at 
$$
\left(2\sqrt h - \frac{2}{\sqrt h}\right)^2 = 4h + \frac 4h - 8 = 1
$$
Da $h>1$ har vi altså
$$
2\sqrt h - \frac{2}{\sqrt h} = 1
$$

Heri sætter vi nu
$$
x= \frac{1}{\sqrt h}
$$
Det giver følgende nye andengradsligning (efter multiplikation med $x$):
$$
-2x^2-x+2=0
$$
Da vi må have $0<x<1$ så er løsningen givet ved:
$$
\frac{1}{\sqrt h} = x=\frac{\sqrt{17}-1}{4}
$$
Den søgte afstand er da:
$$
|DQ| = \frac{3}{8} \left(\sqrt{17}-1\right)
$$


\end{document}

\documentclass[12pt,oneside,a4paper]{article}

\usepackage[utf8]{inputenc} % Lærer LaTeX at forstå unicode - HUSK at filen skal
% være unicode (UTF-8), standard i Linux, ikke i
% Win.

\usepackage[danish]{babel} % Så der fx står Figur og ikke Figure, Resumé og ikke
% Abstract etc. (god at have).

%\renewcommand{\mid}[1]{{\rm E}\!\left[#1\right]}
\newcommand{\bas}{\begin{eqnarray*}}
\newcommand{\eas}{\end{eqnarray*}}

\begin{document}

\section{Opgave 24}
Det oplyses, at
\bas
p+q+r &=& 0 \\
a+b+c &=& 0 \\
\frac{p}{a} + \frac{q}{b} + \frac{r}{c} &=& 0
\eas
Vis, at
$$
pa^2+qb^2+rc^2=0
$$

\section{Løsning}
Betragt matricen $M$ defineret ved:
$$
M = \left(\begin{array}{ccc}
    1 & 1 & 1 \\
    \frac{1}{a} & \frac{1}{b} & \frac{1}{c} \\
    a^2 & b^2 & c^2 
\end{array}
\right)
$$
Determinanten af denne udregnes til
$$
\det(M) = \frac{b^2-c^2}{a} + \frac{c^2-a^2}{b} + \frac{a^2-b^2}{c}
$$

Antag nu yderligere, at $a+b+c=0$. Så kan denne determinant omskrives på følgende måde:
\bas
\det(M) &=& \frac{(b+c)(b-c)}{a} + \frac{(c+a)(c-a)}{b} + \frac{(a+b)(a-b)}{c} \\
        &=& \frac{-a(b-c)}{a} + \frac{-b(c-a)}{b} + \frac{-c(a-b)}{c} \\
        &=& -(b-c) - (c-a) - (a-b) \\
        &=& 0
\eas
Der gælder altså, at hvis $a+b+c=0$, så er de tre rækker i matricen $M$ lineært afhængige. Specielt kan den tredje række skrives som en linearkombination af de to første rækker.
Heraf følger, at hvis $p+q+r=0$ og $\frac{p}{a} + \frac{q}{b} + \frac{r}{c}=0$, så vil der også gælde, at $pa^2+qb^2+rc^2=0$.

Ovenstående udregning er tilstrækkelig, men den præcise linearkombination kan også findes. Vi ønsker altså at bestemme to tal $m$ og $n$, således at
\bas
a^2 &=& m+\frac{n}{a} \\
b^2 &=& m+\frac{n}{b} \\
c^2 &=& m+\frac{n}{c} 
\eas
Ved at trække de to første ligninger fra hinanden får man
$$
a^2-b^2 = n\left(\frac{1}{a}-\frac{1}{b}\right) = n \frac{b-a}{ab}
$$
Heraf følger, at 
$$
n = -ab(a+b)
$$
Under antagelsen $a+b+c=0$ giver det
$$
n=abc
$$
Herefter finder man, at
$$
m=a^2-bc
$$
Denne sidste ligning man omskrives til en symmetrisk form (igen under antagelse af $a+b+c=0$) ved at erstatte $a$ med $-(b+c)$. Det giver:
$$
m=(b+c)^2-bc = b^2+c^2+bc
$$
Vi har to forskellige udtryk for $m$. Ved at lægge dem sammen og dividere med to fås det symmetriske udtryk:
$$
m=\frac{1}{2} (a^2+b^2+c^2)
$$
Hermed er vi nået frem til følgende ligning:
$$
\frac{1}{2} (a^2+b^2+c^2) (p+q+r) + abc \left(\frac{p}{a} + \frac{q}{b} + \frac{r}{c} \right) = pa^2+qb^2+rc^2
$$
som gælder blot $a+b+c=0$.


\end{document}

\documentclass[12pt,oneside,a4paper]{article}

\usepackage[utf8]{inputenc} % Lærer LaTeX at forstå unicode - HUSK at filen skal
% være unicode (UTF-8), standard i Linux, ikke i
% Win.

\usepackage[danish]{babel} % Så der fx står Figur og ikke Figure, Resumé og ikke
% Abstract etc. (god at have).

%\renewcommand{\mid}[1]{{\rm E}\!\left[#1\right]}
\newcommand{\bas}{\begin{eqnarray*}}
\newcommand{\eas}{\end{eqnarray*}}

\begin{document}

\section{Opgave 25}
Løs ligningen
$$
\frac{x-a-b}{c} + \frac{x-b-c}{a} + \frac{x-c-a}{b} = 3,
$$
hvor $a$, $b$ og $c$ ikke er 0.

\section{Løsning}
Ved inspektion ses det, at $x=a+b+c$ opfylder ligningen.

Det bemærkes derefter, at ligningen er lineær i $x$, og dermed er der ikke andre løsninger.

\end{document}

\documentclass[12pt,oneside,a4paper]{article}

\usepackage[utf8]{inputenc} % Lærer LaTeX at forstå unicode - HUSK at filen skal
% være unicode (UTF-8), standard i Linux, ikke i
% Win.

\usepackage[danish]{babel} % Så der fx står Figur og ikke Figure, Resumé og ikke
% Abstract etc. (god at have).

%\renewcommand{\mid}[1]{{\rm E}\!\left[#1\right]}
\newcommand{\bas}{\begin{eqnarray*}}
\newcommand{\eas}{\end{eqnarray*}}

\begin{document}

\section{Opgave 27}
Vis, at hvis $5\le a\le 10$, vil
$$
\sqrt{a+3-4\sqrt{a-1}} + \sqrt{a+8-6\sqrt{a-1}} = 1
$$

\section{Løsning}
Vi ser først, at 
$$
\left(2-\sqrt{a-1}\right)^2 = a+3-4\sqrt{a-1}
$$
og
$$
\left(3-\sqrt{a-1}\right)^2 = a+8-6\sqrt{a-1}
$$

For $5\le a\le 10$ vil der gælde, at

\bas
2-\sqrt{a-1} &\le& 0 \\
3-\sqrt{a-1} &\ge& 0 
\eas
Dermed er
$$
\sqrt{a+3-4\sqrt{a-1}} = -(2-\sqrt{a-1}) 
$$
og 
$$
\sqrt{a+8-6\sqrt{a-1}} = 3-\sqrt{a-1} 
$$
Heraf følger resultatet.


\end{document}

\documentclass[12pt,oneside,a4paper]{article}

\usepackage[utf8]{inputenc} % Lærer LaTeX at forstå unicode - HUSK at filen skal
% være unicode (UTF-8), standard i Linux, ikke i
% Win.

\usepackage[danish]{babel} % Så der fx står Figur og ikke Figure, Resumé og ikke
% Abstract etc. (god at have).

%\renewcommand{\mid}[1]{{\rm E}\!\left[#1\right]}
\newcommand{\bas}{\begin{eqnarray*}}
\newcommand{\eas}{\end{eqnarray*}}

\begin{document}

\section{Opgave 28}
Skaf rationel nævner i brøken
$$
\frac{1}{\sqrt[3]9 + \sqrt[3]6 + \sqrt[3]4}
$$

\section{Løsning}
Vi benytter følgende identitet:
$$
(x+y+z) \left(xy+xz+yz - (x^2+y^2+z^2)\right) = 3xyz - (x^3+y^3+z^3)
$$
som kan verificeres ved at gange parenteserne ud.

Denne ligning omskrives til:
$$
\frac{1}{x+y+z} = \frac{xy+xz+yz - (x^2+y^2+z^2)}{3xyz - (x^3+y^3+z^3)}
$$

Heri skal vi nu indsætte $x=\sqrt[3]9$, $y=\sqrt[3]6$ og $z=\sqrt[3]4$. Vi får følgende mellemregninger:
\bas
xyz &=& \sqrt[3]{216} = 6\\
x^3+y^3+z^3 &=& 19 \\
xy &=& \sqrt[3]{54} = 3\sqrt[3]2 \\
xz &=& \sqrt[3]{36} \\
yz &=& \sqrt[3]{24} = 2\sqrt[3]3 \\
x^2 &=& \sqrt[3]{81} = 3\sqrt[3]3 \\
y^2 &=& \sqrt[3]{36} \\
z^2 &=& \sqrt[3]{16} = 2\sqrt[3]2
\eas
Det endelige resultat bliver
\bas
\frac{1}{\sqrt[3]9 + \sqrt[3]6 + \sqrt[3]4} 
&=& \frac{3\sqrt[3]2 + \sqrt[3]{36} + 2\sqrt[3]3 - (3\sqrt[3]3 + \sqrt[3]{36} + 2\sqrt[3]2)}{18-19} \\
&=& \sqrt[3]3 - \sqrt[3]2
\eas

\end{document}

\documentclass[12pt,oneside,a4paper]{article}

\usepackage[utf8]{inputenc} % Lærer LaTeX at forstå unicode - HUSK at filen skal
% være unicode (UTF-8), standard i Linux, ikke i
% Win.

\usepackage[danish]{babel} % Så der fx står Figur og ikke Figure, Resumé og ikke
% Abstract etc. (god at have).

%\renewcommand{\mid}[1]{{\rm E}\!\left[#1\right]}
\newcommand{\bas}{\begin{eqnarray*}}
\newcommand{\eas}{\end{eqnarray*}}

\begin{document}

\section{Opgave 29}
Vis, at der for positive tal $a$ og $b$ gælder, at
$$
\sqrt{2\left(\sqrt{a^2+b^2}-a\right)\left(\sqrt{a^2+b^2}-b\right)} = a+b-\sqrt{a^2+b^2}
$$

\section{Løsning}
Vi ganger parenteseren i radikanten ud:
\bas
&& 2\left(\sqrt{a^2+b^2}-a\right)\left(\sqrt{a^2+b^2}-b\right) \\
&=& 2\left(a^2+b^2-(a+b)\sqrt{a^2+b^2} + ab\right) \\
&=& (a+b)^2 + a^2 + b^2 - 2(a+b)\sqrt{a^2+b^2} \\
&=& \left(a+b - \sqrt{a^2+b^2}\right)^2
\eas

For positive tal gælder at
$$
(a+b)^2 > a^2+b^2
$$
og dermed
$$
a+b > \sqrt{a^2+b^2}
$$
Heraf følger det ønskede resultat.

\end{document}

\documentclass[12pt,oneside,a4paper]{article}

\usepackage[utf8]{inputenc} % Lærer LaTeX at forstå unicode - HUSK at filen skal
% være unicode (UTF-8), standard i Linux, ikke i
% Win.

\usepackage[danish]{babel} % Så der fx står Figur og ikke Figure, Resumé og ikke
% Abstract etc. (god at have).

%\renewcommand{\mid}[1]{{\rm E}\!\left[#1\right]}
\newcommand{\bas}{\begin{eqnarray*}}
\newcommand{\eas}{\end{eqnarray*}}

\begin{document}

\section{Opgave 31}
Find samtlige hele positive tal $n$, for hvilke $n^2+20n+15$ er et kvadrattal.

\section{Løsning}
Vi skal altså finde talpar $(m,n)$ således at 
$$
n^2+20n+15 = m^2
$$
Heraf følger specielt, at $m>n$.
Ligningen kan også omskrives til
$$
(n+10)^2-85 = m^2
$$
Heraf følger tilsvarende, at $m<n+10$.

Nu indfører vi tallet $k = m-n$. Så gælder der, at $0 < k < 10$. Indsættes nu $m=n+k$ i ligningen giver det:
$$
n^2+20n+15 = n^2+2nk + k^2
$$
Her kan $n$ isoleres:
$$
n = \frac{k^2-15}{2(10-k)}
$$
For at $n$ skal være et positivt heltal, så skal $k$ være et ulige tal større end 4 og mindre end 10. Det giver mulighederne $5$, $7$ og $9$.

Vi samler resultaterne i følgende tabel
$$
\begin{array}{cc}
    k & n \\
    \hline
    5 & 1 \\
    7 & \frac{17}{3} \\
    9 & 33
\end{array}
$$

Der er således to løsninger: $n=1$ og $n=33$.

\end{document}

\documentclass[12pt,oneside,a4paper]{article}

\usepackage[utf8]{inputenc} % Lærer LaTeX at forstå unicode - HUSK at filen skal
% være unicode (UTF-8), standard i Linux, ikke i
% Win.

\usepackage[danish]{babel} % Så der fx står Figur og ikke Figure, Resumé og ikke
% Abstract etc. (god at have).

%\renewcommand{\mid}[1]{{\rm E}\!\left[#1\right]}
\newcommand{\bas}{\begin{eqnarray*}}
\newcommand{\eas}{\end{eqnarray*}}

\begin{document}

\section{Opgave 33}
Bestem mindsteværdien af funktionen $f(x) = (x+1)(x+2)(x+3)(x+4) + 10$.

\section{Løsning}
Først substituerer vi $x=y-\frac 52$. Det giver følgende resultat:
\bas
f(x) &=& \left(y-\frac 32\right) \left(y-\frac 12\right) \left(y+\frac 12\right) \left(y+\frac 32\right) + 10 \\
     &=& \left(y^2 - \frac 14\right) \left(y^2 - \frac 94\right) + 10 \\
     &=& y^4 - \frac 52 y^2 + \frac{169}{16}
\eas
Nu foretager vi yderligere en substitution $z=y^2$ og vi får:
$$
f(x) = z^2 - \frac 52 z + \left(\frac{13}{4}\right)^2
$$

Vi ønsker altså at bestemme mindsteværdien af dette andengradspolynomium for $z\ge 0$.
Vi beregner først diskriminanten
$$
d=\left(\frac 52\right)^2 - 4 \left(\frac{13}{4}\right)^2 = -36
$$
Så finder vi koordinaterne for toppunktet:
$$
\left(\frac 54, 9 \right)
$$
Da første-koordinaten netop er positiv, så har vi vist, at mindsteværdien for $f(x)$ er $9$.


\end{document}

\documentclass[12pt,oneside,a4paper]{article}

\usepackage[utf8]{inputenc} % Lærer LaTeX at forstå unicode - HUSK at filen skal
% være unicode (UTF-8), standard i Linux, ikke i
% Win.

\usepackage[danish]{babel} % Så der fx står Figur og ikke Figure, Resumé og ikke
% Abstract etc. (god at have).

%\renewcommand{\mid}[1]{{\rm E}\!\left[#1\right]}
\newcommand{\bas}{\begin{eqnarray*}}
\newcommand{\eas}{\end{eqnarray*}}

\begin{document}

\section{Opgave 36}
Lad $a$ og $b$ være reelle tal, og sæt
\bas
p &=& a^3 -3ab^2 \\
q &=& b^3 -3a^2b
\eas
Udtryk $a^2+b^2$ ved $p$ og $q$.

\section{Løsning}
Først udregner vi
$$
p+q = a^3+b^3-3ab(a+b)
$$
og
$$
pq = ab\left(10a^2b^2-3(a^4+b^4)\right)
$$

Lad nu $z=a^2+b^2$. Vi vil udtrykke $p+q$ og $pq$ ved $z$.
Først har vi
$$
(a+b)^2 = z + 2ab
$$
Så får vi
\bas
p+q &=& (a+b)^3-6ab(a+b) \\
    &=& (a+b)(z-4ab)
\eas
og
\bas
pq &=& ab(10a^2b^2-3(z^2-2a^2b^2)) \\
   &=& -3abz^2+16(ab)^3
\eas

Nu udregner vi
\bas
(p+q)^2 &=& (z+2ab) (z-4ab)^2 \\
        &=& z^3-6abz^2+32(ab)^3 \\
        &=& z^3+2pq
\eas
Heraf ser vi, at 
$$
z^3 = p^2+q^2
$$
Altså er løsningen:
$$
a^2+b^2 = \sqrt[3]{p^2+q^2}
$$
\end{document}

\documentclass[12pt,oneside,a4paper]{article}

\usepackage[utf8]{inputenc} % Lærer LaTeX at forstå unicode - HUSK at filen skal
% være unicode (UTF-8), standard i Linux, ikke i
% Win.

\usepackage[danish]{babel} % Så der fx står Figur og ikke Figure, Resumé og ikke
% Abstract etc. (god at have).

%\renewcommand{\mid}[1]{{\rm E}\!\left[#1\right]}
\newcommand{\bas}{\begin{eqnarray*}}
\newcommand{\eas}{\end{eqnarray*}}

\begin{document}

\section{Opgave 37}
Opløs følgende udtryk i faktorer
$$
a(b+c)^2+b(c+a)^2+c(a+b)^2-4abc
$$

\section{Løsning}
Vi ganger parenteserne ud:
\bas
&=& a(b^2+c^2+2bc) + b(c^2+a^2+2ac) + c(a^2+b^2+2ab)-4abc \\
&=& ab(a+b)+ac(a+c)+bc(b+c)+2abc \\
&=& ab(a+b) + c\left(a(a+c)+b(b+c)+2ab\right) \\
&=& ab(a+b) + c(a+b)(a+b+c) \\
&=& (a+b)(ab+ca+cb+c^2) \\
&=& (a+b)(a+c)(b+c)
\eas

\end{document}

\documentclass[12pt,oneside,a4paper]{article}

\usepackage[utf8]{inputenc} % Lærer LaTeX at forstå unicode - HUSK at filen skal
% være unicode (UTF-8), standard i Linux, ikke i
% Win.

\usepackage[danish]{babel} % Så der fx står Figur og ikke Figure, Resumé og ikke
% Abstract etc. (god at have).

%\renewcommand{\mid}[1]{{\rm E}\!\left[#1\right]}
\newcommand{\bas}{\begin{eqnarray*}}
\newcommand{\eas}{\end{eqnarray*}}

\begin{document}

\section{Opgave 38}
Løs inden for de reelle tal ligningen
$$
x^{\sqrt x} = \left(\sqrt x\right)^x
$$

\section{Løsning}
Vi har $x>0$. Det ses umiddelbart, at $x=1$ er en løsning. Antag derfor i det følgende at $x\neq 1$. Tag nu logaritmen på begge sider:
$$
\sqrt x \log x = x \log\sqrt x = \frac x2 \log x
$$
Da $x\neq 1$ kan vi dividere med $\log x$ på begge sider:
$$
\sqrt x = \frac x2
$$
Det omskrives til
$$
2 = \sqrt x
$$
som har løsningen $x=4$.

\end{document}

\documentclass[12pt,oneside,a4paper]{article}

\usepackage[utf8]{inputenc} % Lærer LaTeX at forstå unicode - HUSK at filen skal
% være unicode (UTF-8), standard i Linux, ikke i
% Win.

\usepackage[danish]{babel} % Så der fx står Figur og ikke Figure, Resumé og ikke
% Abstract etc. (god at have).

%\renewcommand{\mid}[1]{{\rm E}\!\left[#1\right]}
\newcommand{\bas}{\begin{eqnarray*}}
\newcommand{\eas}{\end{eqnarray*}}

\begin{document}

\section{Opgave 40}
Løs inden for de reelle tal ligningssystemet
\bas
x+y &=& z \\
x^2+y^2 &=& z \\
x^3+y^3 &=& z
\eas

\section{Løsning}
Vi ser umiddelbart, at $z \ge 0$. Endvidere ser vi, at hvis $z=0$, så er der netop én løsning $(0, 0, 0)$.

Antag derfor nu, at $z \ne 0$. Så udregner vi
$$
z^2 = (x+y)^2 = x^2+y^2+2xy = z+2xy
$$
og 
$$
z^3 = (x+y)^3 = x^3+y^3+3xy(x+y) = z(1+3xy)
$$
Da $z\ne 0$ kan den sidste ligning forkortes med $z$. Det giver 
$$
z^2=1+3xy
$$
Isoleres nu $xy$ i de to ovenstående ligninger for $z^2$ får vi følgende ligning:
$$
xy = \frac{z^2-1}{3} = \frac{z^2-z}{2}
$$
Det sidste lighedstegn reducerer til følgende andengradsligning
$$
z^2-3z+2=0
$$
som har løsningerne $z=1$ og $z=2$.

Nu undersøger vi først $z=1$: Det giver $xy=0$, som sammen med $x+y=1$ giver to løsninger $(0, 1, 1)$ og $(1, 0, 1)$.

Nu undersøger vi dernæst $z=2$: Det giver $xy=1$, som sammen med $x+y=2$ giver én løsning $(1, 1, 2)$.

I alt er der fundet fire forskellige løsninger.

\end{document}

\documentclass[12pt,oneside,a4paper]{article}

\usepackage[utf8]{inputenc} % Lærer LaTeX at forstå unicode - HUSK at filen skal
% være unicode (UTF-8), standard i Linux, ikke i
% Win.

\usepackage[danish]{babel} % Så der fx står Figur og ikke Figure, Resumé og ikke
% Abstract etc. (god at have).

%\renewcommand{\mid}[1]{{\rm E}\!\left[#1\right]}
\newcommand{\bas}{\begin{eqnarray*}}
\newcommand{\eas}{\end{eqnarray*}}

\begin{document}

\section{Opgave 42}
Løs (uden hjælpemidler) ligningen
$$
x^7+3x^6+2x^5+4x^4+4x^3+2x^2+3x+1=0
$$

\section{Løsning}
Vi ser ved indsættelse, at $x=-1$ er en løsning.
Polynomiers division giver da følgende ligning:
$$
x^6+2x^5+4x^3+2x+1=0
$$

På grund af symmetrien i koefficienterne, så ser vi straks, at hvis $x$ er en løsning, da er også $\frac 1x$ en løsning.
Derfor dividerer vi med $x^3$ så det giver:
$$
x^3+\frac {1}{x^3} + 2 \left(x^2+\frac{1}{x^2}\right) + 4 = 0
$$
Vi laver nu substitutionen
$$
t=x+\frac 1x
$$
og udregner
$$
t^2 = x^2+\frac{1}{x^2} + 2
$$
og
$$
t^3 = x^3+\frac {1}{x^3} + 3\left(x+\frac 1x\right)
$$
Dermed kan ligningen reduceres til
$$
t^3-3t+2t^2=0
$$
Da $t\neq 0$ så kan vi dividere med $t$:
$$
t^2+2t-3=0
$$
Løsningerne er $t=1$ og $t=-3$.

Vi undersøger de to mulige værdier af $t$ hver for sig. 
Substitutionen for $t$ omskrives til:
$$
x^2-tx+1=0
$$
Hvis $t=1$ så giver det
$$
x^2-x+1=0
$$
som ikke har nogen reelle løsninger.
Dernæst hvis $t=-3$ så har vi
$$
x^2+3x+1=0
$$
Her er løsningerne
$$
x=\frac{-3\pm\sqrt 5}{2}
$$
Vi har således fundet i alt tre reelle løsninger.

\end{document}

