\documentclass[12pt,oneside,a4paper]{article}

\usepackage[utf8]{inputenc} % Lærer LaTeX at forstå unicode - HUSK at filen skal
% være unicode (UTF-8), standard i Linux, ikke i
% Win.

\usepackage[danish]{babel} % Så der fx står Figur og ikke Figure, Resumé og ikke
% Abstract etc. (god at have).

\usepackage{graphicx}
\usepackage{amsfonts}

%\renewcommand{\mid}[1]{{\rm E}\!\left[#1\right]}
\newcommand{\bas}{\begin{eqnarray*}}
\newcommand{\eas}{\end{eqnarray*}}

\begin{document}

\section*{Opgave 52}
Lad $f(x) = x^{44} + x^{33} + x^{22} + x^{11} + 1$ og $q(x) = x^4 + x^3 + x^2 +
x + 1$.  Vi skal vise, at $f(x)$ er deleligt med $q(x)$. Dvs vi skal vise, at
hvis $x_0$ er en rod i $q(x)$, da er $x_0$ også en rod i $f(x)$.

Lad derfor $x_0$ være en rod i $q(x)$, dvs. $q(x_0)=0$. 
Der gælder, at
$$
q(x) = \frac{x^5-1}{x-1}
$$
Derfor er $x_0$ en rod i $x^5-1$, men med $x_0 \neq 1$.
Dette kan skrives som $x_0\in M$, hvor
mængden $M$ er givet ved $M = \{\alpha, \alpha^2, \alpha^3, \alpha^4\}$, hvor
$\alpha=\exp\left(\frac{2\pi i}{5}\right)$.

Da $\alpha^{11} = \left(\alpha^5\right)^2 \cdot \alpha = \alpha$, så vil 
$x_0^{11} = x_0$ for alle $x_0\in M$.
Vi har endvidere, at $f(x) = q(x^{11})$, og dermed er $f(x_0) = q(x_0^{11}) =
q(x_0)$ = 0.

Hermed er det vist, at $f(x)$ er deleligt med $q(x)$.

\section*{Opgave 53}
Lad $a = \sqrt[3]{60}$ og $b=2 + \sqrt[3]{7}$. Vi skal afgøre, hvilket af disse
to tal, der er størst.  Vi udregner derfor først $a^3$ og $b^3$ og derefter
trækker vi dem fra hinanden. Det giver:
\bas
\frac{b^3-a^3}{3} &=& -15 + 4\sqrt[3]{7} + 2 \left(\sqrt[3]{7}\right)^2 \\
                  &=& f\left(\sqrt[3]{7}\right),
\eas
hvor $f(x) = 2x^2+4x-15$. Vi skal således afgøre fortegnet af $f(x)$ for $x =
\sqrt[3]{7}$.

Rødderne for $f(x)$ findes til $x = -1 \pm \frac12 \sqrt{34}$. Definer nu:
$$
x_0 = -1 + \frac12\sqrt{34}.
$$
Da $f(x)$ er voksende i omegnen af $x=x_0$ så gælder der, at hvis $x>x_0$ så
er $f(x) > f(x_0)$. Vi skal derfor sammenligne $x_0$ med $\sqrt[3]{7}$.
Vi udregner derfor
$$
x_0^3-7 = \frac{23}{4}\sqrt{34} - \frac{67}{2}
$$

Lad nu $A = \frac{23}{4}\sqrt{34}$ og $B = \frac{67}{2}$. Vi skal undersøge fortegnet af $A-B$. Vi udregner derfor først
$$
\frac{A}{B} = \frac{23\sqrt{34}}{134}
$$
og dernæst
$$
\frac{A^2}{B^2} = \frac{23^2 \cdot 34}{134^2} = \frac{17986}{17956}.
$$
Derfor er $A > B$. Derfor er $\sqrt[3]{7} < x_0$. Derfor er
$f\left(\sqrt[3]{7}\right) < f(x_0) = 0$. Derfor er $b < a$. Det er således beviset, at
$$
2 + \sqrt[3]{7} < \sqrt[3]{60}.
$$

\section*{Opgave 60}
Lad $f(x) = \left(x^2+3x+2\right)\cdot\left(x^2-7x+12\right)\cdot\left(x^2-2x-1\right)-24$.
Vi skal bestemme samtlige rødder til dette sjettegradspolynomium. De to første faktorer kan faktoriseres til:
$$
x^2+3x+2 = (x+1)(x+2)
$$
og
$$
x^2-7x+12 = (x-3)(x-4) = (3-x)(4-x)
$$
Den sidste faktor kan skrives som:
$$
x^2-2x-1 = (x-1)^2 - 2 = (1-x)^2-2
$$
Heraf ses, at substitutionen $x \rightarrow 2-x$ lader rødderne uændret. Med andre ord:
$$
f(x) = f(2-x)
$$

På denne baggrund indfører vi nu substitutionen $x = 1+u$.
\bas
f(1+u) &=& \left(u^2+5u+6\right)\cdot\left(u^2-5u+6\right)\cdot\left(u^2-2\right)+24 \\
       &=& \left(\left(u^2+6\right)^2 - \left(5u\right)^2\right)\cdot\left(u^2-2\right)+24 \\
       &=& \left(u^4-13u^2+36\right)\cdot\left(u^2-2\right)+24
\eas
Nu indfører vi substitutionen $u^2=w$, hvor $w\ge 0$
\bas
f(1+u) &=& \left(w^2-13w+36\right)\cdot(w-2)+24\\
       &=& w^3-15w^2+62w-48
\eas

Definér nu
$$
g(w) = w^3-15w^2+62w-48
$$
Vi ser, at $w=1$ er en rod. Vi indfører derfor substitutionen $w=1+y$ og får:
$$
g(1+y) = y\cdot\left(y^2-12y+35\right)
$$
Heraf får vi, at rødderne i $g(1+y)$ er $y=0$ og $y=5$ og $y=7$.
Rødderne i $g(w)$ er derfor $\{1, 6, 8\}$.

Dermed bliver samtlige rødder i $f(x)$ givet ved $\{0, 2, 1\pm\sqrt{6}, 1\pm\sqrt{8}\}$.

\end{document}

