\documentclass[12pt,oneside,a4paper]{article}

\usepackage[utf8]{inputenc} % Lærer LaTeX at forstå unicode - HUSK at filen skal
% være unicode (UTF-8), standard i Linux, ikke i
% Win.

%\usepackage[danish]{babel} % Så der fx står Figur og ikke Figure, Resumé og ikke
% Abstract etc. (god at have).

%\usepackage{graphicx}
\usepackage{amsfonts}
\usepackage{amsthm}        % Theorems
\usepackage{amsmath}
%\usepackage{hyperref}

%\renewcommand{\mid}[1]{{\rm E}\!\left[#1\right]}
\newcommand{\bas}{\begin{eqnarray*}}
\newcommand{\eas}{\end{eqnarray*}}
\newcommand{\be}{\begin{equation}}
\newcommand{\ee}{\end{equation}}
\newcommand{\bea}{\begin{eqnarray}}
\newcommand{\eea}{\end{eqnarray}}

%\newtheorem{thm}{Sætning}[section]
%\newtheorem{mydef}[thm]{Definition}
%\newtheorem{eks}[thm]{Eksempel}

\title{The Pendulum Equation}
\date{May 2022}
\author{Michael Jørgensen}

\begin{document}

\maketitle

We start with the pendulum equation
\be
\ddot y + \sin y = 0,
\label{eq1}
\ee
where $y=y(t)$ is a real function of time, and $\dot y = \frac{dy}{dt}$
indicates derivative with respect to $t$.

We transform this equation by first noting that we can reduce the order of the
equation. This is done by multiplying with $\dot y$ and then integrating.
This gives
\be
\frac12 (\dot y)^2 - \cos y = K,
\label{eq2}
\ee
where $K$ is an integrating factor.

To proceed we apply another trick, which is to differentiate
equation~(\ref{eq1}).  This gives:
\be
\dddot y + \cos y \,\dot y = 0.
\label{eq3}
\ee
We may now eliminate $\cos y$ by multiplying equation~(\ref{eq2}) by $\dot y$
and adding equation~(\ref{eq3}). This gives:
\be
\dddot y + \frac12 (\dot y)^3 = K\dot y.
\ee

Finally we apply the substitution $z=\dot y$ to get:
\be
\ddot z + \frac12 z^3 - Kz = 0.
\ee


\end{document}


