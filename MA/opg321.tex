\documentclass[12pt,oneside,a4paper]{article}

\usepackage[utf8]{inputenc} % Lærer LaTeX at forstå unicode - HUSK at filen skal
% være unicode (UTF-8), standard i Linux, ikke i
% Win.

\usepackage[danish]{babel} % Så der fx står Figur og ikke Figure, Resumé og ikke
% Abstract etc. (god at have).

\usepackage{graphicx}
\usepackage{amsfonts}

%\renewcommand{\mid}[1]{{\rm E}\!\left[#1\right]}
\newcommand{\bas}{\begin{eqnarray*}}
\newcommand{\eas}{\end{eqnarray*}}

\begin{document}

\section*{Løsning til Opgavehjørnet 321}

\subsection*{a}
Vi skal finde alle reelle tal $x$ som opfylder ligningen:
$$
\sqrt[3]{x+1} + \sqrt[3]{3x+1} = \sqrt[3]{x-1}
$$
Først opløfter vi i tredje potens:
$$
(x+1) + (3x+1) + 3\sqrt[3]{x+1}\sqrt[3]{3x+1}\left(\sqrt[3]{x+1} + \sqrt[3]{3x+1}\right) = x-1
$$
Ved at benytte den oprindelige ligning kan dette omskrives til:
$$
3x + 3 + 3\sqrt[3]{(x+1)(3x+1)(x-1)} = 0
$$
Ved at dividere med $3$ og rykke rundt får vi
$$
x + 1 = \sqrt[3]{(x+1)(3x+1)(1-x)}
$$
Ved igen at opløfte i tredje får vi:
$$
(x+1)^3 = (x+1)(3x+1)(1-x)
$$
Heraf ser vi, at $x=-1$ er en mulig løsning. Ved indsættelse i den oprindelige ligning ses løsningen at passe.

I det følgende antager vi, at $x\neq -1$. Ved division med $x+1$ får vi:
$$
(x+1)^2 + (3x+1)(x-1) = 0
$$
som reducerer til
$$
4x^2=0
$$
Desværre er dette ikke en løsning til den oprindelige ligning. Vi har derfor fundet, at der kun er én løsning, nemlig $x=-1$.

\subsection*{b}
Vi skal finde samtlige komplekse løsninger $w\neq z$ til nedenstående lignignssystem:
\bas
w^5 + w &=& z^5 + z \\
w^5 + w^2 &=& z^5 + z^2
\eas
Den første ligning trækkes fra den anden, så det giver:
$$
w^2 - w = z^2 - z
$$
Dette omskrives til
$$
(w+z)(w-z) = w-z
$$
Idet vi har $w\neq z$ kan vi dividere med $w-z$:
$$
w+z=1
$$

For at komme videre substituerer vi
\bas
w &=& \frac12 + u \\
z &=& \frac12 - u ,
\eas
hvor $u\neq 0$.
Indsættelse i den første ligning giver:
$$
\left(u+\frac12\right)^5 + \left(u-\frac12\right)^5+2u=0
$$
Vi ganger parenteserne ud og får:
$$
2u^5 + 5u^3 + \frac{21}{8}u = 0
$$
Efter division med $u$ giver det:
$$
2u^4 + 5u^2 + \frac{21}{8} = 0
$$
Den sædvanlige løsningsformel giver to løsninger:
\bas
u^2 &=& -\frac{3}{4}\\
u^2 &=& -\frac{7}{4}
\eas

Nu kan vi opskrive løsningnerne til de oprindelige ligninger:
\bas
w,z &=& \frac12 \pm \frac12 \sqrt{3} i \\
w,z &=& \frac12 \pm \frac12 \sqrt{7} i
\eas

\end{document}


