\documentclass[12pt,oneside,a4paper]{article}

\usepackage[utf8]{inputenc} % Lærer LaTeX at forstå unicode - HUSK at filen skal
% være unicode (UTF-8), standard i Linux, ikke i
% Win.

\usepackage[danish]{babel} % Så der fx står Figur og ikke Figure, Resumé og ikke
% Abstract etc. (god at have).

\usepackage{graphicx}
\usepackage{amsfonts}
\usepackage{esvect}
\usepackage{hyperref}

%\renewcommand{\mid}[1]{{\rm E}\!\left[#1\right]}
\newcommand{\bas}{\begin{eqnarray*}}
\newcommand{\eas}{\end{eqnarray*}}

\begin{document}

\section*{Opgave 19 -- side 3278}
Vi skal vise, at
$$
\cos\left(2\arctan\frac17\right) = \sin\left(4\arctan\frac13\right)
$$

\section*{Løsning}
Vi udregner begge sider hver for sig ved hjælp af komplekse tal. Betragt først
det komplekse enhedstal
$$
z = \frac{7+i}{\sqrt{50}} = e^{i\phi}
$$
hvor $\phi = \arctan\frac17$. Så har vi, at 
\bas
\cos\left(2\arctan\frac17\right) &=&
\cos(2\phi) \\
&=& \Re(z^2) \\
&=& \Re\left(\frac{48+14i}{50}\right) \\
&=& \frac{24}{25}
\eas

Betragt dernæst det komplekse enhedstal
$$
w = \frac{3+i}{\sqrt{10}} = e^{i\psi}
$$
hvor $\psi = \arctan\frac13$. Så har vi, at
\bas
\sin\left(4\arctan\frac13\right) &=& \sin(4\psi) \\
                                 &=& \Im(w^4) \\
                                 &=& \Im\left(\frac{28+96i}{100}\right) \\
                                 &=& \frac{24}{25}
\eas
Hermed er det ønskede vist.

Som tillæg kan nævnes, at relationen også viser, at 
$$
\phi = \frac{\pi}{2} - \psi
$$
eller med andre ord:
$$
2\arctan\frac17 + 4\arctan\frac13 = \frac{\pi}{2}
$$
Der findes mange relationer af denne form, et andet eksempel er:
$$
4\arctan\frac15 - \arctan\frac{1}{239} = \frac{\pi}{4}
$$
Se mere på \url{https://en.wikipedia.org/wiki/Machin-like_formula}


\end{document}


