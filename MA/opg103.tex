\documentclass[12pt,oneside,a4paper]{article}

\usepackage[utf8]{inputenc} % Lærer LaTeX at forstå unicode - HUSK at filen skal
% være unicode (UTF-8), standard i Linux, ikke i
% Win.

\usepackage[danish]{babel} % Så der fx står Figur og ikke Figure, Resumé og ikke
% Abstract etc. (god at have).

\usepackage{graphicx}
\usepackage{amsfonts}

%\renewcommand{\mid}[1]{{\rm E}\!\left[#1\right]}
\newcommand{\bas}{\begin{eqnarray*}}
\newcommand{\eas}{\end{eqnarray*}}

\begin{document}

\section*{Opgave 103}

I $\Delta ABP$ er $A=45^\circ$ og $B=15^\circ$. Siden $AP$ forlænges ud over
$P$ til $C$, så $PC = 2\cdot AP$. Vi skal vise, at $\angle ACB = 75^\circ$.

Vi har umiddelbart, at $\angle APB = 120^\circ$.  Sinusrelationerne i $\Delta
ABP$ giver derefter:
$$
AB = AP \cdot \frac{\sin 120^\circ}{\sin 15^\circ}
$$
Kald nu den søgte vinkel $\angle ACB$ for $y$.  Da er $\angle ABC = 135^\circ -
y$.  Så giver sinusrelationerne i
$\Delta ABC$:
$$
AB = AC \cdot \frac{\sin (y)}{\sin(135^\circ-y)}
$$
Kombineres disse to ligninger, sammen med $AC = 3 \cdot AP$ giver det
$$
\frac{\sin 120^\circ}{\sin 15^\circ} = 3 \cdot \frac{\sin (y)}{\sin(135^\circ-y)}
$$
som kan omskrives til:
$$
\sin 120^\circ \cdot \sin(135^\circ - y) = 3 \cdot \sin 15^\circ \cdot \sin(y)
$$
Denne ligning skal nu løses.

Additionsformlerne giver:
$$
\sin 120^\circ \cdot \left(\sin 135^\circ \cdot \cos (y) - \cos 135^\circ \cdot \sin
(y)\right) = 3 \cdot \sin 15^\circ \cdot \sin(y)
$$
som reduceres til:
$$
\sin 60^\circ \cdot \sin 45^\circ \cdot (\cos(y) + \sin(y)) = 3 \cdot \sin 15^\circ \cdot \sin(y)
$$
Efter nogle flere omskrivninger kan det skrives som:
$$
\tan(y) = \frac{\sin 60^\circ \cdot \sin 45^\circ}{3\cdot\sin15^\circ -
\sin 60^\circ \cdot \sin 45^\circ}
$$

Additionsformlerne bruges igen, denne gang til at omskrive:
$$
\sin 60^\circ = \sin(45^\circ + 15^\circ) = \sin 45^\circ \cdot (
\cos 15^\circ + \sin 15^\circ)
$$
Ved dernæst at benytte, at $2\sin^2 45^\circ = 1$ reducerer ligningen til:
$$
\tan(y) = \frac{\sin 15^\circ + \cos 15^\circ}{5\cdot\sin 15^\circ - \cos
15^\circ}
$$
Der forkortes med $\cos 15^\circ$:
$$
\tan(y) = \frac{\tan 15^\circ + 1}{5 \cdot \tan 15^\circ - 1}
$$

Heri indsættes $\tan 15^\circ = 2 - \sqrt{3}$. Det giver
\bas
\tan(y) &=& \frac{3-\sqrt{3}}{9-5\sqrt{3}} \\
        &=& \frac{(3-\sqrt{3})(9+5\sqrt{3})}{81-25\cdot 3} \\
        &=& 2 + \sqrt{3}
\eas
Dette sidste genkender vi som $\tan 75^\circ$. Vi har dermed vist, at
$\angle ACB = 75^\circ$.




\end{document}

