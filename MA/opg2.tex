\documentclass[12pt,oneside,a4paper]{article}

\usepackage[utf8]{inputenc} % Lærer LaTeX at forstå unicode - HUSK at filen skal
% være unicode (UTF-8), standard i Linux, ikke i
% Win.

\usepackage[danish]{babel} % Så der fx står Figur og ikke Figure, Resumé og ikke
% Abstract etc. (god at have).

\usepackage{graphicx}
\usepackage{amsfonts}
\usepackage{hyperref}

%\renewcommand{\mid}[1]{{\rm E}\!\left[#1\right]}
\newcommand{\bas}{\begin{eqnarray*}}
\newcommand{\eas}{\end{eqnarray*}}
\newcommand{\be}{\begin{equation}}
\newcommand{\ee}{\end{equation}}

\begin{document}

Alle opgaverne i dette dokument er omskrivninger af resultater fra dokumentet 
\url{https://www.tjhsst.edu/~2010bhamrick/files/dumbassing.pdf}


\section{Opgave 2a}
Bevis følgende ulighed:
$$
a^3+b^3 \ge ab(a+b)
$$
hvor $a$ og $b$ er ikke-negative reelle tal.

\subsection{Løsning}

Vi benytter først den aritmetiske-geometriske ulighed på tallene $a^3$, $a^3$ og $b^3$ til at få:
$$
\frac{a^3+a^3+b^3}{3} \ge \sqrt[3]{a^3 a^3 b^3}
$$
Dette reducerer til
$$
\frac{2a^3+b^3}{3} \ge a^2b
$$
Ved at ombytte $a$ og $b$ og lægge til, så giver det
$$
a^3+b^3 \ge a^2b + ab^2
$$
hvilket er det ønskede.

\section{Opgave 2b}
Bevis følgende ulighed:
$$
2(a^3+b^3+c^3) + 3abc \ge (a^2+b^2+c^2)(a+b+c)
$$
hvor $a$, $b$ og $c$ er ikke-negative reelle tal.

\subsection{Løsning}

Vi trækker højresiden fra venstresiden, så vi i stedet for skal vise
$$
2(a^3+b^3+c^3) + 3abc - (a^2+b^2+c^2)(a+b+c) \ge 0
$$
Nu udnytter vi følgende identitet:
$$
(a^2+b^2+c^2)(a+b+c) = a^3+b^3+c^3 + a^2(b+c) + b^2(a+c) + c^2(a+b)
$$
til at omskrive venstresiden i uligheden:
\bas
&& 2(a^3+b^3+c^3) + 3abc - (a^2+b^2+c^2)(a+b+c) \\
&=& a^3+b^3+c^3 + 3abc - a^2(b+c) - b^2(a+c) - c^2(a+b) \\
&=& a\left(a^2-a(b+c)+bc\right) + b\left(b^2-b(a+c)+ac\right) + c\left(c^2-c(a+b)+ab\right) \\
&=& a(a-b)(a-c) + b(b-a)(b-c) + c(c-a)(c-b) \\
&=& (a-b)\Big(a(a-c)-b(b-c)\Big) + c(a-c)(b-c)
\eas
Vi skal nu vise at dette udtryk aldrig kan være negativt.  Da udtrykket er
symmetrisk ved permutation af de tre variabler, så kan vi uden tab af
generalitet antage, at $a \ge b \ge c$ .  Med denne antagelse er det tydeligt,
at begge led er ikke-negative, og vi har vist det ønskede.
Dette er et specialtilfælde af \href{https://en.wikipedia.org/wiki/Schur\%27s\_inequality}{Schur's ulighed}.

\section{Opgave 2c}
Bevis følgende ulighed:
$$
a^4+b^4+c^4+2abc(a+b+c) \ge (ab+ac+bc)(a^2+b^2+c^2)
$$
hvor $a$, $b$ og $c$ alle er vilkårlige reelle tal.

\subsection{Løsning}

Vi omskriver højresiden:
$$
(ab+ac+bc)(a^2+b^2+c^2) = ab(a^2+b^2) + ac(a^2+c^2) + bc(b^2+c^2) + abc(a+b+c)
$$
Idet vi flytter alle led over på venstresiden, så kan uligheden skrives som:
$$
a^4+b^4+c^4+abc(a+b+c) - ab(a^2+b^2) - ac(a^2+c^2) - bc(b^2+c^2) \ge 0
$$

Vi laver nu følgende omskrivninger af venstresiden i denne ulighed:
\bas
&& a^4+b^4+c^4+abc(a+b+c) - ab(a^2+b^2) - ac(a^2+c^2) - bc(b^2+c^2) \\
&=& a^4+a^2bc-a^3b-a^3c + b^4+b^2ac-b^3a-b^3c + c^4+c^2ab-c^3a-c^3b \\
&=& a^2(a-b)(a-c) + b^2(b-a)(b-c) + c^2(c-a)(c-b) \\
&=& (a-b)\left(a^2(a-c)-b^2(b-c)\right) + c^2(a-c)(b-c) 
\eas

Vi skal nu vise at dette udtryk aldrig kan være negativt.  Da udtrykket er
symmetrisk ved permutation af de tre variabler, så kan vi uden tab af
generalitet antage, at $a \ge b \ge c$ .  Med denne antagelse er det tydeligt,
at begge led er ikke-negative, og vi har vist det ønskede.
Dette er et specialtilfælde af \href{https://en.wikipedia.org/wiki/Schur\%27s\_inequality}{Schur's ulighed}.

\section{Opgave 2d}
Bevis følgende ulighed:
$$
\frac{bc}{2a+b+c} + \frac{ac}{a+2b+c} + \frac{ab}{a+b+2c} \le \frac 14 (a+b+c)
$$
hvor $a$, $b$ og $c$ er ikke-negative reelle tal.

\subsection{Løsning}
Ved at gange uligheden med $4(2a+b+c)(a+2b+c)(a+b+2c)$ og flytte alle led over på højresiden,
så får vi følgende ækvivalente ulighed:
\bas
0 &\le& (2a+b+c)(a+2b+c)(a+b+2c)(a+b+c) - 4bc(a+2b+c)(a+b+2c) \\
  && - 4ac(2a+b+c)(a+b+2c) - 4ab(2a+b+c)(a+2b+c) 
\eas
Vi vil nu vise, at højresiden i denne ulighed er ikke-negativ.  
Højresiden ganges ud og kan -- efter en del mellemregninger -- skrives som
\bas
&&2(a^4+b^4+c^4) + 2abc(a+b+c) \\
&&+ ab(a^2+b^2)+ac(a^2+c^2)+bc(b^2+c^2) - 6a^2b^2-6a^2c^2-6b^2c^2 
\eas

Fra opgave 2c har vi vist, at
\be
a^4+b^4+c^4+abc(a+b+c) - ab(a^2+b^2) - ac(a^2+c^2) - bc(b^2+c^2)  \ge 0
\label{eq1}
\ee
Endvidere er det tydeligt, at 
$$
ab(a-b)^2 + ac(a-c)^2+bc(b-c)^2 \ge 0
$$
hvilket kan omskrives til
\be
ab(a^2+b^2) + ac(a^2+c^2) +bc(b^2+c^2) -2a^2b^2 -2a^2c^2 -2b^2c^2 \ge 0
\label{eq2}
\ee

Ved nu at lægge 2 gange uligheden~(\ref{eq1}) sammen med 3 gange uligheden~(\ref{eq2}) så får vi det ønskede resultat.

\end{document}


