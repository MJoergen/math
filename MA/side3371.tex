\documentclass[12pt,oneside,a4paper]{article}

\usepackage[utf8]{inputenc} % Lærer LaTeX at forstå unicode - HUSK at filen skal
% være unicode (UTF-8), standard i Linux, ikke i
% Win.

\usepackage[danish]{babel} % Så der fx står Figur og ikke Figure, Resumé og ikke
% Abstract etc. (god at have).

\usepackage{graphicx}
\usepackage{amsfonts}

%\renewcommand{\mid}[1]{{\rm E}\!\left[#1\right]}
\newcommand{\bas}{\begin{eqnarray*}}
\newcommand{\eas}{\end{eqnarray*}}

\begin{document}

\section*{Geometriopgave 159}
I en halvcirkel med centrum $O$ og radius $\sqrt{50}$ er $AB$ diameter. $D$ 
er et punkt i cirklens indre og $C$ og $E$ ligger på cirklen, så $CD \perp DE$, $CD \parallel AB$ og $CD=6$, $DE=2$. Bestem $OD$.


\section*{Løsning}
\begin{figure}[ht]
    \centering
    \includegraphics{side3371fig}
    \label{side3371fig}
\end{figure}

Siden $CD$ forlænges ud over $D$, så den skærer cirklen i $F$. Cirklen
omskriver trekanten $CEF$ og arealet af trekant $CEF$ kan derfor skrives på to
måder:
$$
T = \frac12 \cdot DE \cdot CF = \frac{CE\cdot EF\cdot CF}{4R},
$$
hvor $R$ er radius i cirklen.
Der divideres med den fælles faktor $CF$, og resultatet omskrives til:
$$
2 \cdot DE \cdot R = CE \cdot EF
$$
Kvadrering og brug af Pythagoras giver:
$$
4 \cdot DE^2 \cdot R^2 = (CD^2 + DE^2) \cdot (DF^2 + DE^2)
$$
Indsættes nu $DE = 2$, $CD = 6$ og $R = \sqrt{50}$ giver det følgende
ligning til bestemmelse af $DF$:
$$
4\cdot 4\cdot 50 = (6^2 + 2^2)\cdot(DF^2 + 2^2)
$$
Løsning giver
$$
DF = 4
$$

Midtpunktet af $CF$ kaldes $G$ og der gælder så, at $OG\perp CF$ fordi $CF$ er parallel med $AB$. Vi har,
at 
$$
GF = \frac12 CF = \frac12 (CD + DF) = \frac12 (6+4) = 5
$$
Dermed bliver 
$$
GD = GF - DF = 5 - 4 = 1
$$
Dernæst er 
$$
OG = \sqrt{OF^2 - GF^2} = \sqrt{50 - 5^2} = 5
$$
Til sidst har vi:
$$
OD = \sqrt{OG^2 + GD^2} = \sqrt{5^2 + 1^2} = \sqrt{26}.
$$


\end{document}

