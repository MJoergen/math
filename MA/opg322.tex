\documentclass[12pt,oneside,a4paper]{article}

\usepackage[utf8]{inputenc} % Lærer LaTeX at forstå unicode - HUSK at filen skal
% være unicode (UTF-8), standard i Linux, ikke i
% Win.

\usepackage[danish]{babel} % Så der fx står Figur og ikke Figure, Resumé og ikke
% Abstract etc. (god at have).

\usepackage{graphicx}
\usepackage{amsfonts}

%\renewcommand{\mid}[1]{{\rm E}\!\left[#1\right]}
\newcommand{\bas}{\begin{eqnarray*}}
\newcommand{\eas}{\end{eqnarray*}}

\begin{document}

\section*{Løsning til Opgavehjørnet 322}

\subsection*{Opgave}
En sekskant har sidelængder 2, 2, 7, 7, 11, 11. Bestem radius
i den omskrevne cirkel, hvis sekskanten er indskrivelig.

\subsection*{Løsning}
Der gælder følgende sammenhæng for en korde $k$ som spænder
over en centervinkel $v$ i en cirkel med radius $r$:
$$
\sin\left(\frac{v}{2}\right) = \frac{k}{2r}
$$
I sekskanten er der seks centervinkler, som tilsammen udgør 360 grader.
De seks centervinkler er parvis ens og kaldes for $u$, $v$ og $w$.
Vi har derfor, at 
$$
2u+2v+2w = 360
$$
hvor
\bas
\sin\left(\frac{u}{2}\right) &=& \frac{2}{2r} \\
\sin\left(\frac{v}{2}\right) &=& \frac{7}{2r} \\
\sin\left(\frac{w}{2}\right) &=& \frac{11}{2r}
\eas
Dette er fire ligninger med fire ubekendte.

For at lette løsningen introduceres de halve centervinkler $A=u/2$, $B=v/2$ og
$C=w/2$, samt diameteren $x$.
Dette giver:
\bas
\sin A &=& \frac{2}{x} \\
\sin B &=& \frac{7}{x} \\
\sin C &=& \frac{11}{x}
\eas
med
$$
A+B+C=90
$$

Vi benytter nu følgende relation, som gælder for alle vinkler, der opfylder $A+B+C=90$:
$$
\sin^2 A + \sin^2 B + \sin^2 C + 2 \sin A \sin B \sin C = 1
$$
Denne relation bevises senere. Ved indsættelse af ovenstående formler for
$\sin A$ osv får vi
$$
\frac{4}{x^2} + \frac{49}{x^2} + \frac{121}{x^2} + \frac{2\cdot2\cdot7\cdot11}{x^3} = 1
$$
som omskrives til
$$
x^3 - 174\cdot x - 308 = 0
$$

Da diameteren er den længste korde, må vi have $x>11$. For\-tegns\-under\-søgelse
viser, at der netop er én løsning i dette interval. Vi undersøger nu
primfaktorer for konstantleddet: $308 = 2^2\cdot 7\cdot 11$. Den laveste faktor
større end $11$ er $14$. Ved indsættelse ses, at $x=14$ er en rod.

Vi har således fundet diameteren i cirklen. Radius er dermed $r=7$.

\subsection*{Bevis for trigonometrisk relation}
Hvis $A+B+C=90$, så gælder ifølge additionsformlerne
\bas
0 &=& \cos(A+B+C) \\
  &=& \cos A \cos B \cos C - \cos A \sin B \sin C - \sin A \cos B \sin C - \sin A \sin B \cos C
\eas
Vi vil gerne eliminere forekomsten af $\cos{}$ på begge sider.
Det første led rykkes over på venstresiden og der kvadreres:
\bas
&& \cos^2 A \cos^2 B \cos^2 C \\
&=& \left(\cos A \sin B \sin C + \sin A \cos B \sin C + \sin A \sin B \cos C\right)^2 \\
&=& \cos^2 A \sin^2 B \sin^2 C + \sin^2 A \cos^2 B \sin^2 C + \sin^2 A \sin^2 B \cos^2 C \\
&& + 2\sin A\cos A \sin B \cos B \sin^2 C \\
&& + 2\sin A\cos A \sin^2 B \sin C \cos C \\
&& + 2\sin^2 A \sin B \cos B \sin C \cos C 
\eas

De tre første led på højresiden rykkes over på venstresiden, og der reduceres:
\bas
&& \cos^2 A \cos^2 B \cos^2 C - 
\cos^2 A \sin^2 B \sin^2 C - \sin^2 A \cos^2 B \sin^2 C \\
&& - \sin^2 A \sin^2 B \cos^2 C \\
&=& (1-\sin^2 A) (1-\sin^2 B) (1-\sin^2 C) - (1-\sin^2 A) \sin^2 B \sin^2 C \\
&& - \sin^2 A (1-\sin^2 B) \sin^2 C - \sin^2 A \sin^2 B (1-\sin^2 C) \\
&=& 1 - (\sin^2 A + \sin^2 B + \sin^2 C) + 2 \sin^2 A \sin^2 B \sin^2 C 
\eas

Vi har således indtil videre følgende relation:
\bas
&&1 - (\sin^2 A + \sin^2 B + \sin^2 C) + 2 \sin^2 A \sin^2 B \sin^2 C  \\
&=& 2\sin A\cos A \sin B \cos B \sin^2 C \\
&& + 2\sin A\cos A \sin^2 B \sin C \cos C \\
&& + 2\sin^2 A \sin B \cos B \sin C \cos C 
\eas
Højresiden kan reduceres til:
$$
2\sin A\sin B \sin C \left (\cos A \cos B \sin C + \cos A \sin B \cos C + \sin A \cos B \cos C\right)
$$

For at komme videre må vi også bruge additionsformlerne for sinus:
\bas
1 &=& \sin(A+B+C) \\
  &=& \sin A \cos B \cos C + \cos A \sin B \cos C + \cos A \cos B \sin C \\
  &&- \sin A \sin B \sin C
\eas
Heraf følger, at
\bas
&& \cos A \cos B \sin C + \cos A \sin B \cos C + \sin A \cos B \cos C \\
&=& 1 + \sin A \sin B \sin C
\eas

Det hele kombineres nu til:
\bas
&&1 - (\sin^2 A + \sin^2 B + \sin^2 C) + 2 \sin^2 A \sin^2 B \sin^2 C  \\
&=& 2\sin A\sin B \sin C \left ( 1 + \sin A \sin B \sin C \right) 
\eas
Der reduceres nu:
$$
1 - (\sin^2 A + \sin^2 B + \sin^2 C) = 2\sin A\sin B \sin C 
$$
Hermed har vi vist den ønskede relation.

\end{document}

