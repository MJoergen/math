\documentclass[12pt,oneside,a4paper]{article}

\usepackage[utf8]{inputenc} % Lærer LaTeX at forstå unicode - HUSK at filen skal
% være unicode (UTF-8), standard i Linux, ikke i
% Win.

\usepackage[danish]{babel} % Så der fx står Figur og ikke Figure, Resumé og ikke
% Abstract etc. (god at have).

%\usepackage{graphicx}
\usepackage{amsfonts}
\usepackage{amsthm}        % Theorems
\usepackage{amsmath}
%\usepackage{hyperref}

%\renewcommand{\mid}[1]{{\rm E}\!\left[#1\right]}
\newcommand{\bas}{\begin{eqnarray*}}
\newcommand{\eas}{\end{eqnarray*}}
\newcommand{\be}{\begin{equation}}
\newcommand{\ee}{\end{equation}}
\newcommand{\bea}{\begin{eqnarray}}
\newcommand{\eea}{\end{eqnarray}}

\newtheorem{thm}{Sætning}[section]
\newtheorem{mydef}[thm]{Definition}
\newtheorem{eks}[thm]{Eksempel}

\title{Elliptiske funktioner}
\date{August 2015}
\author{Michael Jørgensen}

\begin{document}

\maketitle

Lad der være givet to komplekse tal $\omega_1$ og $\omega_2$, som antages
at være $R$-lineært uafhængige af hinanden. Vi definerer nu et gitter
$L$ bestående af heltallige linearkombinationer af $\omega_1$ og $\omega_2$:
\be
L = \left\{ n_1 \omega_1 + n_2 \omega_2 : n_1, n_2 \in Z\right\}
\ee
Vi indfører nu Weierstrass' P-funktion ved følgende definition:
\be
P(u) = \frac{1}{u^2} + \sum_{\substack{\omega\in L\\ \omega \neq 0}}
  \left(\frac{1}{(u-\omega)^2} - \frac{1}{\omega^2}\right),
  \quad u\not\in L
  \label{pu}
\ee
I det følgende vil vi undersøge egenskaberne af denne funktion.

\section{Indledende observationer}
\begin{thm}
    Funktionen $P(u)$ har følgende elementære egenskaber:
    \begin{enumerate}
        \item Funktionen $P(u)$ er analytisk for alle $u \not\in L$. 
        \item Funktionen $P(u)$ er lige, dvs. $P(u) = P(-u)$.
        \item Funktionen $P(u)$ er dobbelt periodisk med perioderne $\omega_1$ og $\omega_2$, dvs. $P(u) = P(u+\omega_1) = P(u+\omega_2)$.
    \end{enumerate}
\end{thm}
\begin{proof}
    \begin{enumerate}
        \item Følger direkte af definitionen.
        \item Følger af, at gitteret $L$ er symmetrisk omkring $\omega=0$.
        \item Da funktionen er analytisk, så giver ledvis differentiation:
\be
P'(u) = -2 \sum_{\omega\in L} \frac{1}{(u-\omega)^3}, \quad u\not\in L
\ee
Heraf ses, at $P'(u)$ er periodisk over hele gitteret, dvs
\be
P'(u+w) = P'(u),\quad \mbox{for alle $\omega\in L$}
\ee

Ved integration og benyttelse af, at $P(u)$ er lige, så fåes at også 
$P(u)$ selv er periodisk.
    \end{enumerate}
\end{proof}

\section{Taylor}

Vi kan danne en potensudvikling af $P(u)$ omkring $u=0$ ved at benytte:
\be
\frac{1}{(u-\omega)^2} = 
\frac{1}{\omega^2} \cdot \frac{1}{\left(1-\frac{u}{\omega}\right)^2} =
\frac{1}{\omega^2} \sum_{n=0}^\infty 
  (n+1) \left(\frac{u}{\omega}\right)^n
\ee
Indsættes dette i definitionen fåes følgende resultat:
\begin{thm}
    Funktionen P(u) har følgende rækkeudvikling:
\be
P(u) = \frac{1}{u^2} + \alpha_2 u^2 + \alpha_4 u^4 + \cdots,
\ee
hvor
\be
\alpha_n = (n+1) \sum_{\substack{\omega\in L\\ \omega\neq 0}} \frac{1}{\omega^{n+2}},
\quad \mbox{$n>0$, $n$ lige}
\label{alphan}
\ee
\end{thm}
\begin{proof}
    Følger af ovenstående.
\end{proof}

\section{Differentialligning}
Vi kan nu vise følgende hovedresultat:
\begin{thm}
$P(u)$ opfylder følgende simple differentialligning:
\be
\left[P'(u)\right]^2 - 4 \left[P(u)\right]^3
+ 20\,\alpha_2 P(u) + 28\,\alpha_4 = 0,
\label{pmu}
\ee
hvor $\alpha_2$ og $\alpha_4$ er defineret oven for.
\end{thm}
\begin{proof}
Vi definerer
\be
G(u) = \left[P'(u)\right]^2 - 4 \left[P(u)\right]^3
+ 20\,\alpha_2 P(u) + 28\,\alpha_4.
\ee
Ved at benytte rækkeudviklingen for $P(u)$ kan vi finde følgende
resultat for $G(u)$:
$$
G(u) \rightarrow 0 \mbox{ for } u \rightarrow 0.
$$
Dermed er $G(u)$ uden poler, og da $G(u)$ endvidere er dobbeltperiodisk,
så må den være konstant, altså
$$
G(u) = 0, \quad \mbox{for alle $u$.}
$$
\end{proof}

\section{Additionsformel}
Vi vil nu vise følgende additionsformel:
\begin{thm}
    \be
    P(u+v) = \frac{1}{4} \left(\frac{P'(u)-P'(v)}{P(u)-P(v)}\right)^2 - P(u) - P(v)
    \ee
\end{thm}
\begin{proof}
    Vi definerer funktionen $F(u)$ ved:
    \be
    F(u) = P'(u) - A\cdot P(u) - B,
    \ee
    hvor $A$ og $B$ er komplekse konstanter.
    Denne funktion er dobbelt-periodisk med en enkelt pol af tredie orden i $u=0$. Heraf følger, at $F(u)$ har tre (muligvis sammenfaldende) nulpunkter, hvis sum er nul. Kald de tre nulpunkter for $u_1$, $u_2$ og $u_3$. Så haves altså:
    $$
    u_1 + u_2 + u_3 = 0
    \label{u123}
    $$
    samt
    \be
    P'(u_i) = A\cdot P(u_i) + B, \quad i=1,2,3
    \label{pmui}
    \ee

    Indsættes nu denne sidste formel i differentialligningen, og sættes $x=P(u_i)$, så giver det:
    $$
    x^3 - \frac 14 A^2 x^2 - \frac 14 (20 \alpha_2 + 2AB) x - \frac 14 (28\alpha_4 + B^2) = 0
    $$
    Summen af rødderne i dette tredjegradspolynomium er givet ud fra koefficienten til $x^2$. Derfor har vi:
    \be
    P(u_1) + P(u_2) + P(u_3) = \frac 14 A^2.
    \label{pu123}
    \ee

    Konstanten $A$ bestemmes ud fra~(\ref{pmui}) til
    $$
    A = \frac{P'(u_2)-P'(u_1)}{P(u_2)-P(u_1)}
    $$
    Og fra~(\ref{u123}) har vi
    $$
    P(u_3) = P(-u_1-u_2) = P(u_1+u_2)
    $$
    Disse to resultater indsættes i~(\ref{pu123}) og resultatet følger heraf.
\end{proof}
\begin{thm}
    \be
    P(2u) = \frac 14 \left(\frac{P''(u)}{P'(u)}\right)^2 - 2P(u)
    \ee
\end{thm}
\begin{proof}
    Dette følger direkte af grænseovergangen $v\rightarrow u$.
\end{proof}

\section{Halve perioder}
Idet vi definerer $\omega_3 = \omega_1 + \omega_2$ så gælder der:
\begin{thm}
    $$
    P'\left(\frac{\omega_1}{2}\right) =
    P'\left(\frac{\omega_2}{2}\right) =
    P'\left(\frac{\omega_3}{2}\right) = 0
    $$
\end{thm}
\begin{proof}
    Dette følder af, at $P'(u)$ er en ulige funktion.
\end{proof}

\begin{thm}
    Værdierne $P(\frac{\omega_1}{2})$, $P(\frac{\omega_2}{2})$ og $P(\frac{\omega_3}{2})$
    er alle nulpunkter til polynomiet
    $$
    4\,x^3 - 20\,\alpha_2 \,x - 28 \,\alpha_4 = 0
    $$
    Specielt gælder der, at 
    $$
    P\left(\frac{\omega_1}{2}\right) + P\left(\frac{\omega_2}{2}\right) + P\left(\frac{\omega_3}{2}\right) = 0
    $$
\end{thm}
\begin{proof}
    Dette følger af ligning~(\ref{pmu}).
\end{proof}

\section{Omvendt funktion}
I det følgende specialiserer vi til, at $\omega_1$ er reel og $\omega_2$ er
rent imaginær.  Vi ønsker nu at invertere ligning~(\ref{pu}). Specielt ønsker
vi ud fra
kendskab til $P\left(\frac{\omega_1}{2}\right)$,
$P\left(\frac{\omega_2}{2}\right)$ og $P\left(\frac{\omega_3}{2}\right)$ at
bestemme $\omega_1$ og $\omega_2$, hvor vi stadigvæk har, at $\omega_3 = \omega_1 + \omega_2$.

Idet vi
definerer funktionen
\be
I(a, b) = \int_0^\infty \frac{dt}{\sqrt{(a^2+t^2)(b^2+t^2)}},
\ee
samt konstanterne
\bea
a &=& \sqrt{P\left(\frac{\omega_1}{2}\right)-P\left(\frac{\omega_2}{2}\right)} \\
b &=& \sqrt{P\left(\frac{\omega_1}{2}\right)-P\left(\frac{\omega_3}{2}\right)} \\
c &=& \sqrt{P\left(\frac{\omega_3}{2}\right)-P\left(\frac{\omega_2}{2}\right)}
\eea
så vil vi bevise følgende resultat:
\begin{thm}
    Hvis $\omega_1$ er reel og $\omega_2$ er rent imaginær, så er $a$, $b$ og $c$ alle reelle.
    Endvidere er 
    \bea
    \omega_1 &=& 2 I(a,b) \\
    \omega_2 &=& 2i I(a,c) \\
    \omega_3 &=& 2i I(b,ic)
    \eea
\end{thm}
\begin{proof}
    Først skal man vise, at 
    $P\left(\frac{\omega_1}{2}\right)$,
    $P\left(\frac{\omega_2}{2}\right)$ og
    $P\left(\frac{\omega_3}{2}\right)$ alle er reelle.
    Dernæst skal man vise, at 
    $$
    P\left(\frac{\omega_2}{2}\right) <
    P\left(\frac{\omega_3}{2}\right) <
    P\left(\frac{\omega_1}{2}\right)
    $$.
    Så kan man benytte differentialligningen til at vise, at
    $$
    \omega_1 = 2 \int_{P\left(\frac{\omega_1}{2}\right)}^\infty 
    \frac{dx}{\sqrt{4x^3-20\alpha_2x-28\alpha_4}}
    $$
    Til sidst indføres substitutionen $x = P\left(\frac{\omega_1}{2}\right) + t^2$.
    
\end{proof}
\end{document}


